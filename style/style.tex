


%%%%%%%%%%%%%%%%%%%%%%%%%%%%%%%%%%%%%%%%%%%%%%%%%%%%%%%%%%%%%%%%%%%%%%%%%%%%%%%%
% Packages
%%%%%%%%%%%%%%%%%%%%%%%%%%%%%%%%%%%%%%%%%%%%%%%%%%%%%%%%%%%%%%%%%%%%%%%%%%%%%%%%

\usepackage{subfiles}
\usepackage[margin=1.5in]{geometry}

% deutsche Sprache
\usepackage[utf8]{inputenc}
\usepackage[T1]{fontenc}
\usepackage[ngerman]{babel}

% Literaturverzeichnis
\usepackage[backend=biber, style=alphabetic]{biblatex}

\usepackage[pdftex]{graphicx}
\usepackage{latexsym}
\usepackage{amsmath,amssymb,amsthm}
\usepackage{tikz} % \foreach-Befehl
\usepackage{hyperref} % Verweise
\usepackage{lmodern} % keine pixelige Schrift
\usepackage{mathtools} % defby
\usepackage{enumitem} % a b c enumeration
\setlist{nosep} % no vertical space between list items
\usepackage{dsfont} % indicator function
\usepackage{mdframed} % framed theorems
\usepackage{csquotes}




%%%%%%%%%%%%%%%%%%%%%%%%%%%%%%%%%%%%%%%%%%%%%%%%%%%%%%%%%%%%%%%%%%%%%%%%%%%%%%%%
% weitere Einstellungen
%%%%%%%%%%%%%%%%%%%%%%%%%%%%%%%%%%%%%%%%%%%%%%%%%%%%%%%%%%%%%%%%%%%%%%%%%%%%%%%%


% Abstand obere Blattkante zur Kopfzeile ist 2.54cm - 15mm
% \setlength{\topmargin}{-15mm}




%%%%%%%%%%%%%%%%%%%%%%%%%%%%%%%%%%%%%%%%%%%%%%%%%%%%%%%%%%%%%%%%%%%%%%%%%%%%%%%%
% Umgebungen
%%%%%%%%%%%%%%%%%%%%%%%%%%%%%%%%%%%%%%%%%%%%%%%%%%%%%%%%%%%%%%%%%%%%%%%%%%%%%%%%
	
\newmdtheoremenv[backgroundcolor=black!3, innertopmargin=-2pt]{Definition}{Definition}[section]
\newmdtheoremenv[backgroundcolor=black!3, innertopmargin=-2pt]{Satz}[Definition]{Satz}
\newmdtheoremenv[backgroundcolor=black!3, innertopmargin=-2pt]{Hilfssatz}[Definition]{Hilfssatz}   

\newtheorem*{Bemerkung}{Bemerkung}

\numberwithin{equation}{section} 




%%%%%%%%%%%%%%%%%%%%%%%%%%%%%%%%%%%%%%%%%%%%%%%%%%%%%%%%%%%%%%%%%%%%%%%%%%%%%%%%
% Text-Commands
%%%%%%%%%%%%%%%%%%%%%%%%%%%%%%%%%%%%%%%%%%%%%%%%%%%%%%%%%%%%%%%%%%%%%%%%%%%%%%%%

\newcommand{\tobechanged}[1]{\textbf{\textcolor{red}{#1}}}




%%%%%%%%%%%%%%%%%%%%%%%%%%%%%%%%%%%%%%%%%%%%%%%%%%%%%%%%%%%%%%%%%%%%%%%%%%%%%%%%
% Mathe-Commands und -Umgebungen
%%%%%%%%%%%%%%%%%%%%%%%%%%%%%%%%%%%%%%%%%%%%%%%%%%%%%%%%%%%%%%%%%%%%%%%%%%%%%%%%



\newcommand{\C}{\mathbb{C}} % komplexe
\newcommand{\K}{\mathbb{K}} % komplexe
\newcommand{\R}{\mathbb{R}} % reelle
\newcommand{\Q}{\mathbb{Q}} % rationale
\newcommand{\Z}{\mathbb{Z}} % ganze
\newcommand{\N}{\mathbb{N}} % natuerliche
\newcommand{\defby}{\vcentcolon =}
\newcommand{\Set}[1]{\ensuremath{\left\{\, #1 \,\right\}}}
\newcommand{\set}[1]{\ensuremath{\left\{#1\right\}}}
\newcommand{\setcomp}[2]{\Set{#1 \; | \; #2}}
\newcommand{\fctmap}[5]{\ensuremath{#1 \colon #2 \to #3, \; #4 \mapsto #5 }}
\newcommand{\fct}[3]{\ensuremath{#1 \colon #2 \to #3}}
\newcommand{\indfct}{\mathds{1}}
\newcommand\restr[2]{\ensuremath{\left.#1\right|_{#2}}}
\newcommand{\setc}[1]{{#1}^\mathsf{c}}
\newcommand{\convdown}{\downarrow}
\newcommand{\Probmeasures}[1]{\mathcal{M}_{+,1}(#1)}
\newcommand{\Bdcontfct}[1]{C_b(#1)}
\newcommand{\measureintx}[4]{\int_{#1} \, #2 \, #3(\mathrm{d}#4)}
\newcommand{\measureint}[3]{\int_{#1} \, #2 \, \mathrm{d}#3}

\newenvironment{enumeratethm}{\vspace*{0.5em} \begin{enumerate}[label=(\alph*)]}{\end{enumerate} \vspace*{0.5em}}
\newenvironment{equivalentthm}{\vspace*{0.5em} \begin{enumerate}[label=(\roman*)]}{\end{enumerate} \vspace*{0.5em}}



