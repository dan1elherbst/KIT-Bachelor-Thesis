


%%%%%%%%%%%%%%%%%%%%%%%%%%%%%%%%%%%%%%%%%%%%%%%%%%%%%%%%%%%%%%%%%%%%%%%%%%%%%%%%
% Packages
%%%%%%%%%%%%%%%%%%%%%%%%%%%%%%%%%%%%%%%%%%%%%%%%%%%%%%%%%%%%%%%%%%%%%%%%%%%%%%%%

\usepackage{subfiles}

% deutsche Sprache
\usepackage[utf8]{inputenc}
\usepackage[T1]{fontenc}
\usepackage[ngerman]{babel}

% Literaturverzeichnis
\usepackage[backend=biber, style=alphabetic]{biblatex}

\usepackage{latexsym}
\usepackage{amsmath,amssymb,amsthm}
\usepackage{tikz} % \foreach-Befehl
\usepackage{hyperref} % Verweise
\usepackage{lmodern} % keine pixelige Schrift
\usepackage{mathtools} % defby
\usepackage{enumitem} % a b c enumeration
\setlist{nosep} % no vertical space between list items
\usepackage{dsfont} % indicator function
\usepackage{mdframed} % framed theorems
\usepackage{csquotes}
\usepackage{tikz}
\usepackage{pgfplots}
\usepackage{subcaption}
\usepackage{float}
\newcommand{\sectionbreak}{\clearpage}
\usepackage[margin=1cm]{caption}


%%%%%%%%%%%%%%%%%%%%%%%%%%%%%%%%%%%%%%%%%%%%%%%%%%%%%%%%%%%%%%%%%%%%%%%%%%%%%%%%
% weitere Einstellungen
%%%%%%%%%%%%%%%%%%%%%%%%%%%%%%%%%%%%%%%%%%%%%%%%%%%%%%%%%%%%%%%%%%%%%%%%%%%%%%%%


% Abstand obere Blattkante zur Kopfzeile ist 2.54cm - 15mm
% \setlength{\topmargin}{-15mm}




%%%%%%%%%%%%%%%%%%%%%%%%%%%%%%%%%%%%%%%%%%%%%%%%%%%%%%%%%%%%%%%%%%%%%%%%%%%%%%%%
% Umgebungen
%%%%%%%%%%%%%%%%%%%%%%%%%%%%%%%%%%%%%%%%%%%%%%%%%%%%%%%%%%%%%%%%%%%%%%%%%%%%%%%%
    

\theoremstyle{definition}
\newmdtheoremenv[backgroundcolor=black!3, innertopmargin=-2pt]{Definition}{Definition}[chapter]
\newtheorem*{Beispiel}{Beispiel}

\theoremstyle{plain}
\newmdtheoremenv[backgroundcolor=black!3, innertopmargin=-2pt]{Satz}[Definition]{Satz}
\newtheorem{Hilfssatz}[Definition]{Hilfssatz}
\newmdtheoremenv[backgroundcolor=black!3, innertopmargin=-2pt]{Folgerung}[Definition]{Folgerung}

\theoremstyle{remark}
\newtheorem*{Bemerkung}{Bemerkung}



\numberwithin{equation}{section} 




%%%%%%%%%%%%%%%%%%%%%%%%%%%%%%%%%%%%%%%%%%%%%%%%%%%%%%%%%%%%%%%%%%%%%%%%%%%%%%%%
% Text-Commands
%%%%%%%%%%%%%%%%%%%%%%%%%%%%%%%%%%%%%%%%%%%%%%%%%%%%%%%%%%%%%%%%%%%%%%%%%%%%%%%%

\newcommand{\tobechanged}[1]{\textbf{\textcolor{red}{#1}}}




%%%%%%%%%%%%%%%%%%%%%%%%%%%%%%%%%%%%%%%%%%%%%%%%%%%%%%%%%%%%%%%%%%%%%%%%%%%%%%%%
% Mathe-Commands und -Umgebungen
%%%%%%%%%%%%%%%%%%%%%%%%%%%%%%%%%%%%%%%%%%%%%%%%%%%%%%%%%%%%%%%%%%%%%%%%%%%%%%%%



\newcommand{\C}{\mathbb{C}} % komplexe
\newcommand{\K}{\mathbb{K}} % komplexe
\newcommand{\R}{\mathbb{R}} % reelle
\newcommand{\Q}{\mathbb{Q}} % rationale
\newcommand{\Z}{\mathbb{Z}} % ganze
\newcommand{\N}{\mathbb{N}} % natuerliche
\newcommand{\X}{\mathcal{X}}
\newcommand{\Y}{\mathcal{Y}}
\newcommand{\E}{\mathbb{E}}
\newcommand{\defby}{\vcentcolon =}
\newcommand{\Set}[1]{\ensuremath{\left\{\, #1 \,\right\}}}
\newcommand{\set}[1]{\ensuremath{\left\{#1\right\}}}
\newcommand{\setcomp}[2]{\Set{#1 \; \middle| \; #2}}
\newcommand{\fctmap}[5]{\ensuremath{#1 \colon #2 \to #3, \; #4 \mapsto #5 }}
\newcommand{\fct}[3]{\ensuremath{#1 \colon #2 \to #3}}
\newcommand{\indfct}{\mathds{1}}
\newcommand{\restr}[2]{\ensuremath{\left.#1\right|_{#2}}}
\newcommand{\setc}[1]{{#1}^\mathsf{c}}
\newcommand{\convdown}{\downarrow}
\newcommand{\Probmeasures}[1]{\mathcal{P}(#1)}
\newcommand{\Unitmeasures}[1]{\mathcal{M}_{+,\leq 1}(#1)}
\newcommand{\Finitemeasures}[1]{\mathcal{M}_{+}(#1)}
\newcommand{\Bdcontfct}[1]{C_b(#1)}
\newcommand{\Bduniffct}[1]{U_b(#1)}
\newcommand{\Bdlipschitzfct}[1]{\mathrm{Lip}_b(#1)}
\newcommand{\Lipschitzlequnit}[1]{\mathrm{Lip}_1(#1)}
\newcommand{\measureintx}[4]{\int_{#1} \, #2 \, \mathrm{d}#3(#4)}
\newcommand{\measureint}[3]{\int_{#1} \, #2 \, \mathrm{d}#3}
\newcommand{\abs}[1]{\left| #1 \right|}
\newcommand{\absl}[1]{\left| \, #1 \,\right|}
\newcommand{\norm}[1]{\left\lVert#1\right\rVert}
\newcommand{\norml}[1]{\left\lVert \, #1\, \right\rVert}
\newcommand{\sgn}{\mathrm{sgn}}

\newcommand{\Wasprobmeasures}[2]{\mathcal{P}_{#1}(#2)}

% disjoint unions
\makeatletter
\def\moverlay{\mathpalette\mov@rlay}
\def\mov@rlay#1#2{\leavevmode\vtop{%
		\baselineskip\z@skip \lineskiplimit-\maxdimen
		\ialign{\hfil$\m@th#1##$\hfil\cr#2\crcr}}}
\newcommand{\charfusion}[3][\mathord]{
	#1{\ifx#1\mathop\vphantom{#2}\fi
		\mathpalette\mov@rlay{#2\cr#3}
	}
	\ifx#1\mathop\expandafter\displaylimits\fi}
\makeatother

\newcommand{\cupdot}{\charfusion[\mathbin]{\cup}{\cdot}}
\newcommand{\bigcupdot}{\charfusion[\mathop]{\bigcup}{\cdot}}

\newenvironment{itemizethm}{\vspace*{0.5em} \begin{itemize}}{\end{itemize} \vspace*{0.5em}}
\newenvironment{enumeratethm}{\vspace*{0.5em} \begin{enumerate}[label=(\alph*)]}{\end{enumerate} \vspace*{0.5em}}
\newenvironment{equivalentthm}{\vspace*{0.5em} \begin{enumerate}[label=(\roman*)]}{\end{enumerate} \vspace*{0.5em}}



%pgfplots and tikz

%gauss(x, mu, sigma)
\pgfmathdeclarefunction{gauss}{3}{%
	\pgfmathparse{1/(#3*sqrt(2*pi))*exp(-((#1-#2)^2)/(2*#3^2))}%
}
%gauss2d(x, y, mu, nu, sigma, tau, cor)
\pgfmathdeclarefunction{gauss2d}{7}{%
	\pgfmathparse{1 / (2 * pi * #5 * #6 * sqrt(1 - #7^2)) * exp(- 1 / (2 * #5^2 * #6^2 * (1 - #7^2)) * (#6^2 * (#1 - #3)^2 + 2 * #7 * #5 * #6 * (#1 - #3) * (#2 - #4) + #5^2 * (#2 - #4)^2))}%
}
\pgfplotsset{grid style={dotted,gray}}
\pgfplotsset{compat=1.3}
\usetikzlibrary{3d}



