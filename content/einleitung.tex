\documentclass[../main/main.tex]{subfiles}

\begin{document}
	
	\section{Einleitung}
	
	\tobechanged{Hier steht eine Einführung}
	
	\begin{Definition}[Polnischer Raum]
		Ein polnischer Raum ist ein separabler topologischer Raum $(X, \mathcal{O})$, 
		dessen Topologie von einer Metrik, bezüglich der $X$ vollständig ist, erzeugt wird.
	\end{Definition}

	\begin{Definition}[Borelsche $\sigma$-Algebra]
		\label{def:borel}
		Sei $(X, \mathcal{O})$ ein topologischer Raum. Dann definieren wir die Borelsche $\sigma$-Algebra
		$$\mathcal{B} \defby \sigma(\mathcal{O})$$
		über $X$. Ferner sei
		$$\Probmeasures{X} \defby \setcomp{\fct{\mu}{\mathcal{B}}{[0,1]}}{\mu \; \text{ist Wahrscheinlichkeitsmaß}}\text{.}$$
	\end{Definition}

	\begin{Definition}[Schwache Regularität von Maßen]
		\label{def:regularity}
		In der Situation von Definition~\ref{def:regularity}, wobei zusätzlich $\mu$ ein 
		endliches Maß auf $\mathcal{B}$ sei, nennen wir $B \in \mathcal{B}$ schwach von 
		innen bzw. von außen regulär, falls
		$$\mu(B) = \sup_{\substack{C \subseteq B \\ C^\mathsf{c} \in \mathcal{O}}} \mu(C) \quad \text{bzw.} \quad \mu(B) = \inf_{\substack{U \supseteq B \\ U \in \mathcal{O}}} \mu(U)\text{.}$$
		Wir nennen $B \in \mathcal{B}$ schwach regulär, wenn $B$ schwach von innen und 
		von außen regulär ist. Sind alle $B \in \mathcal{B}$ schwach regulär, 
		so nennen wir $\mu$ ein schwach reguläres Maß.
	\end{Definition}

	\begin{Satz}
		\label{thm:weakregularity}
		Ist $(X, d)$ ein metrischer Raum, so ist jedes $\mu \in \Probmeasures{X}$ 
		schwach regulär.
	\end{Satz}

	Für den Beweis des Satzes benötigen wir zunächst zwei Hilfssätze.
	
	\begin{Hilfssatz}
		\label{lem:sigmaalg}
		In der Situation von Definition~\ref{def:regularity} ist
		$$\mathcal{S} \defby \setcomp{B \in \mathcal{B}}{B \text{ schwach regulär bzgl. } \mu}$$
		eine $\sigma$-Algebra.
	\end{Hilfssatz}
	
	\begin{proof}
		$\emptyset, X \in \mathcal{S}$ ist offensichtlich. Jedes $B \in \mathcal{S}$ ist schwach regulär 
		von innen und von außen und daher gilt wegen $\mu(X) < \infty$
		$$\mu(B^\mathsf{c}) \, = \, \mu(X) - \mu(B) \, = \, \mu(X) - \sup_{\substack{C \subseteq B \\ C^\mathsf{c} \in \mathcal{O}}} \mu(C) \, = \, \inf_{\substack{U \supseteq B^\mathsf{c} \\ U \in \mathcal{O}}} \mu(U)$$
		sowie analog
		$$\mu(B^\mathsf{c}) \, = \, \mu(X) - \mu(B) \, = \, \mu(X) - \inf_{\substack{U \supseteq B \\ U \in \mathcal{O}}} \mu(C) \, = \, \sup_{\substack{C \subseteq B^\mathsf{c} \\ C^\mathsf{c} \in \mathcal{O}}} \mu(U) \text{,}$$
		womit die schwache Regularität von $B^\mathsf{c}$ folgt. 
		
		Es bleibt also zu zeigen, dass für $(B_n)_n \in \mathcal{S}^\N$ auch $B \defby \bigcup_{n \in \N} B_n$ 
		schwach regulär ist. Hierfür zeigen zunächst die schwache Regularität von innen. 
		Sei dazu $\varepsilon > 0$ und wähle für $n \in \N$ jeweils abgeschlossene Mengen 
		$C_n \subseteq B_n$ mit $\mu(B_n) - \mu(C_n) < \frac{\varepsilon}{3^n}$.
		Wenn wir nun $N$ so groß wählen, dass $\mu\left( B \setminus \bigcup_{n=1}^N B_n \right) < \frac{\varepsilon}{2}$ 
		(was wegen $\mu(B) < \infty$ immer geht), so gilt mit $C := \bigcup_{n=1}^N C_k$, dass 
		\begin{align*}
			\mu(B) - \mu(C) = \mu(B\setminus C) \; &=    \; \mu\left( \left( B \setminus \bigcup_{n=1}^N B_n \right) \; \cup \; \left( \bigcup_{n=1}^N B_n  \setminus C \right) \right) \\
			&\leq \; \mu \left( B \setminus \bigcup_{n=1}^N B_n \right) + \sum_{n=1}^{N} \mu(B_n \setminus C_n) \\
			&<    \; \frac{\varepsilon}{2} + \sum_{n=1}^{\infty} \frac{\varepsilon}{3^n} \; = \; \varepsilon
		\end{align*}
		und weil $C$ abgeschlossen ist, folgt, dass $B$ schwach regulär von innen ist.
		
		Wählen wir nun für alle $n$ offene Mengen $U_n \supseteq B_n$ so, 
		dass $\mu(U_n) - \mu(B_n) < \frac{\varepsilon}{2^n}$, und setzen $U := \bigcup_{n \in \N} U_n$, so gilt
		$$\mu(U) - \mu(B) \; \leq \; \sum_{n=1}^\infty \mu(U_n \setminus B) \; \leq \; \sum_{n=1}^\infty \mu(U_n \setminus B_n) \; < \; \varepsilon$$
		und weil $U$ offen ist, folgt insgesamt die schwache Regularität von $\mu$.
	\end{proof}
	
	\begin{Hilfssatz}
		\label{lem:opensets}
		Sei $(X, d)$ ein metrischer Raum und $C \subseteq X$ eine abgeschlossene 
		Teilmenge. Ferner definieren wir für $n \in \N$
		$$ A_n \defby \setcomp{y \in X}{d(y, C) < \frac{1}{n}} \quad \text{und} \quad \fctmap{f_n}{X}{\R}{x}{\max \set{0, 1-n d(x, C)}} \text{,}$$
		wobei wir $d(y, C) \defby \inf_{x \in C} d(y, x)$ setzen.
		Dann gilt:
		\begin{enumeratethm}
			\item $A_n$ ist offen für alle $n \in \N$.
			\item $C = \bigcap_{n \in \N} A_n$, insbesondere ist $C$ also eine $G_\delta$-Menge.
			\item Für alle $n$ ist $\restr{f_n}{A_n^\mathsf{c}} = 0$ und $f_n$ ist gleichmäßig stetig.
			\item $f_n \convdown \indfct_C$.
		\end{enumeratethm}
	\end{Hilfssatz}

	\begin{proof}
		zu (a): Wir wählen $y \in A_n$, also $y \in X, \; d(y, C) < \frac{1}{n}$. 
		Mit $r_n \defby \frac{1}{n} - d(y, C) > 0$
		gilt dann für alle $x \in X, \; d(x, y) < r_n$:
		$$d(x, C) \leq d(x, y) + d(y, C) < \frac{1}{n} \text{,}$$
		und damit gilt $B_{r_n}(y) \subseteq A_n$ und $A_n$ ist offen.
		
		zu (b): Sicherlich ist $C \subseteq A_n$ für alle $n \in \N$ und damit 
		$C \subseteq \bigcap_{n \in \N} A_n$. 
		Umgekehrt ist für ein beliebiges $y \in \bigcap_{n \in \N} A_n$
		$$d(y, C) = \inf_{x \in C} d(y, x) = 0 \text{,}$$
		was impliziert, dass es eine Folge $(x_n)_n \in C^\N$ gibt mit $x_n \rightarrow y$, 
		und wegen der Abgeschlossenheit von $C$ ist damit $y \in C$, also auch $\bigcap_{n \in \N} A_n \subseteq C$.
		
		zu (c): Für $x \in A_n^\mathsf{c}$ ist $d(x, C) \geq \frac{1}{n}$ und damit $1 - nd(x, C) \leq 0$, 
		also $\restr{f_n}{A_n^\mathsf{c}} = 0$. Außerdem gilt für beliebige $x, y \in X$:
		\begin{align*}
			| f_n(x) - f_n(y) | \; &= \; | \max \set{0, 1-n d(x, C)} - \max \set{0, 1-n d(y, C)} | \\
								   &= \; \begin{cases}
								   	n | d(y, C) - d(x, C) | & \quad d(x, C) < \frac{1}{n} \; \text{und} \; d(y, C) < \frac{1}{n} \\
								   	1 - nd(x, C)            & \quad d(x, C) < \frac{1}{n} \; \text{und} \; d(y, C) \geq \frac{1}{n} \\
								   	1 - nd(y, C)            & \quad d(x, C) \geq \frac{1}{n} \; \text{und} \; d(y, C) < \frac{1}{n} \\
								   	0                       & \quad d(x, C) \geq \frac{1}{n} \; \text{und} \; d(y, C) \geq \frac{1}{n}
								   \end{cases} \\
							      &\leq \; n | d(y, C) - d(x, C) | \; \leq \; nd(x, y)
		\end{align*}
		und damit ist $f_n$ lipschitzstetig, also auch gleichmäßig stetig.
		
		zu (d): Offensichtlich ist $(f_n)_n$ fallend. Für $x \in C^\mathsf{c}$ ist $d(x, C) > 0$ und damit
		$$f_n(x) = \max \set{0, 1-n d(x, C)} \; \xrightarrow{n \to \infty} \; 0 \text{,}$$
		für $x \in C$ ist dagegen $f_n(x) = 1$ konstant, womit die Behauptung folgt.
	\end{proof}

	\begin{proof}[Beweis von Satz~\ref{thm:weakregularity}]
		Es ist nun zu zeigen, dass für jedes $\mu \in \Probmeasures{X}$
		die Menge 
		$$\mathcal{S} \defby \setcomp{B \in \mathcal{B}}{B \text{ schwach regulär bzgl. } \mu}$$
		bereits ganz $\mathcal{B}$ ist. 
		Da nach Hilfssatz~\ref{lem:sigmaalg} $\mathcal{S}$ eine $\sigma$-Algebra ist und $\mathcal{B}$ von den abgeschlossenen Mengen erzeugt wird, genügt es zu zeigen, dass diese in $\mathcal{S}$
		enthalten sind. 
		
		Jedes abgeschlossene $C \in \mathcal{B}$ ist sicherlich schwach regulär von innen. Nun wählen wir $A_n, \; n \in \N$ wie in Hilfssatz~\ref{lem:opensets}. Wegen $A_n \convdown C$ und $\mu(X) < \infty$
		folgt mit der Maßstetigkeit von oben, dass
		$$\mu(A_n) \convdown \mu(C)$$
		und da alle $A_n$ offen sind, ist $C$ auch schwach regulär von außen.
		
		Insgesamt folgt also $\mathcal{S} = \mathcal{B}$ und damit die Behauptung.
	\end{proof}

	\begin{Satz}
		Sei $(X,d)$ ein metrischer Raum und seien $\mu, \nu \in \Probmeasures{X}$. Dann sind äquivalent:
		\begin{equivalentthm}
			\item $\mu = \nu$
			\item Für alle gleichmäßig stetigen Funktionen $\fct{f}{X}{\R}$ ist $\measureint{}{f}{\mu} = \measureint{}{f}{\nu}$
			\item Für alle abgeschlossenen Mengen $C \in \mathcal{B}$ ist $\mu(C) = \nu(C)$.
		\end{equivalentthm}
	\end{Satz}

	\begin{proof}
		(i) $\Rightarrow$ (ii) ist klar.
		
		(ii) $\Rightarrow$ (iii): Gelte (ii) und sei $C \subseteq X$ abgeschlossen. Dann gilt für die gleichmäßig stetigen Funktionen $f_n$ aus Hilfssatz~\ref{lem:opensets}
		$$\measureint{}{f_n}{\mu} = \measureint{}{f_n}{\nu}\text{,} \quad n \in \N \text{.}$$
		Wegen $| f_n | \leq 1$ und $f_n \convdown \indfct_C$ folgt mit dem Satz von Lebesgue 
		$$\measureint{}{f_n}{\mu} \; \to \; \mu(C) \quad \text{und} \quad \measureint{}{f_n}{\nu} \; \to \; \nu(C)$$
		und damit gilt (iii).
		
		(iii) $\Rightarrow$ (i): Da $\mu$ und $\nu$ nach Satz~\ref{thm:weakregularity} schwach regulär sind, folgt die Behauptung ebenfalls direkt.
	\end{proof}

	\begin{Definition}[Schwache Konvergenz von Maßen]
		Ist $(\mu_n)_n \in \Probmeasures{X}$ eine Folge, so sagen wir, dass $(\mu_n)_n$ schwach gegen $\mu \in \Probmeasures{X}$ konvergiert, falls
		für alle $f \in C(X)$, den stetigen beschränkten Funktionen, gilt, dass
		$$\measureint{}{f}{\mu_n} \; \xrightarrow{n \to \infty} \; \measureint{}{f}{\mu} \text{.}$$
	\end{Definition}

	\begin{Satz}[Portmanteau]
		Sei $(X, d)$ ein metrischer Raum, so sind für $(\mu_n)_n \in \Probmeasures{X}^\N$ und $\mu \in \Probmeasures{X}$ äquivalent:
		\begin{equivalentthm}
			\item $(\mu_n)_n$ konvergiert schwach gegen $\mu$
			\item Für alle abgeschlossenen Mengen $C \subseteq X$ gilt: $$\limsup_{n \to \infty} \mu_n(C) \; \leq \; \mu(C)$$
			\item Für alle offenen Mengen $U \subseteq X$ gilt: $$\liminf_{n \to \infty} \mu_n(U) \; \geq \; \mu(U)$$
			\item Für alle $B \in \mathcal{B}$ mit $\mu(\partial B) = 0$ gilt $$\lim_{n \to \infty} \mu_n(A) \; = \; \mu(A) \text{.}$$
		\end{equivalentthm}
	\end{Satz}

	
	
\end{document}

