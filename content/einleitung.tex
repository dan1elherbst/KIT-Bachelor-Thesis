\documentclass[../thesis/thesis.tex]{subfiles}

\begin{document}
	
	\chapter{Einleitung}
	
	Die \emph{Verteilungskonvergenz} ist ein recht schwacher Konvergenzbegriff für reelle Zufallsvariablen bzw. Wahrscheinlichkeitsmaße, der
	aber dennoch sehr häufig in der Wahrscheinlichkeitstheorie und insbesondere auch in der Statistik verwendet wird -- nicht zuletzt
	wegen der berühmten \emph{zentralen Grenzwertwertsätze}, die aussagen, dass unter bestimmten Bedingungen (abhängig von der Version)
	eine normalisierte Summe von reellen Zufallsvariablen \emph{in Verteilung} gegen eine Standardnormalverteilung konvergiert.
	
	In dieser Arbeit werden wir mit der \emph{schwachen Konvergenz} zunächst eine Möglichkeit vorstellen, das Konzept der \emph{Verteilungskonvergenz} 
	auf die Wahrscheinlichkeitsmaße $\Probmeasures{\X}$ bzw. allgemeiner endliche Borel-Maße $\Finitemeasures{\X}$ eines 
	beliebigen topologischen Raums $\X$ auszuweiten. Hierzu führen wir eine Topologie
	auf $\Finitemeasures{\X}$ ein, die sogenannte \emph{schwache Topologie}, deren Konvergenz gerade der schwachen Konvergenz entspricht.
	Daraufhin werden in der Arbeit zunächst primär Resultate präsentiert, die sich mit den topologischen Eigenschaften 
	von $\Finitemeasures{\X}$ beschäftigen -- und insbesondere auch den Beziehungen zur Topologie von $\X$ selbst.
	
	Eine Schlüsselrolle kommt hierbei der Klasse der \emph{polnischen Räume} zu, die alle topologischen Räume umfasst, welche separabel und vollständig 
	metrisierbar sind. Offenbar handelt es sich bei der Polnizität eines Raumes um eine sehr einfach formulierbare und zugleich allgemeine Eigenschaft. 
	Wir werden in dieser Arbeit sehen, dass die Forderung der Polnizität 
	eines topologischen Raumes $\X$ aber dennoch einige nichttriviale Erkenntnisse über die Beschaffenheit von $\Finitemeasures{\X}$ erlaubt -- und umgekehrt.
	Die zentralen Resultate, die in der Arbeit vorgestellt werden, sind unter anderem
	\begin{itemizethm}
		\item dass für metrisierbares $\X$ die Polnizität von $\Finitemeasures{\X}$ bezüglich der schwachen Topologie
		gleichbedeutend zur Polnizität von $\X$ selbst ist.
		\item der \emph{Satz von Prokhorov}, der unter der Annahme der Polnizität von $\X$ eine Charakterisierung der Kompaktheit 
		von Teilmengen von $\Probmeasures{\X}$ bereitstellt.
	\end{itemizethm}
	Im Anschluss beschäftigt sich die Arbeit mit sogenannten \emph{Wassersteinmetriken}, die im Falle von polnischem $\X$ eine Klasse von 
	besonders intuitiven Metriken darstellen, die -- unter gewissen Voraussetzungen -- die schwache Topologie metrisieren. 
	Hierfür werden insbesondere vorher auch ein paar Grundlagen aus der Theorie des optimalen Transports wiedergegeben, in deren Rahmen sich 
	die Wassersteinmetriken motivieren, formulieren und analysieren lassen.
	
	Die Arbeit setzt grundlegende Kenntnisse in (Funktional-)analysis, Maßtheorie und Topologie, wie sie am Karlsruher 
	Institut für Technologie etwa in den Vorlesungen
	\enquote{Analysis 1-3}, \enquote{Funktionalanalysis} sowie \enquote{Elementare Geometrie} gelehrt werden, voraus und gliedert sich wie folgt:
	\begin{itemizethm}
		\item In Kapitel~\ref{chap:grundlagen} werden zunächst ein paar topologische und maßtheoretische Grundlagen vorgestellt.
		\item Kapitel~\ref{chap:polnische_räume} beschäftigt sich mit Grundlagen der Klasse aller polnischen Räume.
		\item Wir definieren in Kapitel~\ref{chap:schwache_topologie} die schwache Topologie und charakterisieren diese.
		\item Die zentralen Eigenschaften der schwachen Topologie werden in Kapitel~\ref{chap:eigenschaften_der_schwachen_topologie}
		formuliert und bewiesen.
		\item Eine Motivation der Wassersteinmetriken sowie deren Diskussion im Hinblick auf die schwache Topologie findet sich in Kapitel~\ref{chap:wassersteinmetriken}.
	\end{itemizethm}

	Große Teile von Kapitel~\ref{chap:grundlagen} bis \ref{chap:eigenschaften_der_schwachen_topologie} orientieren sich primär an \cite[Section 4.14]{Simon.2015}, 
	wobei wir insbesondere für die Abschnitte~\ref{subsec:einbettung_des_grundraums} und \ref{subsec:beziehung_zu_eigenschaften_des_grundraums} auf \cite{Varadarajan.1958}
	zurückgreifen, da dort allgemeinere und stärkere Resultate als in \cite{Simon.2015} bewiesen werden. Das Kapitel zu Wassersteinmetriken (Kapitel~\ref{chap:wassersteinmetriken})
	folgt überwiegend \cite[Chapter 4-6]{Villani.2009}.
	
	%\newpage
	%\section*{Notation}
	
	%Hier steht Notation.
	
\end{document}