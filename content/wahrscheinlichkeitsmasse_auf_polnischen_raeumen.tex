\documentclass[../main/main.tex]{subfiles}

\begin{document}
	
	\section{Eigenschaften von Maßen auf polnischen Räumen}
	
	\subsection{Metrisierbarkeit der schwachen Konvergenz}
	
	\begin{Satz}
		\label{thm:Lp}
		Sei $X$ ein polnischer Raum. Dann gibt es eine abzählbare Menge 
		$\mathcal{D} \subseteq \Bdcontfct{X}$, die dicht liegt in jedem 
		$L^p(X, \mu)$ mit $p \in [1, \infty)$, $\mu \in \Probmeasures{X}$.
	\end{Satz}
	
	\begin{proof}
		Wir beweisen zuerst, dass $C(H) = \Bdcontfct{H} \subseteq L^p(H, \mu)$ für beliebiges 
		$p \in [1, \infty)$, $\mu \in \Probmeasures{H}$ dicht liegt. 
		
		Sei dazu $A \in \mathcal{B}$ eine Borel-messbare Teilmenge von $H$. 
		Aufgrund der schwachen Regularität von $\mu$ (vgl. Satz~\ref{thm:weakregularity}) 
		gibt es für $n \in \N$ abgeschlossene Mengen $C_n$ und offene Mengen 
		$U_n$ mit $C_n \subseteq A \subseteq U_n$ und 
		$\mu(U_n \setminus C_n) < \frac{1}{n}$. Setzen wir nun für alle $n$
		\[\fctmap{f_n}{H}{[0, 1]}{x}{\frac{\rho(x, C_n)}{\rho(x, C_n) + 
				\rho(x, U_n^\mathsf{c})}} \text{,} \quad f_n \in \Bdcontfct{H} \text{,}\]
		so ist
		\[ \norm{f_n - \indfct_A}_{L^p(H, \mu)}^p \; = \; 
		\measureint{}{\abs{f_n - \indfct_A}^p}{\mu} \; \leq \; 
		\mu(U_n \setminus C_n) \; < \; \frac{1}{n} \to 0 \text{,} 
		\quad n \to \infty \text{.}\]
		Insobesondere folgt direkt, dass sich auch jede einfache Funktion 
		beliebig gut bezüglich $\norm{\cdot}_{L^p(H, \mu)}$ durch Funktionen aus $\Bdcontfct{H}$ 
		approximieren lassen kann. Weil die einfachen Funktionen aber dicht 
		in $L^p(H, \mu)$ liegen, muss auch $\Bdcontfct{H} \subseteq L^p(H, \mu)$ eine 
		dichte Teilmenge sein.
		
		Aus Hilfssatz~\ref{lem:hilbertcubefunctionseparability} wissen wir 
		bereits, dass es eine abzählbare Teilmenge $\mathcal{D}\subseteq \Bdcontfct{H}$ 
		gibt, die $\norm{\cdot}_\infty$-dicht in $\Bdcontfct{H}$ liegt. Zudem gilt 
		$\norm{f}_{L^p(H, \mu)} \leq \norm{f}_\infty$ für $f \in \Bdcontfct{H}$, also impliziert 
		$\norm{\cdot}_\infty$-Konvergenz in $\Bdcontfct{H}$ auch $\norm{\cdot}_{L^p(H, \mu)}$-Konvergenz. 
		Somit folgt insgesamt, dass auch $\mathcal{D} \subseteq L^p(H, \mu)$ dicht ist.
		
		Nun möchten wir die Aussage auf einen beliebigen polnischen Raum $X$ ausweiten, 
		wofür wir $\fct{\varphi}{X}{H}$ aus \eqref{eq:2.6} nutzen.
		Wir definieren hierzu das Bildmaß $\nu \defby \mu^\varphi \in \Probmeasures{H}$ 
		auf der Borelschen $\sigma$-Algebra von $H$ und setzen
		\[\fctmap{\Phi}{L^p(H, \nu)}{L^p(X, \mu)}{f}{f \circ \varphi} \text{.}\]
		Für $f \in L^p(H, \nu)$ gilt dann
		\[ \norm{f \circ \varphi}_{L^p(X, \mu)} 
		\; = \; \measureint{}{\abs{f \circ \varphi}^p}{\mu} 
		\; = \; \measureint{}{\abs{f}^p}{\mu^\varphi} 
		\; = \; \measureint{}{\abs{f}^p}{\nu} 
		\; = \; \norm{f}_{L^p(H, \nu)} \text{.} \]
		Wegen $\nu(\varphi(X)) = 1$ ist $\Phi$ bijektiv und damit insgesamt eine Isometrie 
		zwischen $L^p(H, \nu)$ und $L^p(X, \mu)$. Insbesondere ist also auch 
		\[\Phi(\mathcal{D}) = \setcomp{f \circ \varphi}{f \in \mathcal{D}} \subseteq \Bdcontfct{X}\]
		unabhängig von $p$ und $\mu$ eine abzählbare dichte Teilmenge von $L^p(X, \mu)$.
	\end{proof}
	
	\begin{Satz}
		\label{thm:weakconvergencemetrizable}
		Sei $X$ ein polnischer Raum. Dann gibt es eine Metrik $D$ auf $\Probmeasures{X}$, 
		sodass Konvergenz bezüglich $D$ gleichbedeutend mit schwacher Konvergenz 
		(vgl. Definition~\ref{def:weakconvergence}) ist.
	\end{Satz}
	
	\begin{proof}
		Wir möchten direkt unsere Erkenntnisse aus dem Beweis des vorigen 
		Satzes~\ref{thm:Lp} anwenden. Mit der Notation von dort sei zunächst 
		\[\setcomp{h_m}{m \in \N} \defby \Phi(\mathcal{D}) \subseteq \Bdcontfct{X} \text{.}\] 
		Wir behaupten, dass
		\[D(\mu, \nu) \; \defby \; \sum_{m=1}^{\infty} \frac{\abs{\measureint{}{h_m}{\mu} - 
				\measureint{}{h_m}{\nu}}}{2^m \cdot \norm{h_m}_\infty} \text{,} 
		\qquad \mu, \nu \in \Probmeasures{X} \]
		die schwache Konvergenz metrisiert. Offenbar gilt dann für eine Folge 
		$(\mu_k)_k \in \Probmeasures{X}^\N$ und $\mu \in \Probmeasures{X}$ die 
		Äquivalenz
		\[ D(\mu_k, \mu) \to 0 \quad \iff \quad \forall m \in \N: \; 
		\measureint{}{h_m}{\mu_k} \to \measureint{}{h_m}{\mu}\text{,} 
		\quad k \to \infty \text{.} \label{eq:2.11} \tag{2.11} \]
		An \eqref{eq:2.11} sehen wir direkt, dass aus der schwachen Konvergenz 
		$\mu_k \xrightarrow{w} \mu$ die Konvergenz bezüglich $D$ folgt. Wir 
		müssen also nur noch beweisen, dass die rechte Seite von \eqref{eq:2.11} 
		schwache Konvergenz impliziert. Hierfür werden wir zeigen, dass für jede 
		abgeschlossene Menge $C \subseteq X$ die Funktionen $f_n, \; n \in \N$ aus 
		Hilfssatz~\ref{lem:opensets} (wofür wir eine geeignete Metrik auf $X$ wählen) 
		dieselbe Bedingung wie die $h_m$ auf der rechten Seite von \eqref{eq:2.11} 
		erfüllen. Genau wie im Beweis der Implikation $(1) \Rightarrow (2)$ aus dem 
		Beweis des Portmanteau-Satzes (Satz~\ref{thm:portmanteau}) lässt sich dann 
		die Bedingung $(2)$ aus dem Portmanteau-Satz folgern und damit schließlich 
		die schwache Konvergenz $\mu_k \xrightarrow{w} \mu$.
		
		Für den verbleibenden Teil des Beweises definieren wir auf $X$ die Metrik
		\[ \rho^\ast(x, y) \defby \rho(\varphi(x), \varphi(y)) \text{,} \]
		wobei $\fct{\varphi}{X}{H}$ wieder die Abbildung aus \eqref{eq:2.6} sei. 
		$\rho^\ast$ induziert die Topologie von $X$, vollständig ist $(X, \rho^\ast)$ 
		aber im Allgemeinen natürlich nicht. Nun zeigen wir, dass für 
		\[\fctmap{f_n}{X}{\R}{x}{\max \set{0, 1-n \rho^\ast(x, C)}}\] 
		mit $n \in \N$ und $C \subseteq X$ abgeschlossen (hierbei handelt es sich 
		um die Funktionen aus Hilfssatz~\ref{lem:opensets}) die Konvergenz
		\[ \measureint{}{f_n}{\mu_k} \; \to \; \measureint{}{f_n}{\mu} \text{,} 
		\quad k \to \infty \label{eq:2.12} \tag{2.12}\]
		gilt. Zunächst bemerken wir, dass es Funktionen $g_n \in \Bdcontfct{H}$ 
		gibt mit $f_n = g_n \circ \varphi$. Mit der Notation aus 
		Hilfssatz~\ref{lem:hilbertcubefunctionseparability} und Satz~\ref{thm:Lp} 
		existiert nun eine Folge $(g_n^{(l)})_l \in \mathcal{D}$ mit $g_n^{(l)} \to g_n$ 
		bezüglich $\norm{\cdot}_\infty$. Setzen wir 
		$f_n^{(l)} \defby \Phi(g_n^{(l)}) = g_n^{(l)} \circ \varphi \in \Phi(\mathcal{D})$, 
		so gilt ebenfalls
		\[ \norm{f_n^{(l)} - f_n}_\infty \; \to \; 0 \text{,} 
		\quad l \to \infty \text{.} \label{eq:2.13} \tag{2.13} \]
		Per Annahme ist für alle $l \in \N$
		\[ \measureint{}{f_n^{(l)}}{\mu_k} \; \to \; \measureint{}{f_n^{(l)}}{\mu} \text{,} 
		\quad k \to \infty \text{.} \label{eq:2.14} \tag{2.14} \]
		Also können wir für alle $k, l \in \N$ wie folgt abschätzen:
		\begin{align*}
			\left| \measureint{}{f_n}{\mu} - \measureint{}{f_n}{\mu_k} \right| \; &\leq \; 
			\left| \measureint{}{f_n}{\mu} - \measureint{}{f_n^{(l)}}{\mu} \right| + 
			\left| \measureint{}{f_n^{(l)}}{\mu} - \measureint{}{f_n^{(l)}}{\mu_k} \right| \\
			& \qquad + 
			\left| \measureint{}{f_n^{(l)}}{\mu_k} - \measureint{}{f_n}{\mu_k} \right| \\
			&\leq \; 2 \cdot \norm{f_n^{(l)} - f_n}_\infty + \left| \measureint{}{f_n^{(l)}}{\mu} - 
			\measureint{}{f_n^{(l)}}{\mu_k} \right| \text{.}
		\end{align*}
		Dies impliziert aber wegen \eqref{eq:2.13} und \eqref{eq:2.14}
		\begin{align*}
			\limsup_{k \to \infty} \left| \measureint{}{f_n}{\mu} - 
			\measureint{}{f_n}{\mu_k} \right|
			\; &\leq \; 2 \cdot \norm{f_n^{(l)} - f_n}_\infty + 
			\limsup_{k \to \infty} \left| \measureint{}{f_n^{(l)}}{\mu} - 
			\measureint{}{f_n^{(l)}}{\mu_k} \right| \\
			&= \; 2 \cdot \norm{f_n^{(l)} - f_n}_\infty \; \to \; 0 \text{,} 
			\quad l \to \infty \text{,}
		\end{align*}
		und damit auch \eqref{eq:2.12}, womit wir den Beweis abschließen.
	\end{proof}
	
	\subsection{Straffheit und der Satz von Prokhorov}
	
	\begin{Definition}
		Sei $X$ ein polnischer Raum und $\mu \in \Probmeasures{X}$. Dann nennen wir 
		$\mu$ \emph{straff}, falls es für jedes $\varepsilon > 0$ eine kompakte Teilmenge
		$K_\varepsilon \subseteq X$ gibt mit 
		\[ \mu(K_\varepsilon) \geq 1  - \varepsilon \text{.} \]
	\end{Definition}
	
	\begin{Satz}
		\label{thm:tightness}
		Sei $X$ ein polnischer Raum und $\mu \in \Probmeasures{X}$. Dann ist $\mu$ straff.
	\end{Satz}
	
	\begin{proof}
		Sei $d$ eine Metrik, bezüglich der $(X, d)$ vollständig ist und sei 
		$\mathcal{D} \defby \setcomp{x_m}{m \in \N} \subseteq X$ eine abzählbare dichte Teilmenge. 
		Für $k \in \N$ gilt also $\bigcup_{m \in \N} B_{1/k}(x_m) = X$ und Maßstetigkeit 
		von unten impliziert 
		$\lim_{M \to \infty} \bigcup_{m=1}^{M} B_{1/k}(x_m) = 1$.
		
		Sei nun $\varepsilon > 0$. Dann finden wir natürliche Zahlen $M_1 \leq M_2 \leq \dots$ mit
		\[ \mu\left( \bigcup_{m=1}^{M_k} B_{1/k}(x_m) \right) \; \geq \; 1 - \frac{\varepsilon}{2^k} \]
		für alle $k \in \N$. Wir setzen nun
		\[ S_\varepsilon 
		\; \defby \; \bigcap_{k \in \N} \left( \bigcup_{m=1}^{M_k} B_{1/k}(x_m) \right) 
		\quad \text{und} \quad K_\varepsilon \defby \overline{S_\varepsilon} \text{.} \]
		Offenbar ist $S_\varepsilon$ totalbeschränkt und damit ist $K_\varepsilon = \overline{S_\varepsilon}$ 
		kompakt (dies folgt etwa aus \cite[Satz 2.3.8]{Simon.2015}, 
		hier geht die Vollständigkeit von $(X, d)$ ein).
		Außerdem berechnen wir
		\[ \mu(K_\varepsilon) 
		\; \geq \; \mu(S_\varepsilon) 
		\; \geq \; 1 - \sum_{k=1}^{\infty} \left( 1 - \mu\left( \bigcup_{m=1}^{M_k} B_{1/k}(x_m) \right) \right) 
		\; \geq \; 1 - \sum_{k=1}^{\infty} \frac{\varepsilon}{2^k} \; = \; 1 - \varepsilon \text{.} \]
		Weil $\varepsilon > 0$ beliebig gewählt war, ist $\mu$ also straff.
	\end{proof}
	
	Ausgestattet mit des Konzepts der Straffheit können wir nun folgenden Satz beweisen:
	
	\begin{Korollar}
		Ist $X$ ein polnischer Raum, so ist jedes $\mu \in \Probmeasures{X}$ regulär.
	\end{Korollar}
	
	\begin{proof}
		Sei $\mu \in \Probmeasures{X}$ und $B \subseteq X$ Borel-messbar. 
		Wegen \ref{thm:weakregularity} wissen wir bereits, dass 
		\[\mu(B) = \sup_{\substack{C \subseteq B \\ C \text{ abgeschlossen}}} \mu(C) 
		\quad \text{bzw.} \quad \mu(B) = \inf_{\substack{U \supseteq B \\ U \text{ offen}}} 
		\mu(U)\]
		gilt. Sei nun $\varepsilon > 0$. Dann gibt es eine abgeschlossene Teilmenge 
		$C \subseteq B \subseteq X$ mit $\mu(B \setminus C) < \varepsilon$. 
		Außerdem gibt es nach Satz~\ref{thm:tightness} eine kompakte Menge 
		$K^\ast \subseteq X$ mit $\mu(K^\ast) \geq 1 - \varepsilon$. Setzen 
		wir nun $K \defby K^\ast \cap C$, so ist $K$ ebenfalls kompakt und 
		$K \subseteq B$ mit 
		\[ \mu(B \setminus K) 
		\; = \; \mu((B \setminus C) \cap (B \setminus K^\ast)) 
		\; \leq \; \mu(B \setminus C) + \mu(K^{\ast \mathsf{c}}) 
		\; \leq \; 2 \varepsilon \text{.} \]
		Daher folgt 
		\[\mu(B) 
		= \sup_{\substack{K \subseteq B \\ K \text{ kompakt}}} \mu(K)\] 
		und insgesamt $\mu$ ist regulär.
	\end{proof}
	
\end{document}