\documentclass[../main/main.tex]{subfiles}

\begin{document}
	
	\section{Maße auf polnischen Räumen}
	
	Nachdem wir uns im vorigen Abschnitt zu einem topologischen Raum $X$ den Raum der endlichen Borel-Maße 
	$\Finitemeasures{X}$ bzw. den Raum der Wahrscheinlichkeitsmaße $\Probmeasures{X}$ mit einer Topologie bzw.
	einem Konvergenzbegriff versehen haben, sind nun die topologischen Eigenschaften ebendieser Räume zu 
	untersuchen. Es wird sich herausstellen, dass sich bereits durch die verhältnismäßig schwache Einschränkung 
	des Grundraumes $X$ auf die Klasse der polnischen Räume relativ weitreichende Schlüsse über die Topologie 
	der schwachen Konvergenz von $\Finitemeasures{X}$ bzw. $\Probmeasures{X}$ ziehen lassen. So werden wir in 
	Abschnitt~\ref{subsec:weakconvergencemetrizable} als zentrales Ergebnis die Metrisierbarkeit von 
	$\Probmeasures{X}$ bezüglich der schwachen Konvergenz erhalten und in Abschnitt~\ref{subsec:prokhorov} im 
	\emph{Satz von Prokhorov} (Satz~\ref{thm:prokhorov}) eine Charakterisierung von kompakten Teilmengen 
	in $\Probmeasures{X}$ sehen.
	
	\subsection{Metrisierbarkeit der schwachen Konvergenz}
	\label{subsec:weakconvergencemetrizable}
	
	Das Ziel dieses Abschnittes ist es, für einen beliebigen polnischen Raum $X$ die Metrisierbarkeit von 
	$\Probmeasures{X}$ zu zeigen. Im Beweis dieser Aussage werden wir den folgenden Satz verwenden.
	
	\begin{Satz}
		\label{thm:Lp}
		Sei $X$ ein polnischer Raum. Dann gibt es eine abzählbare Menge 
		$\mathcal{D} \subseteq \Bdcontfct{X}$, die dicht liegt in jedem 
		$L^p(X, \mu)$ mit $p \in [1, \infty)$, $\mu \in \Probmeasures{X}$.
	\end{Satz}
	
	\begin{proof}
		Wir beweisen zuerst, dass $C(H) = \Bdcontfct{H} \subseteq L^p(H, \mu)$ für beliebiges 
		$p \in [1, \infty)$, $\mu \in \Probmeasures{H}$ dicht liegt. 
		
		Sei dazu $A \in \mathcal{B}$ eine Borel-messbare Teilmenge von $H$. 
		Aufgrund der schwachen Regularität von $\mu$ (vgl. Satz~\ref{thm:weakregularity}) 
		gibt es für $n \in \N$ abgeschlossene Mengen $C_n$ und offene Mengen 
		$U_n$ mit $C_n \subseteq A \subseteq U_n$ und 
		$\mu(U_n \setminus C_n) < \frac{1}{n}$. Setzen wir nun für alle $n$
		\[\fctmap{f_n}{H}{[0, 1]}{x}{\frac{\rho(x, C_n)}{\rho(x, C_n) + 
				\rho(x, U_n^\mathsf{c})}} \text{,} \quad f_n \in \Bdcontfct{H} \text{,}\]
		so ist
		\[ \norm{f_n - \indfct_A}_{L^p(H, \mu)}^p \; = \; 
		\measureint{}{\abs{f_n - \indfct_A}^p}{\mu} \; \leq \; 
		\mu(U_n \setminus C_n) \; < \; \frac{1}{n} \to 0 \text{,} 
		\quad n \to \infty \text{.}\]
		Insobesondere folgt direkt, dass sich auch jede einfache Funktion 
		beliebig gut bezüglich $\norm{\cdot}_{L^p(H, \mu)}$ durch Funktionen aus $\Bdcontfct{H}$ 
		approximieren lassen kann. Weil die einfachen Funktionen aber dicht 
		in $L^p(H, \mu)$ liegen, muss auch $\Bdcontfct{H} \subseteq L^p(H, \mu)$ eine 
		dichte Teilmenge sein.
		
		Aus Hilfssatz~\ref{lem:hilbertcubefunctionseparability} wissen wir 
		bereits, dass es eine abzählbare Teilmenge $\mathcal{D}\subseteq \Bdcontfct{H}$ 
		gibt, die $\norm{\cdot}_\infty$-dicht in $\Bdcontfct{H}$ liegt. Zudem gilt 
		$\norm{f}_{L^p(H, \mu)} \leq \norm{f}_\infty$ für $f \in \Bdcontfct{H}$, also impliziert 
		$\norm{\cdot}_\infty$-Konvergenz in $\Bdcontfct{H}$ auch $\norm{\cdot}_{L^p(H, \mu)}$-Konvergenz. 
		Somit folgt insgesamt, dass auch $\mathcal{D} \subseteq L^p(H, \mu)$ dicht ist.
		
		Nun möchten wir die Aussage auf einen beliebigen polnischen Raum $X$ ausweiten, 
		wofür wir $\fct{\varphi}{X}{H}$ aus Hilfssatz~\ref{lem:characterizationpolishspaces} nutzen.
		Wir definieren hierzu das Bildmaß $\nu \defby \mu^\varphi \in \Probmeasures{H}$ 
		auf der Borelschen $\sigma$-Algebra von $H$ und setzen
		\[\fctmap{\Phi}{L^p(H, \nu)}{L^p(X, \mu)}{f}{f \circ \varphi} \text{.}\]
		Für $f \in L^p(H, \nu)$ gilt dann
		\[ \norm{f \circ \varphi}_{L^p(X, \mu)} 
		\; = \; \measureint{}{\abs{f \circ \varphi}^p}{\mu} 
		\; = \; \measureint{}{\abs{f}^p}{\mu^\varphi} 
		\; = \; \measureint{}{\abs{f}^p}{\nu} 
		\; = \; \norm{f}_{L^p(H, \nu)} \text{.} \]
		Wegen $\nu(\varphi(X)) = 1$ ist $\Phi$ bijektiv und damit insgesamt eine Isometrie 
		zwischen $L^p(H, \nu)$ und $L^p(X, \mu)$. Insbesondere ist also auch 
		\[\Phi(\mathcal{D}) = \setcomp{f \circ \varphi}{f \in \mathcal{D}} \subseteq \Bdcontfct{X}\]
		unabhängig von $p$ und $\mu$ eine abzählbare dichte Teilmenge von $L^p(X, \mu)$.
	\end{proof}

	\begin{Bemerkung}
		Es sei angemerkt, dass wir hier nicht die Separabilität von $(\Bdcontfct{X}, \norm{\cdot}_\infty)$ 
		für einen beliebigen polnischen Raum $X$ beweisen. Diese ist im Allgemeinen gar nicht gegeben, wie das Beispiel
		$(l^\infty, \norm{\cdot}_\infty) = (\Bdcontfct{\N}, \norm{\cdot}_\infty)$ zeigt.
	\end{Bemerkung}

	Ausgestattet mit Satz~\ref{thm:Lp} können wir nun zum zentralen Satz dieses Abschnitts übergehen.
	
	\begin{Satz}
		\label{thm:weakconvergencemetrizable}
		Sei $X$ ein polnischer Raum. Dann gibt es eine Metrik $D$ auf $\Probmeasures{X}$, 
		die die Topologie der schwachen Konvergenz auf $\Probmeasures{X}$ metrisiert.
	\end{Satz}

	\tobechanged{Das würde ich gerne beweisen, hier wird allerdings nur gezeigt, 
		dass Konvergenz bzgl. $D$ der schwachen Konvergenz entspricht.}
	
	\begin{proof}
		Wir möchten direkt unsere Erkenntnisse aus dem vorigen 
		Satz~\ref{thm:Lp} anwenden. Mit der Notation von dort sei zunächst 
		\[\setcomp{h_m}{m \in \N} \defby \Phi(\mathcal{D}) \subseteq \Bdcontfct{X} \text{.}\] 
		Wir behaupten, dass
		\[D(\mu, \nu) \; \defby \; \sum_{m=1}^{\infty} \frac{\abs{\measureint{}{h_m}{\mu} - 
				\measureint{}{h_m}{\nu}}}{2^m \cdot \norm{h_m}_\infty} \text{,} 
		\qquad \mu, \nu \in \Probmeasures{X} \]
		die Topologie der schwachen Konvergenz metrisiert. 
		
		Offenbar gilt für eine Folge 
		$(\mu_k)_k \in \Probmeasures{X}^\N$ und $\mu \in \Probmeasures{X}$ zunächst die 
		Äquivalenz
		\[ D(\mu_k, \mu) \to 0 \quad \iff \quad \forall m \in \N: \; 
		\measureint{}{h_m}{\mu_k} \to \measureint{}{h_m}{\mu}\text{,} 
		\quad k \to \infty \text{.} \label{eq:4.1} \tag{4.1} \]
		An \eqref{eq:4.1} sehen wir direkt, dass aus der schwachen Konvergenz 
		$\mu_k \xrightarrow{w} \mu$ die Konvergenz bezüglich $D$ folgt. Wir 
		müssen also nur noch beweisen, dass die rechte Seite von \eqref{eq:4.1} 
		schwache Konvergenz impliziert. Hierfür werden wir zeigen, dass für jede 
		abgeschlossene Menge $C \subseteq X$ die Funktionen $f_n, \; n \in \N$ aus 
		Hilfssatz~\ref{lem:opensets} (wofür wir eine geeignete Metrik auf $X$ wählen) 
		dieselbe Bedingung wie die $h_m$ auf der rechten Seite von \eqref{eq:4.1} 
		erfüllen. Genau wie im Beweis der Implikation (i) $\Rightarrow$ (ii) beim 
		Portmanteau-Theorem (Satz~\ref{thm:portmanteau}) lässt sich dann 
		die Bedingung (ii) aus dem Portmanteau-Theorem folgern und damit schließlich 
		die schwache Konvergenz $\mu_k \xrightarrow{w} \mu$.
		
		Für den verbleibenden Teil des Beweises definieren wir auf $X$ die Metrik
		\[ \tilde{\rho}(x, y) \defby \rho(\varphi(x), \varphi(y)) \text{,} \]
		wobei $\fct{\varphi}{X}{H}$ wieder die Abbildung aus \eqref{eq:2.6} sei. 
		$\tilde{\rho}$ induziert die Topologie von $X$ (vollständig ist $(X, \tilde{\rho})$ 
		aber im Allgemeinen natürlich nicht). Nun zeigen wir, dass für 
		\[\fctmap{f_n}{X}{\R}{x}{\max \set{0, 1-n \tilde{\rho}(x, C)}}\] 
		mit $n \in \N$ und $C \subseteq X$ abgeschlossen (hierbei handelt es sich 
		um die Funktionen aus Hilfssatz~\ref{lem:opensets}) die Konvergenz
		\[ \measureint{}{f_n}{\mu_k} \; \to \; \measureint{}{f_n}{\mu} \text{,} 
		\quad k \to \infty \label{eq:4.2} \tag{4.2}\]
		gilt. Zunächst bemerken wir, dass es Funktionen $g_n \in \Bdcontfct{H}$ 
		gibt mit $f_n = g_n \circ \varphi$. Mit der Notation aus 
		Hilfssatz~\ref{lem:hilbertcubefunctionseparability} und Satz~\ref{thm:Lp} 
		existiert nun eine Folge $(g_n^{(l)})_l \in \mathcal{D}$ mit $g_n^{(l)} \to g_n$ 
		bezüglich $\norm{\cdot}_\infty$. Setzen wir 
		$f_n^{(l)} \defby \Phi(g_n^{(l)}) = g_n^{(l)} \circ \varphi \in \Phi(\mathcal{D})$, 
		so gilt ebenfalls
		\[ \norm{f_n^{(l)} - f_n}_\infty \; \to \; 0 \text{,} 
		\quad l \to \infty \text{.} \label{eq:4.3} \tag{4.3} \]
		Per Annahme ist für alle $l \in \N$
		\[ \measureint{}{f_n^{(l)}}{\mu_k} \; \to \; \measureint{}{f_n^{(l)}}{\mu} \text{,} 
		\quad k \to \infty \text{.} \label{eq:4.4} \tag{4.4} \]
		Also können wir für alle $k, l \in \N$ wie folgt abschätzen:
		\begin{align*}
			\left| \measureint{}{f_n}{\mu} - \measureint{}{f_n}{\mu_k} \right| \; &\leq \; 
			\left| \measureint{}{f_n}{\mu} - \measureint{}{f_n^{(l)}}{\mu} \right| + 
			\left| \measureint{}{f_n^{(l)}}{\mu} - \measureint{}{f_n^{(l)}}{\mu_k} \right| \\
			& \qquad + 
			\left| \measureint{}{f_n^{(l)}}{\mu_k} - \measureint{}{f_n}{\mu_k} \right| \\
			&\leq \; 2 \cdot \norm{f_n^{(l)} - f_n}_\infty + \left| \measureint{}{f_n^{(l)}}{\mu} - 
			\measureint{}{f_n^{(l)}}{\mu_k} \right| \text{.}
		\end{align*}
		Dies impliziert aber wegen \eqref{eq:4.3} und \eqref{eq:4.4}
		\begin{align*}
			\limsup_{k \to \infty} \left| \measureint{}{f_n}{\mu} - 
			\measureint{}{f_n}{\mu_k} \right|
			\; &\leq \; 2 \cdot \norm{f_n^{(l)} - f_n}_\infty + 
			\limsup_{k \to \infty} \left| \measureint{}{f_n^{(l)}}{\mu} - 
			\measureint{}{f_n^{(l)}}{\mu_k} \right| \\
			&= \; 2 \cdot \norm{f_n^{(l)} - f_n}_\infty \; \to \; 0 \text{,} 
			\quad l \to \infty \text{,}
		\end{align*}
		und damit auch \eqref{eq:4.2}, womit wir den Beweis abschließen.
	\end{proof}

	\begin{Satz}
		\label{thm:compactequivalence}
		Sei $(X, d)$ ein metrischer Raum. Dann ist $X$ genau dann kompakt, wenn $\Probmeasures{X}$ 
		bezüglich der Topologie der schwachen Konvergenz kompakt ist.
	\end{Satz}

	\tobechanged{FEHLT: Rückrichtung}

	\begin{proof}
		Sei zunächst $(X, d)$ kompakt. Wir möchten nun so ähnlich wie bei \eqref{3.5} verfahren.
		Hierfür setzen wir zunächst
		\[ A \; \defby \; \setcomp{l \in \Bdcontfct{X}^\ast}{\norm{l}_{\Bdcontfct{X}^\ast} \leq 1, \; l(\indfct_{X}) = 1, \; 
			\forall f \in \Bdcontfct{X} \text{ mit } f \geq 0 : l(f) \geq 0}\]
		und
		\[ \fctmap{\Phi}{\Probmeasures{X}}{A}{\mu}{\left[\fctmap{l_\mu}{\Bdcontfct{X}}{\R}{f}{\measureint{}{f}{\mu}}\right]} \text{.} \]
		Der Darstellungssatz von Riesz-Markov liefert die Bijektivität von $\Phi$, also ist $\Phi$ ein 
		Homöomorphismus zwischen $\Probmeasures{X}$ mit der schwachen Konvergenz und $A \subseteq \Bdcontfct{X}^\ast$ 
		mit der Schwach-$\ast$-Konvergenz. Nach dem Satz von Banach-Alaoglu ist $A$ nun als abgeschlossene Teilmenge von 
		$\setcomp{l \in C_b(X)^\ast}{\norm{l}_{\Bdcontfct{X}^\ast} \leq 1}$ kompakt bezüglich der Schwach-$\ast$-Topologie, 
		was die Kompaktheit von $\Probmeasures{X}$ nach sich zieht.
	\end{proof}

	\tobechanged{TODO: Beweisen bzw. überlegen, ob man das noch machen sollte...}

	\begin{Satz}
		Sei $X$ ein topologischer Raum. Dann ist $X$ genau dann polnisch, wenn $\Probmeasures{X}$ bezüglich der 
		Topologie der schwachen Konvergenz polnisch ist.
	\end{Satz}
	
	\subsection{Straffheit und der Satz von Prokhorov}
	\label{subsec:prokhorov}
	
	\begin{Definition}
		Sei $X$ ein polnischer Raum und $\mu \in \Probmeasures{X}$. Dann nennen wir 
		$\mu$ \emph{straff}, falls es für jedes $\varepsilon > 0$ eine kompakte Teilmenge
		$K_\varepsilon \subseteq X$ gibt mit 
		\[ \mu(K_\varepsilon) \geq 1  - \varepsilon \text{.} \]
	\end{Definition}
	
	\begin{Satz}
		\label{thm:tightness}
		Sei $X$ ein polnischer Raum und $\mu \in \Probmeasures{X}$. Dann ist $\mu$ straff.
	\end{Satz}
	
	\begin{proof}
		Sei $d$ eine Metrik, bezüglich der $(X, d)$ vollständig ist und sei 
		$\mathcal{D} \defby \setcomp{x_m}{m \in \N}$ eine abzählbare dichte Teilmenge von $X$. 
		Für $k \in \N$ gilt also $\bigcup_{m \in \N} B_{1/k}(x_m) = X$ und Maßstetigkeit 
		von unten impliziert 
		$\lim_{M \to \infty} \bigcup_{m=1}^{M} B_{1/k}(x_m) = 1$.
		
		Sei nun $\varepsilon > 0$. Dann finden wir natürliche Zahlen $M_1 \leq M_2 \leq \dots$ mit
		\[ \mu\left( \bigcup_{m=1}^{M_k} B_{1/k}(x_m) \right) \; \geq \; 1 - \frac{\varepsilon}{2^k} \]
		für alle $k \in \N$. Wir setzen nun
		\[ S_\varepsilon 
		\; \defby \; \bigcap_{k \in \N} \left( \bigcup_{m=1}^{M_k} B_{1/k}(x_m) \right) 
		\quad \text{und} \quad K_\varepsilon \defby \overline{S_\varepsilon} \text{.} \]
		Offenbar ist $S_\varepsilon$ totalbeschränkt und damit ist $K_\varepsilon = \overline{S_\varepsilon}$ 
		kompakt (dies wird etwa in \cite[Satz 2.3.8]{Simon.2015} bewiesen, 
		hier geht die Vollständigkeit von $(X, d)$ ein).
		Außerdem berechnen wir
		\[ \mu(K_\varepsilon) 
		\; \geq \; \mu(S_\varepsilon) 
		\; \geq \; 1 - \sum_{k=1}^{\infty} \left( 1 - \mu\left( \bigcup_{m=1}^{M_k} B_{1/k}(x_m) \right) \right) 
		\; \geq \; 1 - \sum_{k=1}^{\infty} \frac{\varepsilon}{2^k} \; = \; 1 - \varepsilon \text{.} \]
		Weil $\varepsilon > 0$ beliebig gewählt war, ist $\mu$ also straff.
	\end{proof}
	
	Ausgestattet mit des Konzepts der Straffheit können wir nun folgenden Satz beweisen:
	
	\begin{Korollar}
		Ist $X$ ein polnischer Raum, so ist jedes $\mu \in \Probmeasures{X}$ regulär.
	\end{Korollar}
	
	\begin{proof}
		Sei $\mu \in \Probmeasures{X}$ und $B \in \mathcal{B}(X)$. 
		Wegen \ref{thm:weakregularity} wissen wir bereits, dass 
		\[\mu(B) = \sup_{\substack{C \subseteq B \\ C \text{ abgeschlossen}}} \mu(C) 
		\quad \text{bzw.} \quad \mu(B) = \inf_{\substack{U \supseteq B \\ U \text{ offen}}} 
		\mu(U)\]
		gilt. Sei nun $\varepsilon > 0$. Dann gibt es eine abgeschlossene Teilmenge 
		$C \subseteq B \subseteq X$ mit $\mu(B \setminus C) < \varepsilon$. 
		Außerdem gibt es nach Satz~\ref{thm:tightness} eine kompakte Menge 
		$\tilde{K} \subseteq X$ mit $\mu(\tilde{K}) \geq 1 - \varepsilon$. Setzen 
		wir nun $K \defby \tilde{K} \cap C$, so ist $K$ ebenfalls kompakt und 
		$K \subseteq B$ mit 
		\[ \mu(B \setminus K) 
		\; = \; \mu((B \setminus C) \cap (B \setminus \tilde{K})) 
		\; \leq \; \mu(B \setminus C) + \mu(K^{\ast \mathsf{c}}) 
		\; \leq \; 2 \varepsilon \text{.} \]
		Daher folgt 
		\[\mu(B) 
		= \sup_{\substack{K \subseteq B \\ K \text{ kompakt}}} \mu(K)\] 
		und insgesamt $\mu$ ist regulär.
	\end{proof}

	\begin{Satz}[Prokhorov]
		\label{thm:prokhorov}
		Sei $X$ ein polnischer Raum und $A \subseteq \Probmeasures{X}$, wobei $\Probmeasures{X}$ mit der 
		Topologie der schwachen Konvergenz versehen sei. Dann sind folgende Aussagen äquivalent:
		\begin{equivalentthm}
			\item $A$ ist straff.
			\item $\overline{A} \subseteq \Probmeasures{X}$ ist kompakt.
		\end{equivalentthm}
	\end{Satz}

	\begin{proof}
		Sei zunächst $A \subseteq \Probmeasures{X}$ straff. Wir möchten nun beweisen, dass $\overline{A}$ 
		kompakt ist, was wegen der Metrisierbarkeit von $\Probmeasures{X}$ äquivalent dazu ist,
		dass jede Folge $(\mu_n)_n \in A^\N$ eine schwach konvergente Teilfolge besitzt. 
		
		Aufgrund der Straffheit von $A$ gibt es für jedes $m \in \N$ ein Kompaktum $K^{(m)} \subseteq X$, 
		das für alle $n \in \N$
		\[ \mu_n(K^{(m)}) \geq 1 - \frac{1}{m+1} \label{4.5} \tag{4.5} \]
		erfüllt. Ohne Einschränkung dürfen wir außerdem annehmen, dass 
		\[ K^{(m)} \subseteq K^{(m+1)} \label{4.6} \tag{4.6} \] 
		für alle $m \in \N$ gilt.
		
		Sei nun also $(\mu_n)_n \in A^\N$.
		In Analogie zu Satz~\ref{thm:compactequivalence} lässt sich nun auch einsehen, dass 
		\[ \Unitmeasures{K^{(m)}} \; \defby \; \setcomp{\fct{\mu}{\mathcal{B}(K^{(m)})}{[0,1]}}{\mu \; 
			\text{ist endliches Maß und } \mu(K^{(m)}) \leq 1} \]
		für alle $m \in \N$ kompakt bezüglich der Topologie der schwachen Konvergenz ist. Damit existiert ein Maß 
		$\nu^{(1)} \in \Unitmeasures{K^{(1)}}$ und eine Teilfolge $(\mu_{n_j}^{(1)})_k$ von $(\mu_n)_n$ mit
		\[ \restr{\mu_{n_j}^{(1)}}{K^{(1)}} \; \xrightarrow{w} \; \nu^{(1)} \text{.} \]
		Auf dieselbe Art und Weise findet man nun für alle $m\geq 2$ sukzessive Maße 
		$\nu^{(m)} \in \Unitmeasures{K^{(m)}}$ und Teilfolgen $(\mu_{n_j}^{(m)})_j$ von $(\mu_{n_j}^{(m-1)})_j$ mit
		\[ \restr{\mu_{n_j}^{(m)}}{K^{(m)}} \; \xrightarrow{w} \; \nu^{(m)}, \quad j \to \infty \text{.} \]
		Nun diagonalisieren wir und erhalten mit $\mu_{n_j} \; \defby \; \mu_{n_j}^{(j)}$ eine Teilfolge 
		$(\mu_{n_j})_j$ von $(\mu_n)_n$, die
		\[ \restr{\mu_{n_j}}{K^{(m)}} \; \xrightarrow{w} \; \nu^{(m)}, \quad j \to \infty \label{4.7} \tag{4.7} \]
		für alle $m \in \N$ erfüllt.
		
		Außerdem können Maße $\nu^{(m)}, \; m \in \N$ via 
		$\nu^{(m)}(B) \defby \nu^{(m)}(B \cap K^{(m)}), \; B \in \mathcal{B}(X)$ einfach auf $X$ fortgesetzt werden. 
		Mit dieser Konvention können wir nun beweisen, dass die Maße
		$\nu^{(m)}, \; m \in \N$ in dem Sinne monoton wachsen, dass 
		\[ \nu^{(m)}(B) \leq \nu^{(m+1)}(B) \label{4.8} \tag{4.8} \]
		für alle $B \in \mathcal{B}(X)$ und $m \in \N$ gilt:
		
		Wegen schwacher Regularität (vgl. Satz~\ref{thm:weakregularity}) genügt es, die Ungleichung \eqref{4.8} 
		für abgeschlossene Mengen zu zeigen. Sei also $m \in \N$ fixiert, $C \subseteq X$ abgeschlossen 
		und seien $\fct{f_k}{X}{\R}, \; k \in \N$ die gleichmäßig stetigen, beschränkten Funktionen aus 
		Hilfssatz~\ref{lem:opensets}. Unter Ausnutzung von \eqref{4.6} folgt nun 
		\[ \measureint{K^{(m)}}{f_k}{\mu_{n_j}} \; \leq \; \measureint{K^{(m+1)}}{f_k}{\mu_{n_j}} \]
		für alle $k, j \in \N$. Wegen 
		\[ \measureint{K^{(m)}}{f_k}{\mu_{n_j}} \to \ \measureint{}{f_k}{\nu^{(m)}} \quad \text{und} 
			\quad \measureint{K^{(m+1)}}{f_k}{\mu_{n_j}} \to \ \measureint{}{f_k}{\nu^{(m+1)}}, \quad j \to \infty \]
		ist damit auch 
		\[ \measureint{}{f_k}{\nu^{(m)}} \; \leq \; \measureint{}{f_k}{\nu^{(m+1)}} \]
		für alle $k \in \N$, was wiederum aufgrund der Konvergenz $f_k \convdown \indfct_C$ und dem Satz von Lebesgue Ungleichung 
		\eqref{4.8} impliziert.
		
		Schlussendlich möchten wir nun nachweisen, dass durch
		\[ \fctmap{\nu}{\mathcal{B}(X)}{[0, 1]}{B}{\sup_{m \in \N} \nu^{(m)}(B)} \]
		ein Wahrscheinlichkeitsmaß auf $X$ definiert wird, gegen das $(\mu_{n_j})_j$ schwach konvergiert.
		
		Zunächst verifizieren wir, dass $\nu$ ein Wahrscheinlichkeitsmaß ist. $\nu(\emptyset) = 0$ ist klar. 
		Weil $\nu^{(m)}$ in $\Unitmeasures{K^{(m)}}$ liegt und wir \eqref{4.5} angenommen haben, gilt 
		\[\nu^{(m)}(X) = \nu^{(m)}(K^{(m)}) \in [1 - \frac{1}{m+1}, 1]\] 
		für alle $m \in \N$ und damit $\nu(X) = 1$. Für die $\sigma$-Additivität seien 
		$B_k \in \mathcal{B}(X), \; k \in \N$ paarweise disjunkt. Dann gilt
		\begin{align*}
			\nu \left( \bigcupdot_{k \in \N} B_k \right) \; &= \; \sup_{m \in \N} \; \nu^{(m)} \left( \bigcupdot_{k \in \N} B_k \right) 
																\; = \; \sup_{m \in \N} \; \sum_{k=1}^{\infty} \nu^{(m)}(B_k) \\
			                                                &\stackrel{\eqref{4.8}}{=} \; \sum_{k=1}^{\infty} \; \sup_{m \in \N} \nu^{(m)}(B_k) 
			                                                	\; = \; \sum_{k=1}^{\infty} \nu(B_k) \text{,}
		\end{align*}
		wobei in der vorletzten Gleichung der Satz von Beppo Levi verwendet wurde. 
		Insgesamt haben wir also $\nu \in \Probmeasures{X}$ nachgewiesen.
		
		Jetzt bleibt lediglich zu zeigen dass $(\mu_{n_j})_j$ tatsächlich schwach gegen $\nu$ konvergiert. 
		Hierfür möchten wir die Charakterisierung (ii) des Portmanteau-Theorems (Satz~\ref{thm:portmanteau}) verwenden.
		Sei dazu $C \subseteq X$ abgeschlossen. Zunächst liefert ebendieser Satz für alle $m \in \N$ die 
		Abschätzung $\limsup_{j \to \infty} \mu_{n_j}(C \cap K^{(m)}) \leq \nu^{(m)}(C)$ und wegen \eqref{4.5} ist 
		zudem $\mu_{n_j}(C \cap K^{(m) \mathsf{c}}) \leq  \frac{1}{m+1}$ für alle $j, m \in \N$. Insgesamt folgt daraus 
		\begin{align*}
			\limsup_{j \to \infty} \mu_{n_j}(C) \; &=    \; \lim_{m \to \infty} \limsup_{j \to \infty} 
														\left( \mu_{n_j}(C \cap K^{(m)}) + \mu_{n_j}(C \cap K^{(m) \mathsf{c}}) \right) \\
			                                       &\leq \; \lim_{m \to \infty} \left( \nu^{(m)}(C) + \frac{1}{m+1} \right) 
			                                       		\; = \; \nu(C) \text{,}
		\end{align*}
		was den Beweis der Hinrichtung von Satz~\ref{thm:prokhorov} abschließt.
		
		Für die Rückrichtung sei nun $\overline{A}$ kompakt. Außerdem sei $d$ eine Metrik, die $X$ vollständig metrisiert 
		und $\mathcal{D} \defby \setcomp{x_n}{n \in \N} \subseteq X$ eine abzählbare dichte Teilmenge.
		
		Nun behaupten wir, dass für alle $\delta > 0$ ein solches $M_\delta \in \N$ existiert, dass
		\[ \mu(\bigcup_{m=1}^{M_\delta} B_\delta(x_m)) \; > \; 1 - \delta \label{4.9} \tag{4.9} \]
		für alle $\mu \in A$ gilt. Denn falls es kein derartiges $M_\delta$ gibt, so lässt sich ein $\delta > 0$ finden, für das
		für alle $M \in \N$ ein $\mu_M \in A$ existiert mit
		\[ \mu_M(\bigcup_{m=1}^{M} B_\delta(x_m)) \; \leq \; 1 - \delta \text{.} \]
		Insbesondere bedeutet das natürlich auch, dass wir für alle $M \in \N$ und $N \geq M$ 
		\[ \mu_N(\bigcup_{m=1}^{M} B_\delta(x_m)) \; \leq \; 1 - \delta \]
		abschätzen dürfen. Aufgrund der Kompaktheit von $\overline{A}$ gibt es eine Teilfolge $(\mu_{N_j})_j$ von $(\mu_N)_N$, 
		die einen schwachen Grenzwert $\mu \in \Probmeasures{X}$ besitzt. Fixiere nun ein $M \in \N$. 
		Dann ist $\bigcup_{m=1}^{M} B_\delta(x_m)$
		offen und daher liefert das Portmanteau-Theorem (Satz~\ref{thm:portmanteau} (iii))
		\[ \mu(\bigcup_{m=1}^{M} B_\delta(x_m)) 
			\; \leq \; \liminf_{j \to \infty} \mu_{N_j}(\bigcup_{m=1}^{M} B_\delta(x_m)) 
			\; \leq \; 1 - \delta \text{.} \label{4.10} \tag{4.10} \]
		Wegen $\bigcup_{m=1}^{\infty} B_\delta(x_m) = X$ und Maßstetigkeit von unten 
		liefert uns \eqref{4.10} unmittelbar 
		\[ \mu(X) \; \leq \; 1 - \delta \; < \; 1 \text{,} \]
		was einen Widerspruch dazu darstellt, dass es sich bei $\mu$ um ein 
		Wahrscheinlichkeitsmaß auf $X$ handelt. Also muss unsere Behauptung in \eqref{4.9} tatsächlich gelten.
		
		Sei nun ein $\varepsilon > 0$ gegeben. Wir setzen 
		\[ S_\varepsilon 
			\; \defby \; \bigcap_{k \in \N} \bigcup_{m=1}^{M_{\varepsilon / 2^k}} 
			B_{\varepsilon / 2^k}(x_m) \]
		und verfahren nun ähnlich wie im Beweis von Satz~\ref{thm:tightness}: Offensichtlich ist $S_\varepsilon$ total beschränkt und damit auch 
		$ K_\varepsilon \defby \overline{S_\varepsilon}$ kompakt (vgl. \cite[Satz 2.3.8]{Simon.2015}). 
		Außerdem gilt für alle $\mu \in A$
		\[
			\mu(K_\varepsilon) \; \geq \; \mu(S_\varepsilon) \; = \; \mu\left(\bigcap_{k \in \N} \bigcup_{m=1}^{M_{\varepsilon / 2^k}} 
										B_{\varepsilon / 2^k}(x_m)\right)
			                      \; \geq \; 1 - \sum_{k=1}^{\infty} \frac{\varepsilon}{2^k} \; = \; 1 - \varepsilon
		\]
		und damit ist $A$ straff.
	\end{proof}
	
\end{document}