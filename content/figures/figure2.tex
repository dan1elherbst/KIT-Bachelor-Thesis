\documentclass[../wassersteinmetrik.tex]{subfiles}

\begin{document}
	
	\begin{figure}[H]
		\centering
		\begin{subfigure}[t]{0.5\textwidth}
			\centering
			\begin{tikzpicture}
				
				\definecolor{startcolor}{RGB}{0,156,129}
				\definecolor{endcolor}{RGB}{0,0,255}
				\definecolor{mixcolor}{RGB}{0,109,167}
				
				\begin{axis}[zmax=0.5,zmin=0,colormap={kit}{rgb255=(250,250,250) rgb255=(0,109,167)},grid=major,width=0.95\textwidth]
					\addplot3[color=endcolor,thick,samples y=0,samples=45,domain=-5:5] (x,5,{1/2 * gauss(x,3,0.7) + 3/8 * gauss(x,0,1) + 1/8 * gauss(x,-3,0.3)}) node[above, pos=0.4] {$f_{\nu}$}; 
					\addplot3[color=startcolor,thick,samples y=0,domain=-5:5] (-5,x,{(gauss(x,-1,1) + gauss(x,1.5,0.7))/2}) node[above, pos=0.4] {$f_{\mu}$}; 
					\addplot3[surf, domain=-5:5,domain y=-5:5] { 1.5 *
						1/4  * gauss2d(x, y, 3, -1, 0.7, 1, 0) + 1/4  * gauss2d(x, y, 3, 1.5, 0.7, 0.7, 0) + 
						3/16 * gauss2d(x, y, 0, -1, 1, 1, 0) + 3/16 * gauss2d(x, y, 0, 1.5, 1, 0.7, 0) + 
						1/16 * gauss2d(x, y, -3, -1, 0.3, 1, 0) + 1/16 * gauss2d(x, y, -3, 1.5, 0.3, 0.7, 0)
					}; 
					\node at (rel axis cs:0.45,0,0) [above, color=mixcolor] {$f_{\pi}$}; 
				\end{axis}
			\end{tikzpicture}
		\end{subfigure}%
		~
		\begin{subfigure}[t]{0.5\textwidth}
			\centering
			\begin{tikzpicture}
				
				\definecolor{startcolor}{RGB}{0,156,129}
				\definecolor{endcolor}{RGB}{0,0,255}
				\definecolor{mixcolor}{RGB}{0,109,167}
				
				\begin{axis}[zmax=0.5,zmin=0,colormap={kit}{rgb255=(250,250,250) rgb255=(0,109,167)},grid=major,width=0.95\textwidth]
					\addplot3[color=endcolor,thick,samples y=0,samples=45,domain=-5:5] (x,5,{1/2 * gauss(x,3,0.7) + 3/8 * gauss(x,0,1) + 1/8 * gauss(x,-3,0.3)}) node[above, pos=0.4] {$f_{\nu}$}; 
					\addplot3[color=startcolor,thick,samples y=0,domain=-5:5] (-5,x,{(gauss(x,-1,1) + gauss(x,1.5,0.7))/2}) node[above, pos=0.4] {$f_{\mu}$}; 
					\addplot3[surf, domain=-5:5,domain y=-5:5] { 1.5 *
						0.125  * gauss2d(x, y, 3, -1, 0.7, 1, -0.59) + 0.375  * gauss2d(x, y, 3, 1.5, 0.7, 0.7, 0.81) + 
						0.375 * gauss2d(x, y, 0, -1, 1, 1, -0.63) + 
						0.125 * gauss2d(x, y, -3, -1, 0.3, 1, 0.72)
					}; 
					\node at (rel axis cs:0.45,0,0) [above, color=mixcolor] {$f_{\pi}$};
				\end{axis}
			\end{tikzpicture}
		\end{subfigure}
		
		\vspace*{1em}
		
		\begin{subfigure}[t]{0.5\textwidth}
			\centering
			\begin{tikzpicture}
				
				\definecolor{startcolor}{RGB}{0,156,129}
				\definecolor{endcolor}{RGB}{0,0,255}
				\definecolor{mixcolor}{RGB}{0,109,167}
				
				\begin{axis}[zmax=0.5,zmin=0,colormap={kit}{rgb255=(250,250,250) rgb255=(0,109,167)},grid=major,width=0.95\textwidth]
					\addplot3[color=endcolor,thick,samples y=0,samples=45,domain=-5:5] (x,5,{1/2 * gauss(x,3,0.7) + 3/8 * gauss(x,0,1) + 1/8 * gauss(x,-3,0.3)}) node[above, pos=0.4] {$f_{\nu}$}; 
					\addplot3[color=startcolor,thick,samples y=0,domain=-5:5] (-5,x,{(gauss(x,-1,1) + gauss(x,1.5,0.7))/2}) node[above, pos=0.4] {$f_{\mu}$}; 
					\addplot3[surf, domain=-5:5,domain y=-5:5] { 1.5 *
						0.5  * gauss2d(x, y, 3, 1.5, 0.7, 0.7, 0) + 
						0.375 * gauss2d(x, y, 0, -1, 1, 1, 0.95) + 
						0.125 * gauss2d(x, y, -3, -1, 0.3, 1, 0.20)
					}; 
					\node at (rel axis cs:0.45,0,0) [above, color=mixcolor] {$f_{\pi}$};
				\end{axis}
			\end{tikzpicture}
		\end{subfigure}%
		~
		\begin{subfigure}[t]{0.5\textwidth}
			\centering
			\begin{tikzpicture}
				
				\definecolor{startcolor}{RGB}{0,156,129}
				\definecolor{endcolor}{RGB}{0,0,255}
				\definecolor{mixcolor}{RGB}{0,109,167}
				
				\begin{axis}[zmax=0.5,zmin=0,colormap={kit}{rgb255=(250,250,250) rgb255=(0,109,167)},grid=major,width=0.95\textwidth]
					\addplot3[color=endcolor,thick,samples y=0,samples=45,domain=-5:5] (x,5,{1/2 * gauss(x,3,0.7) + 3/8 * gauss(x,0,1) + 1/8 * gauss(x,-3,0.3)}) node[above, pos=0.4] {$f_{\nu}$}; 
					\addplot3[color=startcolor,thick,samples y=0,domain=-5:5] (-5,x,{(gauss(x,-1,1) + gauss(x,1.5,0.7))/2}) node[above, pos=0.4] {$f_{\mu}$}; 
					\addplot3[surf, domain=-5:5,domain y=-5:5] { 1.5 *
						1/4  * gauss2d(x, y, 3, -1, 0.7, 1, 0.9) + 1/4  * gauss2d(x, y, 3, 1.5, 0.7, 0.7, 0.9) + 
						3/16 * gauss2d(x, y, 0, -1, 1, 1, -0.9) + 3/16 * gauss2d(x, y, 0, 1.5, 1, 0.7, -0.9) + 
						1/16 * gauss2d(x, y, -3, -1, 0.3, 1, 0.9) + 1/16 * gauss2d(x, y, -3, 1.5, 0.3, 0.7, 0.9)
					}; 
					\node at (rel axis cs:0.45,0,0) [above, color=mixcolor] {$f_{\pi}$};
				\end{axis}
			\end{tikzpicture}
		\end{subfigure}
		\caption{Mögliche Dichten $f_{\pi}$ von Kopplungen zweier reeller Wahrscheinlichkeitsmaße $\mu$ und $\nu$ mit Dichten $f_{\mu}$, $f_{\nu}$.}
		\label{Fig:Main2}
	\end{figure}
	
\end{document}