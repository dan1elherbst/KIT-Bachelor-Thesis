\documentclass[../wassersteinmetrik.tex]{subfiles}

\begin{document}
	
	\begin{figure}[H]
		\centering
		\begin{subfigure}{0.47\textwidth}
			\centering
			\begin{tikzpicture}

				\definecolor{startcolor}{RGB}{0,156,129}
				\definecolor{endcolor}{RGB}{0,0,255}
				
				\begin{axis}[
					xmax=5, ymax=2, xmin=-5, ymin=-2, samples=200,
					width=0.85\textwidth,
					height=0.5\textwidth,
					axis y line=none,
					axis x line=middle,
					tick style={draw=none},
					xticklabels=\empty,
					x axis line style=-,
					axis equal
					]
					\addplot[startcolor, thick]{- 6 * (gauss(x,-1,1) + gauss(x,1.5,0.7))/2} node[below, pos=0.2] {$f_{\mu}$};
				\end{axis}
				
			\end{tikzpicture}
		\end{subfigure}%
		{\Large$\xrightarrow{?}$}%
		\begin{subfigure}{0.47\textwidth}
			\centering
			\begin{tikzpicture}
				
				\definecolor{startcolor}{RGB}{0,156,129}
				\definecolor{endcolor}{RGB}{0,0,255}
				
				\begin{axis}[
					xmax=5, ymax=2, xmin=-5, ymin=-2, samples=200,
					width=0.85\textwidth,
					height=0.5\textwidth,
					axis y line=none,
					axis x line=middle,
					tick style={draw=none},
					xticklabels=\empty,
					x axis line style=-,
					axis equal
					]
					\addplot[endcolor, thick]{6 * (1/2 * gauss(x,3,0.7) + 3/8 * gauss(x,0,1) + 1/8 * gauss(x,-3,0.3))} node[above, pos=0.4] {$f_{\nu}$};
				\end{axis}
				
			\end{tikzpicture}
		\end{subfigure}
		\caption{Mögliche Dichten $f_{\mu}$, $f_{\nu}$ von $\mu$ und $\nu$, die die räumliche Verteilung der Erde bzw. Masse (hier in $\R$) darstellen.}
		\label{Fig:Main3}
	\end{figure}
	
\end{document}