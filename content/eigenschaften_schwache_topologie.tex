\documentclass[../main/main.tex]{subfiles}

\begin{document}
	
	\section{Eigenschaften der schwachen Topologie}
	
	Nachdem wir im vorigen Abschnitt die schwache Topologie definiert haben, möchten wir uns nun einige Eigenschaften dieser ansehen, wenn man weitere Bedingungen an den Grundraum $X$ stellt. 
	Als Grundbedingung möchten wir in diesem Kapitel die Metrisierbarkeit von $X$ fordern.
	Unter dieser Voraussetzung werden die folgenden Aussagen unsere zentralen Erkenntnisse sein:
	\begin{itemize}
		\item $\Finitemeasures{X}$ ist genau dann metrisierbar und separabel, wenn $X$ separabel ist.
		\item $\Finitemeasures{X}$ ist genau dann polnisch, wenn $X$ polnisch ist.
		\item $\Probmeasures{X}$ ist genau dann metrisierbar und kompakt, wenn $X$ kompakt ist.
	\end{itemize}
	Abgesehen von Satz~\ref{thm:compactimpliescompactmeasures} orientieren wir uns in diesem Abschnitt an \cite{Varadarajan.1958}.
	
	\begin{Satz}
		\label{thm:embeddingdiracmeasures}
		Sei $X$ ein metrisierbarer topologischer Raum. Für $x \in X$ schreiben wir $\delta_x \in \Probmeasures{X}$ für das Dirac-Maß im Punkt $x$ und definieren
		\[ \fctmap{\delta}{X}{\Finitemeasures{X}}{x}{\delta_x} \text{.} \]
		Außerdem setzen wir $\Delta \defby \delta(X) = \setcomp{\delta_x}{x \in X}$. Dann gelten die folgenden Aussagen:
		\begin{enumeratethm}
			\item $\delta$ ist ein Homöomorphismus zwischen $X$ und $\Delta$.
			\item $\Delta \subseteq \Finitemeasures{X}$ ist sequentiell abgeschlossen.
		\end{enumeratethm}
	\end{Satz}

	\begin{proof}
		Wir beweisen zunächst Aussage (a). Weil $X$ metrisierbar ist und die einelementigen Mengen $\set{x}, \; x \in X$ daher in $\mathcal{B}(X)$ enthalten sind, ist $\delta$ sicherlich injektiv.
		Wegen Satz~\ref{thm:netconvergence} (b)) genügt es nachzuweisen, dass für jedes Netz $(x_\iota)_\iota$ in $X$ sowie $z \in X$ die Äquivalenz
		\[ x_\iota \to z \quad \iff \quad \delta_{x_\iota} \xrightarrow{w} \delta_z, \quad \iota \to \infty \]
		gilt. 
		
		Gelte für ein Netz $(x_\iota)_\iota$ in $X$ und $z \in X$ die Konvergenz $x_\iota \to z$ und sei $f \in \Bdcontfct{X}$. Dann ist
		\[ \measureint{}{f}{\delta_{x_\iota}} \; = \; f(x_\iota) \; \to \; f(z) = \measureint{}{f}{\delta_{z}}, \quad \iota \to \infty \text{,} \]
		also konvergiert $\delta_{x_\iota}$ schwach gegen $\delta_z$. Gelte umgekehrt $\delta_{x_\iota} \xrightarrow{w} \delta_z$. Wähle eine Metrik $d$, die $X$ metrisiert und
		setze $f \defby \min \set{1, d(\cdot, z)} \in \Bdcontfct{X}$. Dann ist 
		\[ \min \set{1, d(x_\iota, z)} \; = \; \measureint{}{f}{\delta_{x_\iota}} \; \to \; \measureint{}{f}{\delta_{z}} = 0 \text{,} \]
		was $x_\iota \to z$ nach sich zieht.
		
		Nun möchten wir Aussage (b) zeigen. Sei $(x_n)_n \in X^\N$ und sei $\mu \in \Finitemeasures{X}$ mit $\delta_{x_n} \xrightarrow{w} \mu$, $n \to \infty$. Zunächst gilt sicherlich $\mu \in \Probmeasures{X}$.
		Wir zeigen, dass $(x_n)_n$ eine konvergente Teilfolge besitzen muss. Ansonsten ist nämlich $D \defby \setcomp{x_n}{n \in \N} \subseteq X$ eine unendliche und abgeschlossene Teilmenge. Das Portmanteau-Theorem (Satz~\ref{thm:portmanteau}) liefert also für jede unendliche Teilmenge $\tilde{D} \subseteq D$ die Abschätzung
		\[ 1 \; = \; \limsup_{n \to \infty} \delta_{x_n}(\tilde{D}) \; \leq \; \mu(\tilde{D}) \; \leq \; 1 \text{,} \]
		die einen Widerspruch darstellt. Also existiert eine Teilfolge $(x_{n_k})_k$ und $x \in X$ mit $x_{n_k} \to x, \; x \to \infty$. Damit ist aber auch $\delta_{x_n} \xrightarrow{w} \delta_x$, weshalb nach
		Satz~\ref{thm:measureequality} die Gleichheit $\mu = \delta_x \in \Delta$ folgt.
	\end{proof}
	
	\begin{Satz}
		\label{thm:finitemeasuresmetrizableseparable}
		Sei $X$ ein metrisierbarer topologischer Raum. Dann ist $\Finitemeasures{X}$ ist genau dann metrisierbar und separabel, wenn $X$ separabel ist.
	\end{Satz}

	Für den Beweis benötigen wir zunächst noch einige wenige Hilfssätze, die wir im Folgenden vorstellen möchten.
	
	\begin{Hilfssatz}
		\label{lem:bdcontfctbanach}
		Sei $X$ ein topologischer Raum. Dann ist $(\Bdcontfct{X}, \norm{\cdot}_\infty)$ ein Banachraum. Ist $X$ kompakt, so ist $\Bdcontfct{X}$ separabel.
	\end{Hilfssatz}

	\begin{proof}
		\tobechanged{Ohne Beweis.}
	\end{proof}
	
	\begin{Hilfssatz}
		\label{lem:bduniffctbanach}
		Sei $(X, d)$ ein totalbeschränkter metrischer Raum. Dann ist $\Bduniffct{X}$ ein separabler Banachraum.
	\end{Hilfssatz}
	
	\begin{proof}
		Zunächst ist $\Bduniffct{X}$ offensichtlich ein Banachraum.
		Sei nun $\tilde{X}$ die Vervollständigung von $X$. Diese ist kompakt und damit ist $\Bdcontfct{\tilde{X}}$ nach Hilfssatz~\ref{lem:bdcontfctbanach} ein separabler Banachraum.
		Jedes $f \in \Bduniffct{X}$ besitzt eine eindeutige Fortsetzung $\tilde{f} \in \Bdcontfct{\tilde{X}}$ mit $\norm{f}_\infty = \norm{\tilde{f}}_\infty$. Damit wird über die Zuordnung
		$f \mapsto \tilde{f}$ eine isometrische Einbettung von $\Bduniffct{X}$ nach $\Bdcontfct{\tilde{X}}$ definiert, was die Separabilität von $\Bduniffct{X}$ impliziert.
	\end{proof}
	
	\begin{Hilfssatz}
		\label{lem:R}
		Sei $R \defby \R^\N$ ausgestattet mit der Produkttopologie von $\R$, also mit der Initialtopologie der Projektionen $\fctmap{\pi_k}{R}{\R}{x = (x_n)_n}{x_k}, \; k \in \N$. Dann ist $R$ 
		separabel und metrisierbar.
	\end{Hilfssatz}
	
	\begin{proof}
		$R \cong (0, 1)^\N$ ist homöomorph zu einer Teilmenge des Hilbertwürfels $H$, worauf mit Satz~\ref{thm:characterizationpolishspaces} unmittelbar die Behauptung folgt.
	\end{proof}

	Ausgestattet mit den obigen Hilfssätzen können wir nun zum Beweis von Satz~\ref{thm:finitemeasuresmetrizableseparable} übergehen. Im Wesentlichen werden wir eine Einbettung von $\Finitemeasures{X}$ nach $R$ konstruieren, die uns unmittelbar die Behauptung liefert.
	
	\begin{proof}[Beweis von Satz~\ref{thm:finitemeasuresmetrizableseparable}]
		Für die Hinrichtung fixieren wir auf $X$ eine Metrik $d$, die $X$ metrisiert. Nach Hilfsssatz~\ref{lem:characterizationpolishspaces} existiert dann ein Homöomorphismus $\fct{\varphi}{X}{H}$ von
		$X$ auf eine Teilmenge $\varphi(X)$ des Hilbertwürfels $H$. Wir wählen nun die Metrik $\rho$ aus Hilfssatz~\ref{lem:hilbertcube} auf $H$. Aufgrund der Kompaktheit von $H$ ist
		$(H, \rho)$ totalbeschränkt und damit auch jede Teilmenge von $H$. Also können wir über $\rho$ eine Metrik $\tilde{d}$ auf $X$ einführen, sodass $X$ von $\tilde{d}$ 
		metrisiert wird und $(X, \tilde{d})$ totalbeschränkt ist. Im Folgenden sei auf $X$ diese Metrik fixiert. Hilfssatz~\ref{lem:bduniffctbanach} liefert uns nun, dass 
		$(\Bduniffct{X}, \norm{\cdot}_\infty)$ ein separabler Banachraum ist. Sei nun $\mathcal{D} \defby \setcomp{f_n}{n \in \N} \subseteq \Bduniffct{X}$ eine abzählbare dichte Teilmenge, 
		die wir so wählen, dass $f_1 = \indfct_X$ ist.
		
		Wir setzen nun
		\[ \fctmap{T}{\Finitemeasures{X}}{R}{\mu}{\left( \measureint{}{f_n}{\mu} \right)_{\! n}} \label{4.1} \tag{4.1} \]
		und möchten nachweisen, dass $T$ ein Homöomorphismus auf sein Bild ist. Da $R$ nach Hilfssatz~\ref{lem:R} separabel und metrisierbar ist, übertragen sich diese Eigenschaften dann auch auf $\Finitemeasures{X}$.
		
		Wir zeigen zunächst die Injektivität. Seien dazu also $\mu, \nu \in \Finitemeasures{X}$ mit $T(\mu) = T(\nu)$. 
		Sei außerdem $g \in \Bduniffct{X}$. Dann gibt es eine Folge $(g_n)_n \in \mathcal{D}^\N$ mit $\norm{g - g_n}_\infty \to 0, \; n \to \infty$, was auch 
		\[ \measureint{}{g_n}{\mu} \to \measureint{}{g}{\mu} \quad \text{und} \quad \measureint{}{g_n}{\nu} \to \measureint{}{g}{\nu}, \quad n \to \infty \]
		impliziert. Da aber $\measureint{}{g_n}{\mu} = \measureint{}{g_n}{\nu}$ für alle $n \in \N$ gilt, folgt auch $\measureint{}{g}{\mu} = \measureint{}{g}{\nu}$.
		Satz~\ref{thm:measureequality} liefert nun die Gleichheit $\mu = \nu$.
		
		Um nun den Beweis abzuschließen, genügt es wegen Satz~\ref{thm:netconvergence} zu beweisen, dass für jedes Netz $(\mu_\iota)_{\iota \in I}$ in $\Finitemeasures{X}$ und $\mu \in \Finitemeasures{X}$ die
		Äquivalenz
		\[ \mu_\iota \xrightarrow{w} \mu \quad \iff \quad T(\mu_\iota) \to T(\mu), \quad \iota \to \infty  \]
		gilt.
		
		Gelte $\mu_\iota \xrightarrow{w} \mu$, was nach Satz~\ref{thm:portmanteau} äquivalent dazu ist, dass $\lim_{\iota \to \infty} \measureint{}{f}{\mu_\iota} = \measureint{}{f}{\mu}$ für alle 
		$f \in \Bduniffct{X}$ ist. Offenbar impliziert dies die Konvergenz $T(\mu_\iota) \to T(\mu), \; \iota \to \infty$.
		
		Sei nun umgekehrt $T(\mu_\iota) \to T(\mu), \; \iota \to \infty$, was gleichbedeutend damit ist, dass 
		\[ \lim_{\iota \to \infty} \measureint{}{f_n}{\mu_\iota} = \measureint{}{f_n}{\mu} \label{4.2} \tag{4.2} \] 
		für alle $n \in \N$ gilt.
		Wegen $f_1 = \indfct_X$ gilt auch
		\[ \mu_\iota(X) \; = \; \measureint{}{\indfct_X}{\mu_\iota} \; \to \; \measureint{}{\indfct_X}{\mu} \; = \; \mu(X), \quad \iota \to \infty \]
		und damit können wir eine Konstante $C < \infty$ und ein $\iota_0 \in I$ finden, sodass $\mu_\iota(X) \leq C$ für alle $\iota_0 \preceq \iota$ ist.
		Sei $g \in \Bduniffct{X}$ fest aber beliebig. Dann gibt es eine Folge $(g_n)_n \in \mathcal{D}^\N$ mit $\norm{g - g_n}_\infty \to 0, \; n \to \infty$ und für jedes $n \in \N$, $\iota_0 \preceq \iota$ folgt
		\begin{align*}
			\left| \measureint{}{f}{\mu_\iota} - \measureint{}{f}{\mu} \right| \; &\leq \; 
			\left| \measureint{}{f}{\mu_\iota} - \measureint{}{f_n}{\mu_\iota} \right| + 
			\left| \measureint{}{f_n}{\mu_\iota} - \measureint{}{f_n}{\mu} \right| \\
			& \qquad + 
			\left| \measureint{}{f_n}{\mu} - \measureint{}{f}{\mu} \right| \\
			&\leq \; 2 C \norm{f - f_n}_\infty + \left| \measureint{}{f_n}{\mu_\iota} - 
			\measureint{}{f_n}{\mu} \right| \text{,}
		\end{align*}
		woraus schließlich mit \eqref{4.2}
		\[ \limsup_{\iota \to \infty} \left| \measureint{}{f}{\mu_\iota} - \measureint{}{f}{\mu} \right| \; \leq \; 2 C \norm{f - f_n}_\infty \; \to \; 0, \quad n \to \infty \]
		folgt. Damit gilt 
		\[ \measureint{}{f}{\mu_\iota} \; \to \; \measureint{}{f}{\mu}, \quad \iota \to \infty \]
		für alle $f \in \Bduniffct{X}$, was nach Satz~\ref{thm:portmanteau} die schwache Konvergenz $\mu_\iota \xrightarrow{w} \mu, \; \iota \to \infty$ impliziert.
		
		Die Rückrichtung folgt direkt aus Satz~\ref{thm:embeddingdiracmeasures}, da die Separabilität von $\Finitemeasures{X}$ direkt auch die Separabilität von $X \cong \Delta \subseteq \Finitemeasures{X}$ nach sich zieht.
	\end{proof}

	Wir werden uns nun der zweiten Aussage vom Beginn dieses Abschnittes widmen. 
	
	\begin{Satz}
		\label{thm:finitemeasurespolish}
		Sei $X$ ein metrisierbarer topologischer Raum. Dann ist $\Finitemeasures{X}$ genau dann polnisch, wenn $X$ polnisch ist.
	\end{Satz}
	
	Für die Hinrichtung zeigen wir in einem Hilfssatz zunächst den wichtigen Spezialfall, dass der Grundraum $X$ kompakt ist und weiten dies anschließend auf die Klasse aller polnischen Räume aus.

	\begin{Hilfssatz}
		\label{lem:compactimpliespolishmeasures}
		Sei $X$ ein kompakter metrisierbarer topologischer Raum. Dann ist $\Finitemeasures{X}$ polnisch.
	\end{Hilfssatz}
	
	\begin{proof}
		Wir wählen zunächst eine Metrik $d$, die $X$ vollständig metrisiert. Da $X$ insbesondere separabel ist, können wir zunächst genau wie 
		im Beweis von Satz~\ref{thm:finitemeasuresmetrizableseparable} verfahren
		(abgesehen davon, dass wir anstelle von $\tilde{d}$ direkt mit $d$ arbeiten). Für eine abzählbare dichte Teilmenge 
		$\mathcal{D} \defby \setcomp{f_n}{n \in \N} \subseteq \Bduniffct{X}$ mit $f_1 = \indfct_X$ erhalten wir also wieder die Abbildung
		\[ \fctmap{T}{\Finitemeasures{X}}{R}{\mu}{\left( \measureint{}{f_n}{\mu} \right)_{\! n}} \text{,} \]
		die ein Homöomorphismus auf ihr Bild ist. 
		
		Nun möchten wir nachweisen, dass $T(\Finitemeasures{X}) \subseteq R$ abgeschlossen ist. Hierfür verwenden wir die Charakterisierung von Abgeschlossenheit 
		durch Netze aus aus Satz~\ref{thm:netconvergence} (a). Ist $(\mu_\iota)_{\iota \in I}$ ein Netz in $\Finitemeasures{X}$ mit $T(\mu_\iota) \to r, \; \iota \to \infty$
		für ein $r = (r_n)_n \in R$, so müssen wir zeigen, dass $r$ in $T(\Finitemeasures{X})$ enthalten ist. Sicherlich gilt
		$r_n = \lim_{\iota \to \infty} \measureint{}{f_n}{\mu_\iota}$ für alle $n \in \N$ und analog zum Beweis von Satz~\ref{thm:finitemeasuresmetrizableseparable} 
		können wir wieder eine Konstante $C < \infty$ und ein $\iota_0 \in I$ finden, sodass $\mu_\iota(X) \leq C$ für alle $\iota_0 \preceq \iota$ ist.
		
		Wir setzen nun $A \defby \mathrm{span}_\R \, \mathcal{D}$, was ein dichter Untervektorraum von $\Bduniffct{X}$ ist. Auf $A$ definieren wir nun
		\[ \fctmap{l}{A}{\R}{f}{\lim_{\iota \to \infty} \measureint{}{f}{\mu_\iota}} \text{.} \]
		Offenbar ist $l$ wohndefiniert und linear. Da $l(f) \leq C \norm{f}_\infty$ für alle $f \in \Bduniffct{X}$ gilt, folgt $l \in A^\ast$. Nach dem Satz von Hahn-Banach
		existiert nun eine (eindeutige) Fortsetzung von $l$ auf ganz $\Bduniffct{X}$, die wir ab jetzt $l$ nennen. Es ist $l \in \Bduniffct{X}^\ast$ 
		mit $\norm{l}_{\Bduniffct{X}^\ast} = \norm{l}_{A^\ast}$. Offenbar gilt $l(f) \geq 0$ für alle $f \in \Bdcontfct{X}$ mit $f \geq 0$ und damit gibt es 
		nach dem Darstellungssatz von Riesz-Markov ein Maß $\mu \in \Finitemeasures{X}$ mit 
		\[ l(f) \; = \; \measureint{}{f}{\mu}, \quad f \in \Bduniffct{X} \text{.} \]
		Also folgt
		\[ \lim_{\iota \to \infty} \measureint{}{f_n}{\mu_\iota} \; = \; r_n \; = \; l(f_n) \; = \measureint{}{f_n}{\mu} \text{.} \]
		für alle $n \in \N$. In völliger Analogie zum Beweis von Satz~\ref{thm:finitemeasuresmetrizableseparable} folgt nun die schwache Konvergenz $\mu_\iota \xrightarrow{w} \mu, \; \iota \to \infty$,
		die wiederum $T(\mu_\iota) \to T(\mu)$ impliziert. Demnach ist $r = T(\mu) \in T(\Finitemeasures{X})$ und insgesamt haben wir gezeigt, dass $T(\Finitemeasures{X}) \subseteq R$ abgeschlossen ist.
	\end{proof}

	\begin{proof}[Beweis von Satz~\ref{thm:finitemeasurespolish}]
		Wir widmen uns zunächst der Hinrichtung des Beweises. Analog zum Beweis von Satz~\ref{thm:finitemeasuresmetrizableseparable} fixieren wir auf 
		$X$ eine Metrik $\tilde{d}$, bezüglich der $X$ totalbeschränkt ist und
		bezeichnen die Vervollständigung von $X$ bezüglich dieser Metrik mit $\tilde{d}$. $\tilde{X}$ ist als kompakter metrischer Raum polnisch, weshalb $X \subseteq \tilde{X}$
		nach Satz~\ref{thm:gdeltasubsetsofpolishspaces} eine $G_\delta$-Teilmenge ist. Indem wir ein Maß $\mu \in \Finitemeasures{X}$ mittels 
		$\mu(B) \defby \mu(B \cap X), \; B \in \mathcal{B}(\tilde{X})$ auf $\tilde{X}$ fortsetzen, können wir $\Finitemeasures{X}$ als Teilmenge von $\Finitemeasures{\tilde{X}}$
		auffassen. Da $\Finitemeasures{\tilde{X}}$ nach 
		Hilfssatz~\ref{lem:compactimpliespolishmeasures} polnisch ist, genügt es also nach Satz~\ref{thm:gdeltasubsetsofpolishspaces} zu zeigen, 
		dass $\Finitemeasures{X} \subseteq \Finitemeasures{\tilde{X}}$ eine $G_\delta$-Teilmenge ist.
		
		Da $X \subseteq \tilde{X}$ eine $G_\delta$-Teilmenge ist, existieren offene Mengen $U_k \subseteq \tilde{X}, \; k \in \N$ so, dass 
		$X \; = \; \bigcap_{k \in \N} U_k$
		gilt. Insbesondere gilt damit auch
		$\Finitemeasures{X} \; = \; \bigcap_{k \in \N} \Finitemeasures{U_k}$.
		Für $k, r \in \N$ setzen wir nun
		\[ V_{k, r} \; \defby \; \setcomp{\mu \in \Finitemeasures{\tilde{X}}}{\mu(U_k^{\mathsf{c}}) < \frac{1}{r}} \text{.} \]
		Mit diesen Mengen folgt dann $\Finitemeasures{U_k} = \bigcap_{r \in \N} V_{k, r}$ für alle $k \in \N$ und insbesondere auch
		\[ \Finitemeasures{X} \; = \; \bigcap_{k \in \N} \bigcap_{r \in \N} V_{k, r} \text{.} \]
		Es genügt also, zu beweisen, dass $V_{k, r}$ für alle $k, r \in \N$ offen ist, was äquivalent zur Abgeschlossenheit von 
		\[C_{k, r} \; \defby \; V_{k, r}^{\mathsf{c}} \; = \; \setcomp{\mu \in \Finitemeasures{\tilde{X}}}{\mu(U_k^{\mathsf{c}}) \geq \frac{1}{r}} \] 
		für alle $k, r \in \N$ ist. Wir verwenden hierfür die Charakterisierung abgeschlossener Mengen durch Netze aus Satz~\ref{thm:netconvergence} (a).
		Fixiere $k, r \in \N$. Sei nun $(\mu_\iota)_{\iota \in I}$ ein Netz in $C_{k, r}$, das schwach gegen ein $\mu \in \Finitemeasures{\tilde{X}}$ konvergiert.
		Dann ist $\mu_\iota(U_k^{\mathsf{c}}) \geq \frac{1}{r}$ für alle $\iota \in I$ und damit nach dem Portmanteau-Theorem (Satz~\ref{thm:portmanteau} (iii))
		\[ \mu(U_k^{\mathsf{c}}) \; \geq \; \limsup_{\iota \to \infty} \mu_\iota(U_k^{\mathsf{c}}) \; \geq \; \frac{1}{r} \text{,} \]
		was $\mu \in C_{k, r}$ impliziert. Daher ist $C_{k, r}$ für alle $k, r \in \N$ abgeschlossen und der Beweis vollständig.
		
		Die Rückrichtung folgt wieder unmittelbar aus Satz~\ref{thm:embeddingdiracmeasures}: Ist $\Finitemeasures{X}$ polnisch, 
		so ist betrachten wir $X \cong \Delta \subseteq \Finitemeasures{X}$. Weil $\Delta$ wegen der Metrisierbarkeit von $\Finitemeasures{X}$ nun auch abgeschlossen und damit nach 
		Satz~\ref{lem:opensets} eine $G_\delta$-Teilmenge von $\Finitemeasures{X}$ ist, liefert Satz~\ref{thm:gdeltasubsetsofpolishspaces} die Behauptung.
	\end{proof}

	\begin{Bemerkung}
		Satz~\ref{thm:finitemeasuresmetrizableseparable} und Satz~\ref{thm:finitemeasurespolish} gelten auch, wenn man $\Finitemeasures{X}$ durch $\Probmeasures{X}$ ersetzt 
		(für Satz~\ref{thm:finitemeasurespolish} folgt dies aus der Abgeschlossenheit von $\Probmeasures{X} \subseteq \Finitemeasures{X}$ und Hilfssatz~\ref{lem:opensets}).
	\end{Bemerkung}

	\begin{Satz}
		\label{thm:compactimpliescompactmeasures}
		Sei $X$ ein metrisierbarer topologischer Raum. Dann ist $\Probmeasures{X}$ genau dann metrisierbar und kompakt, wenn $X$ kompakt ist.
	\end{Satz}
	
	\begin{proof}
		Im Wesentlichen stellen wir den Beweis von Proposition 5.3 in \cite{vanGaans.200203} vor.
		
		Für die Hinrichtung sei $X$ zunächst kompakt. Dann ist $X$ insbesondere separabel und damit ist $\Probmeasures{X}$ 
		nach Satz~\ref{thm:finitemeasuresmetrizableseparable} metrisierbar.
		Außerdem setzen wir
		\[ A \; \defby \; \setcomp{l \in \Bdcontfct{X}^\ast}{\norm{l}_{\Bdcontfct{X}^\ast} \leq 1, \; l(\indfct_{X}) = 1, \; 
			\forall f \in \Bdcontfct{X} \text{ mit } f \geq 0 : l(f) \geq 0}\]
		und analog zu \eqref{5.1}
		\[ \fctmap{\Phi}{\Probmeasures{X}}{A}{\mu}{\left[\fctmap{l_\mu}{\Bdcontfct{X}}{\R}{f}{\measureint{}{f}{\mu}}\right]} \text{.} \]
		Der Darstellungssatz von Riesz-Markov liefert die Bijektivität von $\Phi$, also ist $\Phi$ ein 
		Homöomorphismus zwischen $\Probmeasures{X}$ und $A \subseteq \Bdcontfct{X}^\ast$ 
		mit der Schwach-$\ast$-Topologie. Mit dem Satz von Banach-Alaoglu folgt nun, dass $A$ als abgeschlossene Teilmenge der kompakten Menge 
		$\setcomp{l \in C_b(X)^\ast}{\norm{l}_{\Bdcontfct{X}^\ast} \leq 1}$ selbst kompakt bezüglich der Schwach-$\ast$-Topologie ist, 
		was schließlich die Kompaktheit von $\Probmeasures{X}$ nach sich zieht.
		
		Die Rückrichtung ist wieder eine direkte Konsequenz aus Satz~\ref{thm:embeddingdiracmeasures}: Ist $\Probmeasures{X}$ metrisierbar und kompakt, so ist 
		$X \cong \Delta$ als abgeschlossene Teilmenge von $\Probmeasures{X}$ kompakt.
	\end{proof}
	
\end{document}