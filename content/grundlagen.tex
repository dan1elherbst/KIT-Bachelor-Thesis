\documentclass[../main/main.tex]{subfiles}

\begin{document}
	
	\section{Grundlagen}
	
	In diesem Abschnitt möchten wir einige Begriffe und Schreibweisen festlegen sowie ein paar wenige Grundlagen vorstellen, 
	die wir im weiteren Verlauf der Arbeit benötigen werden. 
	In Teilabschnitt~\ref{subsec:topologische_grundlagen} führen wir zunächst zwei grundlegende Konzepte aus der Topologie ein, während wir
	uns in Teilabschnitt~\ref{subsec:borel-maße_auf_topologischen_räumen} neben der Definition einiger Begriffe mit der 
	\emph{Regularität} bzw. \emph{schwachen Regularität} eine grundlegende Approximationseigenschaft von Maßen ansehen werden.
	
	\subsection{Topologische Grundlagen}
	\label{subsec:topologische_grundlagen}
	
	\subsubsection*{Netze}
	
	Im Allgemeinen beschreiben Folgen und deren Konvergenz die Eigenschaften eines topologischen Raums nur unzureichend:
	Beispielsweise ist Folgenstetigkeit nicht notwendigerweise äquivalent zu Stetigkeit und zwei unterschiedliche Topologien auf einer 
	gegebenen Menge können generell dieselben konvergenten Folgen haben (siehe etwa \cite[Beispiel 2.6.1]{Simon.2015}). 
	Um solche Probleme im Folgenden zu umgehen, führen wir das
	Konzept des Netzes aus der mengentheoretischen Topologie ein. Netze verallgemeinern Folgen und können in vielerlei Hinsicht die 
	Eigenschaften eines topologischen Raums besser erfassen. In diesem Teilabschnitt möchten wir auf einige Grundlagen zu Netzen eingehen,
	wobei wir uns überwiegend an \cite[Kapitel 2.6]{Simon.2015} orientieren.
	
	\begin{Definition}[Gerichtete Menge]
		\label{def:gerichtete_menge}
		Sei $I$ eine Menge. Eine Relation $\preceq$ auf $I$ mit
		\begin{enumeratethm}
			\item $\forall \iota \in I: \; \iota \preceq \iota$
			\item $\forall \iota, \kappa, \lambda \in I: \; \iota \preceq \kappa$ und $\kappa \preceq \lambda$ impliziert $\iota \preceq \lambda$
			\item $\forall \iota, \kappa \in I: \; \exists \lambda \in I: \; \iota \preceq \lambda$ und $\kappa \preceq \lambda$
		\end{enumeratethm}
		heißt \emph{gerichtet}. In diesem Fall nennen wir das Paar $(I, \preceq)$ auch eine \emph{gerichtete Menge}.
	\end{Definition}
	
	\begin{Definition}[Netz]
		\label{def:netz}
		Sei $\X$ eine Menge und $(I, \preceq)$ eine gerichtete Menge. Ein \emph{Netz} in $\X$ ist dann eine Abbildung $\fct{x}{I}{\X}$. In diesem
		Fall schreiben wir auch $x = (x_\iota)_{\iota \in I}$.
	\end{Definition}
	
	\begin{Definition}[Konvergenz von Netzen]
		Sei $\X$ ein topologischer Raum und sei $x = (x_\iota)_{\iota \in I}$ ein Netz in $\X$. Wir sagen, dass $x$ gegen $z \in \X$ konvergiert, falls
		es für jede Umgebung $U$ von $z$ ein derartiges $\iota_0 \in I$ gibt, dass $x_\iota \in U$ für alle $\iota_0 \preceq \iota \in I$ erfüllt ist.
		In diesem Fall schreiben wir auch $\lim_{\iota \to \infty} x_\iota = z$ oder $x_\iota \to z, \; \iota \to \infty$.
	\end{Definition}
	
	\begin{Bemerkung}
		Offenbar sind Folgen einfach Netze, bei denen $(I, \preceq) \defby (\N, \leq)$ gewählt wird. Der Konvergenzbegriff von Netzen stimmt
		dann auch mit dem von uns bekannten Konvergenzbegriff von Folgen in topologischen Räumen überein.
	\end{Bemerkung}
	
	Mit dem folgenden Satz konkretisieren wir unsere zu Beginn dieses Abschnitts getroffene Aussage, dass Netze die Eigenschaften topologischer Räume
	besser als Folgen erfassen: Tatsächlich lässt sich jede Topologie einzig und allein durch ihre konvergenten Netze charakterisieren.
	
	\begin{Satz}
		\label{satz:netz_konvergenz}
		Seien $\X$ und $\Y$ topologische Räume. Dann gelten die folgenden Aussagen:
		\begin{enumeratethm}
			\item Eine Teilmenge $C \subseteq \X$ ist genau dann abgeschlossen, wenn der Grenzwert $z \in \X$ eines jeden konvergenten Netzes 
			$x = (x_\iota)_{\iota \in I}$ in $\X$ bereits in $C$ liegt.
			\item Eine Funktion $\fct{f}{\X}{\Y}$ ist genau dann stetig, wenn für jedes konvergente Netz $x = (x_\iota)_{\iota \in I}$ in $\X$ 
			\[ \lim_{\iota \to \infty} f(x_\iota) \; = \; f(\lim_{\iota \to \infty} x_\iota) \]
			erfüllt ist.
			\item Zwei Topologien auf $\X$ sind genau dann identisch, wenn alle Netze in $\X$ bezüglich beider Topologien dasselbe Konvergenzverhalten aufweisen.
		\end{enumeratethm}
	\end{Satz}
	
	\begin{proof}
		Siehe \cite[Satz 2.6.3]{Simon.2015}.
	\end{proof}
	
	\begin{Bemerkung}
		Auch wenn man in spezielleren Fällen (wie etwa bei metrisierbarem $\X$) problemlos mit Folgen arbeiten kann, sei an dieser Stelle noch einmal angemerkt, dass im Allgemeinen 
		keine der drei Aussagen gelten, wenn man \enquote{Netz} durch \enquote{Folge} ersetzt (vgl. \cite[Beispiel 2.6.1]{Simon.2015}).
	\end{Bemerkung}
	
	Für reelle Netze können wir in Analogie zu Folgen auch den Limes Superior bzw. Inferior von Netzen definieren. Hierfür orientieren wir uns an
	\cite[Kapitel 2.1 und Aufgabe 2.55]{Megginson.1998} und \cite[Kapitel 2.4]{Aliprantis.2006}.
	
	\begin{Definition}[Limes Superior und Inferior von Netzen]
		Sei $x = (x_\iota)_{\iota \in I}$ ein Netz in $\R$. Dann definieren wir 
		\[ \limsup_{\iota \to \infty} x_\iota \; \defby \; \inf_{\iota \in I} \sup_{\iota \preceq \kappa} x_\kappa \quad \text{bzw.} \quad 
		\liminf_{\iota \to \infty} x_\iota \; \defby \; \sup_{\iota \in I} \inf_{\iota \preceq \kappa} x_\kappa \]
		und nennen diese den \emph{Limes Superior} bzw. \emph{Limes Inferior} von $x$.
	\end{Definition}
	
	\begin{Bemerkung}
		Offenbar ist auch diese Definition mit den bekannten Begriffen für reelle Folgen kompatibel. Die meisten grundlegenden Eigenschaften der Konvergenz und des 
		Limes Superior bzw. Inferior von reellen Folgen lassen sich auch auf reelle Netze übertragen, 
		wie etwa die üblichen Rechenregeln für die Grenzwerte von Summen und Produkten reeller Folgen.
		
		Außerdem gelten etwa für reelle Netze $x = (x_\iota)_{\iota \in I}$ und
		$y = (y_\iota)_{\iota \in I}$ die Ungleichungen
		\begin{align*}
			\limsup_{\iota \to \infty} \, (x_\iota + y_\iota) \; &\leq \; \limsup_{\iota \to \infty} x_\iota + \limsup_{\iota \to \infty} y_\iota \quad \text{und}\\
			\liminf_{\iota \to \infty} \, (x_\iota + y_\iota) \; &\geq \; \liminf_{\iota \to \infty} x_\iota + \liminf_{\iota \to \infty} y_\iota \text{,}
		\end{align*}
		sofern die Summen jeweils definiert sind und
		mit Gleichheit, falls eines der beiden Netze konvergiert (vgl. \cite[Aufgabe 2.55 (e)]{Megginson.1998}).
		Nach \cite[Aufgabe 2.55 (i)]{Megginson.1998} ist für ein konvergentes Netz $x = (x_\iota)_{\iota \in I}$ in $\R$ ebenfalls
		\[ \lim_{\iota \to \infty} x_\iota \; = \; \limsup_{\iota \to \infty} x_\iota \; = \; \liminf_{\iota \to \infty} x_\iota \text{.} \]
	\end{Bemerkung}
	
	\subsubsection*{Initialtopologie}
	
	Wir möchten nun noch ein weiteres Konzept aus der Topologie, die sogenannte \emph{Initialtopologie} einführen.
	Auch wenn es sich hier um einen sehr elementaren Begriff handelt, wird dieser uns an späterer Stelle gerade 
	in Kombination mit Hilfssatz~\ref{hilfssatz:konvergenz_initialtopologie} häufig ermöglichen, 
	gewisse Topologien, mit denen wir arbeiten, und deren Konvergenzverhalten in knapper Form auszudrücken. 
	Der folgende Abschnitt orientiert sich an \cite[Kapitel 2.13]{Aliprantis.2006}.
	
	\begin{Definition}[Initialtopologie]
		\label{def:initialtopologie}
		Sei $\X$ eine Menge, $A$ eine beliebige Indexmenge und für jedes $\alpha \in A$ sei $\Y_\alpha$ ein topologischer Raum sowie $\fct{f_\alpha}{\X}{\Y_\alpha}$ eine Abbildung.
		Die Initialtopologie auf $\X$ bezüglich der Abbildungen $f_\alpha, \; \alpha \in A$ ist dann die kleinste Topologie auf $\X$, bezüglich der alle Abbildungen 
		$f_\alpha, \; \alpha \in A$ stetig sind.
	\end{Definition}

	\begin{Bemerkung}[Produkttopologie]
		Sei $A$ eine beliebige Indexmenge und sei für jedes $\alpha \in A$ ein topologischer Raum $\X_\alpha$ gegeben. Die Produkttopologie der $\X_\alpha, \; \alpha \in A$
		auf dem kartesischen Produkt $\X \defby \prod_{\alpha \in A} \X_\alpha$ ist dann die Initialtopologie der kanonischen Projektionen 
		$\fctmap{\pi_\beta}{\X}{\X_\beta}{x = (x_\alpha)_{\alpha \in A}}{x_\beta}, \; \beta \in A$.
	\end{Bemerkung}
	
	\begin{Hilfssatz}[Konvergenz bezüglich der Initialtopologie]
		\label{hilfssatz:konvergenz_initialtopologie}
		In der Situation von Definition~\ref{def:initialtopologie} konvergiert ein Netz $x = (x_\iota)_{\iota \in I}$ in $\X$ genau dann gegen ein $z \in \X$, wenn für alle $\alpha \in A$
		die Konvergenz $f_\alpha(x_\iota) \to f_\alpha(z), \; \iota \to \infty$ gilt.
	\end{Hilfssatz}

	\begin{proof}
		Wegen Satz~\ref{satz:netz_konvergenz} (b) folgt die Hinrichtung direkt aus der Stetigkeit der Abbildungen $f_\alpha, \; \alpha \in A$.
		
		Für die Rückrichtung möchten wir zeigen, dass für alle Umgebungen $V$ von $z$ ein derartiges $\iota_0 \in I$ existiert, dass $x_\iota \in V$ für $\iota_0 \preceq \iota$ gilt.
		Man bemerke zunächst, dass die Menge alle Urbilder von offenen Mengen in $\Y_\alpha$ unter $f_\alpha$ für alle $\alpha \in A$ eine Subbasis der Topologie von $\X$ ist.
		Daher genügt es, Mengen der Form
		\[ \tilde{V} \; \defby \; \bigcap_{i=1}^{n} f_{\alpha_i}^{-1}(U_i) \; \subseteq \; \X \]
		zu betrachten, wobei $\alpha_1, \dots, \alpha_n \in A$ sind und $U_i$ offene Umgebungen von $f_{\alpha_i}(z)$, $i \in \set{1, \dots, n}$.
		Sei nun ein solches $\tilde{V}$ gegeben. Dann können wir für $i \in \set{1, \dots, n}$ jeweils ein $\iota_0^{(i)} \in I$ finden, sodass $x_\iota \in f_{\alpha_i}^{-1}(U_i)$ für 
		$\iota_0^{(i)} \preceq \iota$ gilt. Sukzessive Anwendung von Eigenschaft (c) in Definition~\ref{def:gerichtete_menge} liefert ein $\iota_0 \in I$ mit 
		$x_\iota \in \tilde{V}$ für $\iota_0 \preceq \iota$, was schließlich die Konvergenz $x_\iota \to z, \; \iota \to \infty$ bedeutet.
	\end{proof}

	\begin{Bemerkung}
		Man kann sich die Frage stellen, warum wir bei der Definition eines Netzes $(x_\iota)_{\iota \in I}$ fordern, dass die Indexmenge $I$ \emph{gerichtet} ist und etwa nicht lediglich
		partiell geordnet. Am Ende des vorigen Beweises lässt sich allerdings der Mehrwert, den wir durch Eigenschaft (c) in der Definition einer gerichteten Menge erhalten, sehr gut erkennen: 
		Wir können ein \enquote*{Maximum} 
		von endlich vielen $\iota \in I$ finden und damit ähnlich argumentieren, als würden wir gerade mit Folgen arbeiten.
	\end{Bemerkung}
	
	\subsection{Borel-Maße auf topologischen Räumen}
	\label{subsec:borel-maße_auf_topologischen_räumen}
	
	Nun wenden wir uns (endlichen) Borel-Maßen auf topologischen bzw. metrischen Räumen zu. Da diese den primären Betrachtungsgegenstand der
	Arbeit darstellen, möchten wir zunächst einige Begriffe definieren, Schreibweisen festlegen, und -- zumindest im Falle von metrischen Räumen -- eine im Folgenden sehr 
	hilfreiche Eigenschaft dieser Maße, \emph{Schwache Regularität} bzw. \emph{Regularität}, vorstellen.
	Der Teilabschnitt folgt \cite[Kapitel 4.14]{Simon.2015}.
	
	\begin{Definition}[Borel-$\sigma$-Algebra]
		\label{def:borel-sigma-algebra}
		Sei $\X$ ein topologischer Raum mit Topologie $\mathcal{O}$. Dann definieren wir die
		\emph{Borel-$\sigma$-Algebra} 
		$$\mathcal{B}(\X) = \mathcal{B} \defby \sigma(\mathcal{O})$$
		als die kleinste $\sigma$-Algebra auf $\X$, die $\mathcal{O}$ umfasst.
	\end{Definition}

	\begin{Definition}[Borel-Maße]
		In der Situation von Definition~\ref{def:borel-sigma-algebra} nennen wir Maße auf $\mathcal{B}(\X)$ \emph{Borel-Maße}. Wir führen außerdem die Schreibweisen
		\begin{align*}
			\Finitemeasures{\X}  \; &\defby \; \setcomp{\fct{\mu}{\mathcal{B}(\X)}{[0,\infty)}}{\mu \; 
				\text{ist endliches Maß}} \\
			\Probmeasures{\X} \; &\defby \; \setcomp{\fct{\mu}{\mathcal{B}(\X)}{[0,1]}}{\mu \; 
				\text{ist Wahrscheinlichkeitsmaß}}
		\end{align*}
		für die Menge aller \emph{endlichen Borel-Maße} beziehungsweise \emph{Borel-Wahrscheinlichkeitsmaße} auf $\X$ ein.
	\end{Definition}
	
	\begin{Definition}[Stetige beschränkte Funktionen]
		Sei $\X$ ein topologischer Raum. Dann schreiben wir
		\[ \Bdcontfct{\X} \; \defby \; \setcomp{\fct{f}{\X}{\R}}{f \; \text{ist stetig und beschränkt}} \]
		für die Menge aller stetigen und beschränkten Funktionen auf $\X$. 
		Ist $\X$ ein metrischer Raum, so schreiben wir außerdem
		\begin{align*}
			\Bduniffct{\X} \; &\defby \; \setcomp{\fct{f}{\X}{\R}}{f \; \text{ist gleichmäßig stetig und beschränkt}} \quad \text{und} \\
			\Bdlipschitzfct{\X} \; &\defby \; \setcomp{\fct{f}{\X}{\R}}{f \; \text{ist lipschitzstetig und beschränkt}} \text{.}
		\end{align*}
	\end{Definition}
	
	\begin{Bemerkung}
		Sei $\X$ ein metrischer Raum. Sicherlich ist dann $\Bdlipschitzfct{\X} \subseteq \Bduniffct{\X} \subseteq \Bdcontfct{\X}$, allerdings hängen $\Bdlipschitzfct{\X}$ und $\Bduniffct{\X}$ im Gegensatz 
		zu $\Bdcontfct{\X}$ von der Metrik selbst und nicht ausschließlich von der induzierten Topologie ab.
	\end{Bemerkung}
	
	Wie angekündigt werden wir uns im Folgenden mit der \emph{(schwachen) Regularität} von endlichen Borel-Maßen auseinandersetzen. 
	Hierbei handelt es sich um eine Approximationseigenschaft, die es uns etwa erlaubt, gewisse Aussagen 
	zunächst für abgeschlossene bzw. kompakte und für offene Mengen (die gewissermaßen \enquote*{strukturierter} als lediglich messbare Mengen sind) zu zeigen, 
	um diese anschließend auf ganz $\mathcal{B}(\X)$ auszuweiten. 
	
	\begin{Definition}[Schwache Regularität von Maßen]
		\label{def:schwache_regularität}
		Sei $\X$ ein topologischer Raum und $\mu$ ein endliches Borel-Maß auf $\X$. 
		Dann nennen wir $B \in \mathcal{B}(\X)$ \emph{schwach von 
			innen bzw. von außen regulär}, falls
		$$\mu(B) = \sup_{\substack{C \subseteq B \\ C \; \text{abgeschlossen}}} \mu(C) 
		\quad \text{bzw.} \quad \mu(B) = \inf_{\substack{U \supseteq B \\ U \; \text{offen}}} 
		\mu(U)$$
		gelten. Wir nennen $B \in \mathcal{B}(\X)$ \emph{schwach regulär}, wenn $B$ schwach von innen und 
		von außen regulär ist. Sind alle $B \in \mathcal{B}(\X)$ schwach regulär, 
		so nennen wir $\mu$ ein \emph{schwach reguläres Maß}.
	\end{Definition}
	
	\begin{Bemerkung}
		Ersetzen wir in der obigen Definition \enquote{abgeschlossen} durch \enquote{kompakt}, 
		so erhalten wir analog den Begriff der \emph{Regularität}. 
		Außerdem ist eine abgeschlossene (bzw. offene) Menge trivialerweise schwach regulär von innen (bzw. außen).
	\end{Bemerkung}
	
	Im Falle von endlichen Maßen auf metrischen Räumen kann schwache Regularität 
	recht leicht gezeigt werden, wie wir im Folgenden sehen werden.
	
	\begin{Satz}
		\label{satz:schwache_regularität}
		Ist $(\X, d)$ ein metrischer Raum, so ist jedes endliche Borel-Maß $\mu$ auf $\X$ schwach regulär.
	\end{Satz}
	
	Für den Beweis des Satzes benötigen wir noch zwei Hilfssätze. 
	Der erste Hilfssatz behandelt die recht elementare Tatsache, dass sich in metrischen Räumen 
	abgeschlossene Mengen von außen gewissermaßen \enquote*{beliebig gut} durch offene Mengen approximieren lassen. 
	Diese Erkenntnis werden wir im weiteren Verlauf der Arbeit noch häufiger benötigen.
	Der zweite Hilfssatz, den wir vorstellen,
	wird uns ermöglichen, die schwache Regularität nur auf einem Erzeuger von $\mathcal{B}(\X)$ 
	zu zeigen. Im endgültigen Beweis des Satzes werden wir dann die abgeschlossenen Mengen als Erzeuger wählen.
	
	\begin{Hilfssatz}
		\label{hilfssatz:offene_mengen}
		Sei $(\X, d)$ ein metrischer Raum und $C \subseteq \X$ eine abgeschlossene 
		Teilmenge. Ferner definieren wir für $n \in \N$
		$$ A_n \defby \setcomp{y \in \X}{d(y, C) < \frac{1}{n}} \quad \text{und} \quad 
		\fctmap{f_n}{\X}{\R}{x}{\max (0, 1-n d(x, C))} \text{,}$$
		wobei wir $d(y, C) \defby \inf_{x \in C} d(y, x)$ setzen.
		Dann gilt:
		\begin{enumeratethm}
			\item $A_n$ ist offen für alle $n \in \N$.
			\item $C = \bigcap_{n \in \N} A_n$, insbesondere ist $C$ also eine $G_\delta$-Menge.
			\item Für alle $n$ ist $\restr{f_n}{A_n^\mathsf{c}} = 0$ und $f_n$ ist lipschitzstetig.
			\item $f_n \convdown \indfct_C$.
		\end{enumeratethm}
	\end{Hilfssatz}
	
	\begin{proof}
		Aussage (a) folgt aus der Stetigkeit von $y \mapsto d(y, C)$.
		
		Weiter ist $C \subseteq A_n$ für alle $n \in \N$ und damit 
		$C \subseteq \bigcap_{n \in \N} A_n$. 
		Umgekehrt gibt es für ein beliebiges $y \in \bigcap_{n \in \N} A_n$
		eine Folge $(x_n)_n \in C^\N$ mit $x_n \rightarrow y$. 
		Wegen der Abgeschlossenheit von $C$ liegt $y$ damit in $C$, sodass (b) gezeigt ist.
		
		Aussage (c) ist klar ($f_n$ ist als Komposition 
		lipschitzstetiger Funktionen selbst lipschitzstetig).
		
		Schließlich fällt $f_n$ und für $x \in C$ gilt $f_n(x) = 1$. 
		Für $x \in C^\mathsf{c}$ ist $d(x, C) > 0$ und damit
		$$f_n(x) = \max (0, 1-n d(x, C))
		\to 0 \text{,} \quad n \to \infty \text{,}$$
		womit auch Aussage (d) folgt.
	\end{proof}
	
	\begin{Hilfssatz}
		\label{hilfssatz:schwach_reguläre_mengen_sigma_algebra}
		In der Situation von Definition~\ref{def:schwache_regularität} ist
		$$\mathcal{S} \defby \setcomp{B \in \mathcal{B}(\X)}{B \text{ schwach regulär bzgl. } \mu}$$
		eine $\sigma$-Algebra.
	\end{Hilfssatz}
	
	\begin{proof}
		Wir stellen eine Anpassung des Beweises von 
		\cite[Hilfssatz 4.5.5]{Simon.2015} vor, wo die entsprechende Aussage mit \enquote{regulär} 
		anstelle \enquote{schwach regulär} für kompakte $\X$ bewiesen wird.
		
		Offensichtlich liegen $\emptyset$ und $\X$ in $\mathcal{S}$. Sei 
		$B \in \mathcal{S}$ und damit schwach regulär 
		von innen und von außen. Wegen $\mu(\X) < \infty$ gilt dann
		$$\mu(B^\mathsf{c}) 
		\, = \, \mu(\X) - \mu(B) 
		\, = \, \mu(\X) - \sup_{\substack{C \subseteq B \\ C^\mathsf{c} \in \mathcal{O}}} \mu(C) 
		\, = \, \inf_{\substack{C \subseteq B \\ C^\mathsf{c} \in \mathcal{O}}} (\mu(\X) - \mu(C))
		\, = \, \inf_{\substack{U \supseteq B^\mathsf{c} \\ U \in \mathcal{O}}} \mu(U)$$
		sowie analog
		$$\mu(B^\mathsf{c}) 
		\, = \, \mu(\X) - \mu(B) 
		\, = \, \mu(\X) - \inf_{\substack{U \supseteq B \\ U \in \mathcal{O}}} \mu(C)
		\, = \, \sup_{\substack{U \supseteq B \\ U \in \mathcal{O}}} (\mu(\X) - \mu(U))
		\, = \, \sup_{\substack{C \subseteq B^\mathsf{c} \\ C^\mathsf{c} \in 
				\mathcal{O}}} \mu(U) \text{.}$$
		Also ist auch $B^\mathsf{c}$ schwach regulär. 
		
		Es bleibt nun zu zeigen, dass für $(B_n)_n \in \mathcal{S}^\N$ auch $B \defby 
		\bigcup_{n \in \N} B_n$ schwach regulär ist. 
		Hierfür beweisen wir zunächst die schwache Regularität von innen. 
		Sei dazu $\varepsilon > 0$. Für $n \in \N$ gibt es jeweils abgeschlossene Mengen 
		$C_n \subseteq B_n$ mit $\mu(B_n) - \mu(C_n) < \frac{\varepsilon}{3^n}$.
		Wir wählen nun $N$ so groß, dass $\mu\left( B \setminus \bigcup_{n=1}^N B_n \right) 
		< \frac{\varepsilon}{2}$ ist (was wegen $\mu(B) < \infty$ immer geht). 
		Für die abgeschlossene Menge $C := \bigcup_{n=1}^N C_k$ gilt dann die Ungleichung 
		\begin{align*}
			\mu(B) - \mu(C) = \mu(B\setminus C) \; &=
			\; \mu\left( \left( B \setminus \bigcup_{n=1}^N B_n \right) \; \cup \; 
			\left( \bigcup_{n=1}^N B_n  \setminus C \right) \right) \\
			&\leq \; \mu \left( B \setminus \bigcup_{n=1}^N B_n \right) + 
			\sum_{n=1}^{N} \mu(B_n \setminus C_n) \\
			&<    \; \frac{\varepsilon}{2} + 
			\sum_{n=1}^{\infty} \frac{\varepsilon}{3^n} \; = \; \varepsilon \text{,}
		\end{align*}
		also ist $B$ schwach regulär von innen.
		
		Ferner existieren für alle $n$ offene Mengen $U_n \supseteq B_n$ 
		mit $\mu(U_n) - \mu(B_n) < \frac{\varepsilon}{2^n}$. Wir setzen 
		$U := \bigcup_{n \in \N} U_n$ und berechnen  
		$$\mu(U) - \mu(B) \; \leq \; \sum_{n=1}^\infty \mu(U_n \setminus B) \; \leq \; 
		\sum_{n=1}^\infty \mu(U_n \setminus B_n) \; < \; \varepsilon \text{.}$$
		Weil $U$ offen ist, folgt insgesamt die schwache Regularität von $\mu$.
	\end{proof}
	
	Ausgestattet mit den Hilfssätzen \ref{hilfssatz:offene_mengen} und \ref{hilfssatz:schwach_reguläre_mengen_sigma_algebra} können wir jetzt zum Beweis von Satz~\ref{satz:schwache_regularität} 
	übergehen.
	
	\begin{proof}[Beweis von Satz~\ref{satz:schwache_regularität}]
		Es ist nun zu zeigen, dass für jedes endliche Borel-Maß $\mu$ auf $\X$
		die Menge 
		$$\mathcal{S} \defby \setcomp{B \in \mathcal{B}(\X)}{B \text{ schwach regulär bzgl. } \mu}$$
		bereits ganz $\mathcal{B}(\X)$ ist. 
		Da $\mathcal{S}$ nach Hilfssatz~\ref{hilfssatz:schwach_reguläre_mengen_sigma_algebra} eine 
		$\sigma$-Algebra ist und 
		$\mathcal{B}(\X)$ von den abgeschlossenen Mengen erzeugt wird, genügt es zu zeigen, 
		dass diese in $\mathcal{S}$ enthalten sind. 
		
		Sei $C \in \mathcal{B}(\X)$. Dann ist $C$ sicherlich schwach regulär von innen. 
		Nun verwenden wir die offenen Mengen $A_n, \; n \in \N$ aus 
		Hilfssatz~\ref{hilfssatz:offene_mengen}. 
		Wegen $A_n \convdown C$ und $\mu(\X) < \infty$ folgt mit der 
		Maßstetigkeit von oben die Konvergenz $\mu(A_n) \convdown \mu(C)$,
		sodass $C$ auch schwach regulär von außen ist.
	\end{proof}

	\begin{Bemerkung}
		Handelt es sich in der Situation von Satz~\ref{satz:schwache_regularität} bei $\X$ um einen polnischen Raum, 
		so werden wir zu einem späteren Zeitpunkt (in Folgerung~\ref{folgerung:polnischer_raum_reguläre_maße}) noch sehen, dass hier tatsächlich sogar Regularität gilt.
	\end{Bemerkung}
	
	Schließlich möchten wir noch kurz eine direkte Folgerung aus Satz~\ref{satz:schwache_regularität} vorstellen, die etwas 
	schwächere hinreichende Bedingungen für die Gleichheit 
	zweier endlicher Maße auf metrischen Räumen bereitstellt. Im weiteren Verlauf 
	wird sich dies noch als nützlich erweisen.
	
	\begin{Folgerung}
		\label{folgerung:gleichheit_von_maßen}
		Sei $(\X,d)$ ein metrischer Raum und seien $\mu, \nu \in \Finitemeasures{\X}$. 
		Dann sind die folgenden Aussagen äquivalent:
		\begin{equivalentthm}
			\item $\mu = \nu$.
			\item Für alle $f \in \Bdcontfct{\X}$ ist
			$\measureint{}{f}{\mu} = \measureint{}{f}{\nu}$.
			\item Für alle $f \in \Bduniffct{\X}$ ist
			$\measureint{}{f}{\mu} = \measureint{}{f}{\nu}$.
			\item Für alle $f \in \Bdlipschitzfct{\X}$ ist
			$\measureint{}{f}{\mu} = \measureint{}{f}{\nu}$.
			\item Für alle abgeschlossenen Mengen $C \subseteq \X$ ist $\mu(C) = \nu(C)$.
		\end{equivalentthm}
	\end{Folgerung}
	
	\begin{proof}
		Die Implikationen (i) $\Rightarrow$ (ii) $\Rightarrow$ (iii) $\Rightarrow$ (iv) sind klar und (v) 
		$\Rightarrow$ (i) folgt aus Satz~\ref{satz:schwache_regularität}.
		
		(iv) $\Rightarrow$ (v): Gelte (iv) und sei $C \subseteq \X$ abgeschlossen. 
		Dann gilt für die Funktionen $f_n \in \Bduniffct{\X}$ aus Hilfssatz~\ref{hilfssatz:offene_mengen}
		$$\measureint{}{f_n}{\mu} = \measureint{}{f_n}{\nu}\text{,} \quad n \in \N \text{.}$$
		Wegen $| f_n | \leq 1$ und $f_n \convdown \indfct_C$ folgt mit dem Satz von Lebesgue 
		$$\measureint{}{f_n}{\mu} \; \to \; \mu(C) \quad \text{und} \quad \measureint{}{f_n}{\nu} 
		\; \to \; \nu(C)$$
		und damit gilt (v).
	\end{proof}
	
\end{document}