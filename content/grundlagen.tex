\documentclass[../main/main.tex]{subfiles}

\begin{document}
	
	\section{Grundlagen}
	
	\subsection{Topologische Grundlagen}
	
	\subsubsection*{Netze}
	
	Im Allgemeinen beschreiben Folgen und deren Konvergenz die Eigenschaften eines topologischen Raumes nur unzureichend:
	Beispielsweise ist Folgenstetigkeit nicht notwendigerweise äquivalent zu Stetigkeit und zwei unterschiedliche Topologien auf einer 
	gegebenen Menge können generell dieselben konvergenten Folgen haben. Um solche Probleme im Folgenden zu umgehen, führen wir das
	Konzept des Netzes aus der mengentheoretischen Topologie ein. Netze verallgemeinern Folgen und können in vielerlei Hinsicht die 
	Eigenschaften eines topologischen Raumes besser erfassen.
	
	Überwiegend orientieren wir uns in diesem Abschnitt an \cite[Kapitel 2.6]{Simon.2015}.
	
	\begin{Definition}[Gerichtete Menge]
		\label{def:directedset}
		Sei $I$ eine Menge. Eine Relation $\preceq$ auf $I$ mit
		\begin{enumeratethm}
			\item $\forall \iota \in I: \; \iota \preceq \iota$
			\item $\forall \iota, \kappa, \lambda \in I: \; \iota \preceq \kappa$ und $\kappa \preceq \lambda$ impliziert $\iota \preceq \lambda$
			\item $\forall \iota, \kappa \in I: \; \exists \lambda \in I: \; \iota \preceq \lambda$ und $\kappa \preceq \lambda$
		\end{enumeratethm}
		heißt \emph{gerichtet}. In diesem Fall nennen wir das Paar $(I, \preceq)$ auch eine \emph{gerichtete Menge}.
	\end{Definition}
	
	\begin{Definition}[Netz]
		Sei $X$ eine Menge und $(I, \preceq)$ eine gerichtete Menge. Ein \emph{Netz} in $X$ ist dann eine Abbildung $\fct{x}{I}{X}$. In diesem
		Fall schreiben wir auch $x = (x_\iota)_{\iota \in I}$.
	\end{Definition}
	
	\begin{Definition}[Konvergenz von Netzen]
		Sei $X$ ein topologischer Raum und sei $x = (x_\iota)_{\iota \in I}$ ein Netz in $X$. Wir sagen, dass $x$ gegen $z \in X$ konvergiert, falls
		es für jede Umgebung $U$ von $z$ ein derartiges $\iota_0 \in I$ gibt, dass $x_\iota \in U$ für alle $\iota_0 \preceq \iota \in I$ erfüllt ist.
		In diesem Fall schreiben wir auch $\lim_{\iota \to \infty} x_\iota = z$ oder $x_\iota \to z, \; \iota \to \infty$.
	\end{Definition}
	
	\begin{Bemerkung}
		Offenbar sind Folgen einfach Netze, bei denen $(I, \preceq) \defby (\N, \leq)$ gewählt wird. Der Konvergenzbegriff von Netzen stimmt
		dann auch mit dem von uns bekannten Konvergenzbegriff von Folgen in topologischen Räumen überein.
	\end{Bemerkung}
	
	Wie bereits angekündigt lassen sich topologische Räume im Allgemeinen viel besser über Netze charakterisieren als lediglich über Folgen. Der nächste Satz
	konkretisiert dies.
	
	\begin{Satz}
		\label{thm:netconvergence}
		Seien $X$ und $Y$ topologische Räume. Dann gelten die folgenden Aussagen:
		\begin{enumeratethm}
			\item Eine Teilmenge $C \subseteq X$ ist genau dann abgeschlossen, wenn der Grenzwert $z \in X$ eines jeden konvergenten Netzes 
			$x = (x_\iota)_{\iota \in I}$ in $X$ bereits in $C$ liegt.
			\item Eine Funktion $\fct{f}{X}{Y}$ ist genau dann stetig, wenn für jedes konvergente Netz $x = (x_\iota)_{\iota \in I}$ in $X$ 
			\[ \lim_{\iota \to \infty} f(x_\iota) \; = \; f(\lim_{\iota \to \infty} x_\iota) \]
			erfüllt ist.
			\item Zwei Topologien auf $X$ sind genau dann identisch, wenn alle Netze in $X$ bezüglich beiden Topologien dasselbe Konvergenzverhalten aufweisen.
		\end{enumeratethm}
	\end{Satz}
	
	\begin{proof}
		Siehe \cite[Satz 2.6.3]{Simon.2015}.
	\end{proof}
	
	\begin{Bemerkung}
		Auch wenn man in spezielleren Fällen (wie etwa bei metrisierbarem $X$) problemlos mit Folgen arbeiten kann, sei an dieser Stelle angemerkt, dass im Allgemeinen 
		keine der drei Aussagen gelten, wenn man \enquote{Netz} durch \enquote{Folge} ersetzt (vgl. \cite[Beispiel 2.6.1]{Simon.2015}).
	\end{Bemerkung}
	
	Für reelle Netze können wir in Analogie zu Folgen auch den Limes Superior bzw. Inferior von Netzen definieren. Hierfür orientieren wir uns an
	\cite[Kapitel 2.1 und Aufgabe 2.55]{Megginson.1998} und \cite[Kapitel 2.4]{Aliprantis.2006}.
	
	\begin{Definition}[Limes Superior und Inferior von Netzen]
		Sei $x = (x_\iota)_{\iota \in I}$ ein Netz in $\R$. Dann definieren wir 
		\[ \limsup_{\iota \to \infty} x_\iota \; \defby \; \inf_{\iota \in I} \sup_{\iota \preceq \kappa} x_\kappa \quad \text{bzw.} \quad 
		\liminf_{\iota \to \infty} x_\iota \; \defby \; \sup_{\iota \in I} \inf_{\iota \preceq \kappa} x_\kappa \]
		und nennen diese den \emph{Limes Superior} bzw. \emph{Limes Inferior} von $x$.
	\end{Definition}
	
	\begin{Bemerkung}
		Offenbar ist auch diese Definition mit den bekannten Begriffen für reelle Folgen kompatibel. Die meisten grundlegenden Eigenschaften der Konvergenz und des 
		Limes Superior bzw. Inferior von reellen Folgen lassen sich auch auf reelle Netze übertragen, 
		wie etwa die üblichen Rechenregeln für die Grenzwerte von Summen und Produkten reeller Folgen.
		
		Außerdem gelten etwa für reelle Netze $x = (x_\iota)_{\iota \in I}$ und
		$y = (y_\iota)_{\iota \in I}$ die Ungleichungen
		\begin{align*}
			\limsup_{\iota \to \infty} \, (x_\iota + y_\iota) \; &\leq \; \limsup_{\iota \to \infty} x_\iota + \limsup_{\iota \to \infty} y_\iota \quad \text{und}\\
			\liminf_{\iota \to \infty} \, (x_\iota + y_\iota) \; &\geq \; \liminf_{\iota \to \infty} x_\iota + \liminf_{\iota \to \infty} y_\iota \text{,}
		\end{align*}
		sofern die Summen jeweils definiert sind und
		mit Gleichheit, falls eines der beiden Netze konvergiert (vgl. \cite[Aufgabe 2.55 (e)]{Megginson.1998}).
		Nach \cite[Aufgabe 2.55 (i)]{Megginson.1998} ist für ein konvergentes Netz $x = (x_\iota)_{\iota \in I}$ in $\R$ ebenfalls
		\[ \lim_{\iota \to \infty} x_\iota \; = \; \limsup_{\iota \to \infty} x_\iota \; = \; \liminf_{\iota \to \infty} x_\iota \text{.} \]
	\end{Bemerkung}
	
	\subsubsection*{Initialtopologie}
	
	Wir möchten nun noch ein weiteres Konzept aus der Topologie, die sogenannte \emph{Initialtopologie} einführen.
	Auch wenn es sich hier um einen sehr elementaren Begriff handelt, wird dieser uns an späterer Stelle 
	in Kombination mit Hilfssatz~\ref{lem: convergenceinitialtopology} häufig ermöglichen, 
	gewisse Topologien und deren Konvergenzverhalten in knapper Form auszudrücken. Dieser Abschnitt folgt \cite[Kapitel 2.13]{Aliprantis.2006}.
	
	\begin{Definition}[Initialtopologie]
		\label{def:initialtopology}
		Sei $X$ eine Menge, $A$ eine beliebige Indexmenge und für jedes $\alpha \in A$ sei $Y_\alpha$ ein topologischer Raum sowie $\fct{f_\alpha}{X}{Y_\alpha}$ eine Abbildung.
		Die Initialtopologie auf $X$ bezüglich der Abbildungen $f_\alpha, \; \alpha \in A$ ist dann die kleinste Topologie auf $X$, bezüglich der alle Abbildungen 
		$f_\alpha, \; \alpha \in A$ stetig sind.
	\end{Definition}

	\begin{Bemerkung}[Produkttopologie]
		Sei $A$ eine beliebige Indexmenge und sei für jedes $\alpha \in A$ ein topologischer Raum $X_\alpha$ gegeben. Die Produkttopologie der $X_\alpha, \; \alpha \in A$
		auf dem kartesischen Produkt $X \defby \prod_{\alpha \in A} X_\alpha$ ist dann offensichtlich die Initialtopologie der kanonischen Projektionen 
		$\fctmap{\pi_\beta}{X}{X_\beta}{x = (x_\alpha)_{\alpha \in A}}{x_\beta}, \; \beta \in A$.
	\end{Bemerkung}
	
	\begin{Hilfssatz}[Konvergenz bezüglich der Initialtopologie]
		\label{lem: convergenceinitialtopology}
		In der Situation von Definition~\ref{def:initialtopology} konvergiert ein Netz $x = (x_\iota)_{\iota \in I}$ in $X$ genau dann gegen ein $z \in X$, wenn für alle $\alpha \in A$
		die Konvergenz $f_\alpha(x_\iota) \to f_\alpha(z), \; \iota \to \infty$ gilt.
	\end{Hilfssatz}

	\begin{proof}
		Wegen Satz~\ref{thm:netconvergence} (b) folgt die Hinrichtung direkt aus der Stetigkeit der Abbildungen $f_\alpha, \; \alpha \in A$.
		
		Für die Rückrichtung möchten wir zeigen, dass für alle Umgebungen $V$ von $z$ ein derartiges $\iota_0 \in I$ existiert, dass $x_\iota \in V$ für $\iota_0 \preceq \iota$ gilt.
		Man bemerke zunächst, dass die Menge alle Urbilder von offenen Mengen in $Y_\alpha$ unter $f_\alpha$ für alle $\alpha \in A$ eine Subbasis der Topologie von $X$ ist.
		Daher genügt es, Mengen der Form
		\[ \tilde{V} \; \defby \; \bigcap_{i=1}^{n} f_{\alpha_i}^{-1}(U_i) \; \subseteq \; X \]
		zu betrachten, wobei $\alpha_1, \dots, \alpha_n \in A$ sind und $U_i$ offene Umgebungen von $f_{\alpha_i}(z)$, $i \in \set{1, \dots, n}$.
		Sei nun ein solches $\tilde{V}$ gegeben. Dann können wir für $i \in \set{1, \dots, n}$ jeweils ein $\iota_0^{(i)} \in I$ finden, sodass $x_\iota \in f_{\alpha_i}^{-1}(U_i)$ für 
		$\iota_0^{(i)} \preceq \iota$ gilt. Sukzessive Anwendung von Eigenschaft (c) in Definition~\ref{def:directedset} liefert ein $\iota_0 \in I$ mit 
		$x_\iota \in \tilde{V}$ für $\iota_0 \preceq \iota$, was schließlich die Konvergenz $x_\iota \to z, \; \iota \to \infty$ bedeutet.
	\end{proof}
	
	\subsection{Polnische Räume}
	
	In diesem Abschnitt beschäftigen wir uns mit den grundlegenden Eigenschaften von polnischen Räumen, 
	einer Klasse von speziellen topologischen Räumen. 
	Wir orientieren uns hierbei an den relevanten Teilen aus \cite[Kapitel 4.14]{Simon.2015}.
	
	An erster Stelle möchten wir die Definition eines polnischen Raumes liefern.
	
	\begin{Definition}[Polnischer Raum]
		Ein \emph{polnischer Raum} ist ein separabler und vollständig metrisierbarer topologischer Raum.
	\end{Definition}
	
	Es sei hier anzumerken, dass dies tatsächlich eine rein topologische Eigenschaft ist: Wir fordern für einen polnischen Raum $X$ nur die Existenz einer Metrik,
	die die Topologie von $X$ erzeugt und bezüglich der $X$ vollständig ist, möchten uns aber die Flexibilität
	bewahren, diese Metrik nicht zu fixieren und bei Bedarf zwischen verschiedenen solchen Metriken
	zu wechseln. Es kann durchaus Metriken geben, die $X$ zwar metrisieren, jedoch nicht vollständig.
	Beispielsweise ist $\R$ ausgestattet mit der Standardtopologie offensichtlich ein polnischer Raum,
	denn $\R$ ist separabel und die euklidische Metrik metrisiert $\R$ vollständig. 
	Allerdings ist $(0, 1)$ als Teilraum von $\R$ mit der euklidischen Metrik zwar separabel, aber nicht vollständig.
	Da $(0, 1)$ homöomorph zu $\R$ ist, ist $(0, 1)$ aber dennoch ein polnischer Raum.
	
	Der folgende Hilfssatz~\ref{lem:opensets} behandelt die recht elementare Tatsache, dass sich in metrischen Räumen 
	abgeschlossene Mengen von außen gewissermaßen \enquote{beliebig gut} durch offene Mengen approximieren lassen. 
	Diese Erkenntnis werden wir im weiteren Verlauf der Arbeit noch häufiger benötigen.
	
	\begin{Hilfssatz}
		\label{lem:opensets}
		Sei $(X, d)$ ein metrischer Raum und $C \subseteq X$ eine abgeschlossene 
		Teilmenge. Ferner definieren wir für $n \in \N$
		$$ A_n \defby \setcomp{y \in X}{d(y, C) < \frac{1}{n}} \quad \text{und} \quad 
		\fctmap{f_n}{X}{\R}{x}{\max \set{0, 1-n d(x, C)}} \text{,}$$
		wobei wir $d(y, C) \defby \inf_{x \in C} d(y, x)$ setzen.
		Dann gilt:
		\begin{enumeratethm}
			\item $A_n$ ist offen für alle $n \in \N$.
			\item $C = \bigcap_{n \in \N} A_n$, insbesondere ist $C$ also eine $G_\delta$-Menge.
			\item Für alle $n$ ist $\restr{f_n}{A_n^\mathsf{c}} = 0$ und $f_n$ ist lipschitzstetig.
			\item $f_n \convdown \indfct_C$.
		\end{enumeratethm}
	\end{Hilfssatz}
	
	\begin{proof}
		Aussage (a) folgt aus der Stetigkeit von $y \mapsto d(y, C)$.
		
		Weiter ist $C \subseteq A_n$ für alle $n \in \N$ und damit 
		$C \subseteq \bigcap_{n \in \N} A_n$. 
		Umgekehrt gibt es für ein beliebiges $y \in \bigcap_{n \in \N} A_n$
		eine Folge $(x_n)_n \in C^\N$ mit $x_n \rightarrow y$. 
		Wegen der Abgeschlossenheit von $C$ liegt $y$ damit in $C$, sodass (b) gezeigt ist.
		
		Aussage (c) ist klar ($f_n$ ist als Komposition 
		lipschitzstetiger Funktionen selbst lipschitzstetig).
		
		Schließlich fällt $f_n$ und für $x \in C$ gilt $f_n(x) = 1$. 
		Für $x \in C^\mathsf{c}$ ist $d(x, C) > 0$ und damit
		$$f_n(x) = \max \set{0, 1-n d(x, C)} 
		\to 0 \text{,} \quad n \to \infty \text{,}$$
		womit die Behauptung folgt.
	\end{proof}

	Im Folgenden lernen wir Eigenschaften und Charakterisierungen polnischer Räume kennen.
	
	Der folgende Satz beantwortet uns zunächst, welche Teilmengen eines polnischen Raumes selbst wieder polnisch sind:

	\begin{Satz}[Alexandroff]
		\label{thm:gdeltasubsetsofpolishspaces}
		Sei $X$ ein polnischer Raum. Dann ist $A \subseteq X$ genau dann selbst ein polnischer Raum bezüglich der Teilraumtopologie, 
		wenn $A \subseteq X$ eine $G_\delta$-Teilmenge ist.
	\end{Satz}

	\begin{proof}
		Der Beweis der Hinrichtung folgt \cite[Satz 7]{JordanBell.2014}, während der Beweis der Rückrichtung eine Anpassung des Beweises
		von Satz 4.14.6 aus \cite[Kapitel 4.14]{Simon.2015} ist.
		
		Für die Hinrichtung sei $A \subseteq X$ eine $G_\delta$-Teilmenge und seien 
		$U_n \subseteq X, \, n \in \N$ offene Mengen mit 
		$A = \bigcap_{n \in \N} U_n$, wobei wir außerdem 
		$C_n \defby U_n^{\mathsf{c}}$ schreiben. Ferner sei 
		$d$ eine Metrik, die $X$ vollständig metrisiert. 
		Da $A$ als Teilmenge eines separablen metrischen Raumes 
		selbst separabel ist, genügt es, die vollständige Metrisierbarkeit 
		von $A$ zu beweisen.
		
		Für $x, y \in A$ definieren wir
		\[\tilde{d}(x, y) \; \defby \; d(x, y) + \sum_{n=1}^{\infty} \min \left\{
		\frac{1}{2^n}, \left| \frac{1}{d(x, C_n)} - \frac{1}{d(y, C_n)} \right|
		\right\} \text{.}\]
		Offensichtlich ist $\tilde{d}$ eine Metrik auf $A$, 
		die dieselbe Topologie wie $d$ erzeugt. 
		
		Wir möchten nun zeigen, dass $(A, \tilde{d})$ vollständig ist. 
		Sei dazu $(x_k)_k \in A^\N$ eine Cauchyfolge. Wegen $d \leq \tilde{d}$ 
		ist $(x_k)_k$ dann auch bezüglich $d$ eine Cauchyfolge und die 
		Vollständigkeit von $(X, d)$ liefert die Existenz eines Grenzwertes 
		$x \in X$. Angenommen $x \notin A$, so gibt es ein $m \in \N$ mit 
		$x \in C_m$. Damit ist aber für alle $k \in \N$
		$$\sup_{l \geq k} \tilde{d}(x_k, x_l) \; \geq \; \sup_{l \geq k} 
		\, \sum_{n=1}^{\infty} \min \left\{
		\frac{1}{2^n}, \left| \frac{1}{d(x_k, C_n)} -
		\frac{1}{d(x_l, C_n)} \right|
		\right\} \; \geq \; \frac{1}{2^m} \; > \; 0 \text{,}$$
		was der Tatsache widerspricht, dass $(x_k)_k$ eine Cauchyfolge 
		bezüglich $\tilde{d}$ ist. Also muss $x \in A$ gelten. 
		Aus der Stetigkeit von $x \mapsto d(x, C_n)$ folgt mit dem 
		Satz von Lebesgue unmittelbar die Konvergenz 
		$\tilde{d}(x_k, x) \to 0$ für $k \to \infty$,
		also ist $(A, \tilde{d})$ vollständig und $A$ damit selbst ein
		polnischer Raum.
		
		Nun beweisen wir noch die Rückrichtung. Sei dazu $A \subseteq X$ eine derartige Teilmenge, dass
		$A$ bezüglich der Teilraumtopologie selbst polnisch ist. $\tilde{d}$ metrisiere im Folgenden 
		$A$ vollständig.
		Sicherlich ist $B_{1/k}(x_m) \subseteq A$ für alle 
		$k, m \in \N$ relativ offen, also gibt es jeweils offene Mengen 
		$V_{k, m} \subseteq X$ mit
		\[ B_{1/k}(x_m) \; = \; V_{k, m} 
		\cap A \text{.} \label{eq:2.1} \tag{2.1}\]
		Wir möchten jetzt zeigen, dass
		\[ A \; = \; \overline{A} \, \cap \, 
		\bigcap_{k \in \N} \left( \bigcup_{m \in \N} V_{k, m} \right) 
		\label{eq:2.2} \tag{2.2}\]
		gilt. 
		Weil $\overline{A}$ nach Hilfssatz~\ref{lem:opensets} 
		als abgeschlossene Menge eine $G_\delta$-Menge ist, 
		folgt aus \eqref{eq:2.2} direkt die Aussage (b).
		
		Für jedes $m \in \N$ ist sicherlich $A \subseteq 
		\bigcup_{m \in \N} V_{k, m}$, weshalb \enquote{$\subseteq$} 
		in \eqref{eq:2.2} unmittelbar folgt.
		Wir nehmen nun an, dass $z$ in der rechten Seite von \eqref{eq:2.2} liegt. 
		Dann gibt es für jedes $k \in \N$ ein $m_k \in \N$ mit $z \in V_{k, m_k}$. 
		Ferner existiert wegen $z \in \overline{A}$ eine Folge $(y_k)_k \in A^\N$ 
		mit $y_k \to z$, wobei wir zusätzlich
		$y_k \in \bigcap_{j=1}^{k} V_{j, m_j}$
		fordern. Weil damit nach \eqref{eq:2.1} auch für alle $k \in \N$
		\[y_k \in \bigcap_{j=1}^{k} B_{1/k}(x_m)\]
		gilt, ist $(y_k)_k$ eine Cauchyfolge bezüglich $\tilde{d}$. Aufgrund der Vollständigkeit von 
		$(A, \tilde{d})$ gibt es also ein $y \in A$ mit $y_k \to y$, was 
		$z = y \in A$ und damit \enquote{$\supseteq$} 
		in \eqref{eq:2.2} impliziert.
	\end{proof}
	
	Mit dem \emph{Hilbertwürfel} möchten wir uns nun einen speziellen polnischen Raum ansehen, 
	der sich später in gewisser Weise als universell für alle polnischen Räume erweisen wird.
	
	\begin{Definition}[Hilbertwürfel]
		Der \emph{Hilbertwürfel} ist der topologische Raum $H = [0, 1]^\N$ 
		ausgestattet mit der Produkttopologie, also der Initialtopologie bezüglich 
		aller Projektionen $\fctmap{\pi_n}{H}{[0, 1]}{x}{x_n}, \; n \in \N$.
	\end{Definition}

	Einige grundlegende Eigenschaften des Hilbertwürfels fassen wir im folgenden Hilfssatz zusammen.
	
	\begin{Hilfssatz}
		\label{lem:hilbertcube}
		Für den Hilbertwürfel $H$ gelten folgende Aussagen:
		\begin{enumeratethm}
			\item Eine Folge $(x^{(k)})_k \in H^\N$ konvergiert genau dann gegen 
			ein $x \in H$, wenn alle Komponenten konvergieren, also, wenn
			$\lim_{n \to \infty} x_n^{(k)} = x_n$ für alle $n \in \N$ ist.
			\item Setzen wir für $x, y \in H$
			$$\rho(x, y) \defby \max_{n \in \N} \frac{|x_n - y_n|}{2^n} \text{,}$$
			so definiert $\rho$ eine Metrik, die $H$ metrisiert.
			\item $H$ ist kompakt.
		\end{enumeratethm}
	\end{Hilfssatz}
	
	\begin{proof}
		Aussage (a) folgt direkt aus Hilfssatz~\ref{lem: convergenceinitialtopology}.
		
		Für den Beweis von (b) bemerke man zunächst, dass Mengen der Form 
		\[U = \prod_{n=1}^{\infty} U_n\text{,} \quad U_n \subseteq [0, 1] \text{ offen, }
		\quad U_n = [0, 1] \text{ für fast alle } n \label{eq:2.3} \tag{2.3}\]
		eine Basis der Topologie von $H$ bilden.
		
		Offenbar wird durch $\rho$ aus (b) eine Metrik auf $H$ definiert. 
		Es genügt also zu zeigen, dass $\rho$ die Topologie von $H$ induziert. 
		Für $x \in H$ und $r > 0$ ist 
		$B_r(x) = \prod_{n=1}^{\infty} B_{2^n r}(x_n) \subseteq H$ 
		offen. Ist nun $U = \prod_{n=1}^{\infty} U_n$ wie in \eqref{eq:2.3}, 
		so lässt sich leicht einsehen, dass es für jedes $x \in U$ ein $r > 0$ 
		gibt mit $B_r(x) \subseteq U$, also ist $U$ offen bezüglich der von $\rho$ 
		erzeugten Topologie und insgesamt wird $H$ von $\rho$ metrisiert.
		
		Der aus der Topologie bekannte \emph{Satz von Tychonoff} (vgl. \cite[Satz 2.7.1]{Simon.2015}) liefert 
		unmittelbar die Kompaktheit von $H$.
	\end{proof}

	\begin{Bemerkung}
		Insbesondere handelt es sich bei $H$ um einen polnischen Raum, denn alle kompakten metrischen Räume sind insbesondere 
		separabel und vollständig und damit auch polnisch.
	\end{Bemerkung}
	
	$H$ erfüllt nun die folgende Universalitätseigenschaft.
	
	\begin{Satz}[Universalitätseigenschaft des Hilbertwürfels]
		\label{thm:characterizationpolishspaces}
		Ein topologischer Raum $X$ ist genau dann separabel und metrisierbar, wenn $X$ homöomorph zu einer Teilmenge von $H$ ist.
		
		Ferner ist $X$ genau dann polnisch, wenn $X$ homöomorph zu einer $G_\delta$-Teilmenge von $H$ ist.
	\end{Satz}
	
	Wir werden den Beweis von Satz~\ref{thm:characterizationpolishspaces} simultan für metrisierbare und 
	separable bzw. für polnische Räume führen, benötigen hierfür aber zunächst noch einen Hilfssatz.
	
	\begin{Hilfssatz}
		\label{lem:characterizationpolishspaces}
		Sei $(X, d)$ ein separabler metrischer Raum und sei 
		$\mathcal{D} \defby \setcomp{x_n}{n \in \N} \subseteq X$ eine abzählbare 
		dichte Teilmenge. Außerdem setzen wir
		\[\fctmap{\varphi}{X}{H}{x}{\left(\min \set{1, d(x, x_n)} \right)_n} \text{.} \label{eq:2.4} \tag{2.4}\]
		Dann ist $\varphi$ ein Homöomorphismus zwischen $X$ und $\varphi(X)$.
	\end{Hilfssatz}
	
	\begin{proof}
		Wir zeigen zunächst, dass $\varphi$ injektiv ist. 
		Seien dazu $y, z \in H$ mit $\varphi(y) = \varphi(z)$, also gilt
		\[\min\set{1, d(y, x_n)} = \min \set{1, d(z, x_n)} \label{eq:2.5} \tag{2.5}\]
		für alle $n \in \N$. Weil es eine Folge $(y_n)_n \in \mathcal{D}^\N$ mit 
		$y_n \to y$ gibt und diese wegen \eqref{eq:2.5} auch gegen $z$ 
		konvergiert, folgt die Gleichheit von $y$ und $z$.
		
		Die Stetigkeit von $\varphi$ folgt direkt, weil die 
		Komponentenfunktionen $x \mapsto \min\set{1, d(x, x_n)}$ jeweils stetig sind. 
		Seien nun $(y_k)_k \in X^\N$ und $y \in X$ mit $\varphi(y_k) \to \varphi(y)$, 
		also
		\[\min\set{1, d(y_k, x_n)} \; \to \; \min\set{1, d(y, x_n)}, 
		\quad k \to \infty \label{eq:2.6} \tag{2.6}\]
		für alle $n \in \N$. Wählt man nun $\varepsilon < 1$ beliebig, 
		so existiert ein $n \in \N$ mit $d(y, x_n) \leq \varepsilon$. 
		Wegen \eqref{eq:2.6} ist dann auch
		$d(y_k, x_n) \to d(y, x_n), \; k \to \infty$. Es gilt also die Ungleichung
		$$\limsup_{k \to \infty} d(y_k, y) \; \leq \; 
		\limsup_{k \to \infty} d(y_k, x_n) + d(y, x_n) \; \leq \; 2\varepsilon$$
		und mit $\varepsilon \to 0$ folgt die Stetigkeit von $\varphi^{-1}$, 
		womit wir nachgewiesen haben, dass $\varphi$ ein Homöomorphismus auf sein Bild ist. 
	\end{proof}

	\begin{proof}[Beweis von Satz~\ref{thm:characterizationpolishspaces}]
		Da wir in Hilfssatz~\ref{lem:characterizationpolishspaces} bereits 
		gesehen haben, dass jeder separable metrisierbare Raum homöomorph zu einer 
		Teilmenge des Hilbertwürfels $H$ ist und jede Teilmenge von $H$ separabel und metrisierbar ist,
		folgt der erste Teil von Satz~\ref{thm:characterizationpolishspaces}. 
		
		Sei $X$ nun ein polnischer Raum und fixiere eine Metrik $d$, die $X$ vollständig metrisiert. 
		Nach Hilfssatz~\ref{lem:characterizationpolishspaces} ist $X$ via $\fct{\varphi}{X}{H}$ aus \eqref{eq:2.6}
		homöomorph zu einer Teilmenge von $H$ und da $H$ selbst polnisch ist, muss $\varphi(X) \subseteq H$ nach 
		Satz~\ref{thm:gdeltasubsetsofpolishspaces} eine $G_\delta$-Teilmenge sein.
		Ferner folgt aus demselben Satz, dass jede $G_\delta$-Teilmenge von $H$ polnisch ist.
	\end{proof}
	
	\subsection{Borel-Maße auf topologischen Räumen}
	
	Nun wenden wir uns (endlichen) Borel-Maßen auf topologischen bzw. metrischen Räumen zu. Da diese den primären Betrachtungsgegenstand der
	Arbeit darstellen, möchten wir zunächst einige Begriffe definieren, Schreibweisen festlegen, und - zumindest im Falle von metrischen Räumen - eine im Folgenden sehr 
	hilfreiche Eigenschaft dieser Maße, die \emph{(Schwache) Regularität} vorstellen.
	
	Wieder werden wir uns an den relevanten Teilen von \cite[Kapitel 4.14]{Simon.2015} orientieren.
	
	\begin{Definition}[Borel-$\sigma$-Algebra]
		\label{def:borelsigmaalg}
		Sei $X$ ein topologischer Raum mit Topologie $\mathcal{O}$. Dann definieren wir die
		\emph{Borel-$\sigma$-Algebra} 
		$$\mathcal{B}(X) = \mathcal{B} \defby \sigma(\mathcal{O})$$
		als die kleinste $\sigma$-Algebra auf $X$, die $\mathcal{O}$ umfasst.
	\end{Definition}

	\begin{Definition}[Borel-Maße]
		In der Situation von Definition~\ref{def:borelsigmaalg} nennen wir Maße auf $\mathcal{B}(X)$ \emph{Borel-Maße}. Wir führen außerdem die Schreibweisen
		\begin{align*}
			\Finitemeasures{X}  \; &\defby \; \setcomp{\fct{\mu}{\mathcal{B}(X)}{[0,\infty)}}{\mu \; 
				\text{ist endliches Maß}} \\
			\Probmeasures{X} \; &\defby \; \setcomp{\fct{\mu}{\mathcal{B}(X)}{[0,1]}}{\mu \; 
				\text{ist Wahrscheinlichkeitsmaß}}
		\end{align*}
		für die Menge aller \emph{endlichen Borel-Maße} beziehungsweise \emph{Borel-Wahrscheinlichkeitsmaße} auf $X$ ein.
	\end{Definition}
	
	\begin{Definition}[Stetige beschränkte Funktionen]
		Sei $X$ ein topologischer Raum. Dann schreiben wir
		\[ \Bdcontfct{X} \; \defby \; \setcomp{\fct{f}{X}{\R}}{f \; \text{ist stetig und beschränkt}} \]
		für die Menge aller stetigen und beschränkten Funktionen auf $X$. 
		Ist $X$ ein metrischer Raum, so schreiben wir
		\[ \Bduniffct{X} \; \defby \; \setcomp{\fct{f}{X}{\R}}{f \; \text{ist gleichmäßig stetig und beschränkt}} \]
		für die Menge aller gleichmäßig stetigen und beschränkten Funktionen auf $X$. 
	\end{Definition}
	
	\begin{Bemerkung}
		Sei $X$ ein metrischer Raum. Sicherlich ist dann $\Bduniffct{X} \subseteq \Bdcontfct{X}$, allerdings hängt $\Bduniffct{X}$ im Gegensatz 
		zu $\Bdcontfct{X}$ von der Metrik selbst und nicht ausschließlich von der Topologie ab.
	\end{Bemerkung}
	
	Wie angekündigt werden wir uns im Folgenden mit der \emph{(schwachen) Regularität} von endlichen Borel-Maßen auseinandersetzen. 
	Hierbei handelt es sich um eine Approximationseigenschaft, die es uns etwa erlaubt, gewisse Aussagen 
	zunächst für abgeschlossene bzw. kompakte und für offene Mengen zu zeigen, um diese 
	anschließend auf ganz $\mathcal{B}(X)$ auszuweiten. 
	
	\begin{Definition}[Schwache Regularität von Maßen]
		\label{def:regularity}
		Sei $X$ ein topologischer Raum und $\mu$ ein endliches Borel-Maß auf $X$. 
		Dann nennen wir $B \in \mathcal{B}(X)$ \emph{schwach von 
			innen bzw. von außen regulär}, falls
		$$\mu(B) = \sup_{\substack{C \subseteq B \\ C \; \text{abgeschlossen}}} \mu(C) 
		\quad \text{bzw.} \quad \mu(B) = \inf_{\substack{U \supseteq B \\ U \; \text{offen}}} 
		\mu(U)$$
		gelten. Wir nennen $B \in \mathcal{B}(X)$ \emph{schwach regulär}, wenn $B$ schwach von innen und 
		von außen regulär ist. Sind alle $B \in \mathcal{B}(X)$ schwach regulär, 
		so nennen wir $\mu$ ein \emph{schwach reguläres Maß}.
	\end{Definition}
	
	\begin{Bemerkung}
		Ersetzen wir in der obigen Definition \enquote{abgeschlossen} durch \enquote{kompakt}, 
		so erhalten wir analog den Begriff der \emph{Regularität}. 
		
		Offenbar ist eine abgeschlossene (bzw. offene) Menge schwach regulär von innen (bzw. außen).
	\end{Bemerkung}
	
	Im Falle von endlichen Maßen auf metrischen Räumen kann schwache Regularität 
	recht leicht gezeigt werden, wie wir im Folgenden sehen werden.
	
	\begin{Satz}
		\label{thm:weakregularity}
		Ist $(X, d)$ ein metrischer Raum, so ist jedes endliche Borel-Maß $\mu$ auf $X$ schwach regulär.
	\end{Satz}
	
	Für den Beweis des Satzes benötigen wir noch einen Hilfssatz, welcher 
	uns ermöglichen wird, die schwache Regularität nur auf einem Erzeuger von $\mathcal{B}(X)$ 
	zu zeigen (für den wir dann die abgeschlossenen Mengen wählen).
	
	\begin{Hilfssatz}
		\label{lem:sigmaalg}
		In der Situation von Definition~\ref{def:regularity} ist
		$$\mathcal{S} \defby \setcomp{B \in \mathcal{B}(X)}{B \text{ schwach regulär bzgl. } \mu}$$
		eine $\sigma$-Algebra.
	\end{Hilfssatz}
	
	\begin{proof}
		Wir stellen eine Anpassung des Beweises von 
		\cite[Lemma 4.5.5]{Simon.2015} vor, wo die entsprechende Aussage mit \enquote{regulär} 
		anstelle \enquote{schwach regulär} für kompakte $X$ bewiesen wird.
		
		Offensichtlich liegen $\emptyset$ und $X$ in $\mathcal{S}$. Sei 
		$B \in \mathcal{S}$ und damit schwach regulär 
		von innen und von außen. Wegen $\mu(X) < \infty$ gilt dann
		$$\mu(B^\mathsf{c}) 
		\, = \, \mu(X) - \mu(B) 
		\, = \, \mu(X) - \sup_{\substack{C \subseteq B \\ C^\mathsf{c} \in \mathcal{O}}} \mu(C) 
		\, = \, \inf_{\substack{C \subseteq B \\ C^\mathsf{c} \in \mathcal{O}}} (\mu(X) - \mu(C))
		\, = \, \inf_{\substack{U \supseteq B^\mathsf{c} \\ U \in \mathcal{O}}} \mu(U)$$
		sowie analog
		$$\mu(B^\mathsf{c}) 
		\, = \, \mu(X) - \mu(B) 
		\, = \, \mu(X) - \inf_{\substack{U \supseteq B \\ U \in \mathcal{O}}} \mu(C)
		\, = \, \sup_{\substack{U \supseteq B \\ U \in \mathcal{O}}} (\mu(X) - \mu(U))
		\, = \, \sup_{\substack{C \subseteq B^\mathsf{c} \\ C^\mathsf{c} \in 
				\mathcal{O}}} \mu(U) \text{.}$$
		Also ist auch $B^\mathsf{c}$ schwach regulär. 
		
		Es bleibt nun zu zeigen, dass für $(B_n)_n \in \mathcal{S}^\N$ auch $B \defby 
		\bigcup_{n \in \N} B_n$ schwach regulär ist. 
		Hierfür beweisen wir zunächst die schwache Regularität von innen. 
		Sei dazu $\varepsilon > 0$. Für $n \in \N$ gibt es jeweils abgeschlossene Mengen 
		$C_n \subseteq B_n$ mit $\mu(B_n) - \mu(C_n) < \frac{\varepsilon}{3^n}$.
		Wir wählen nun $N$ so groß, dass $\mu\left( B \setminus \bigcup_{n=1}^N B_n \right) 
		< \frac{\varepsilon}{2}$ ist (was wegen $\mu(B) < \infty$ immer geht). 
		Für die abgeschlossene Menge $C := \bigcup_{n=1}^N C_k$ gilt dann die Ungleichung 
		\begin{align*}
			\mu(B) - \mu(C) = \mu(B\setminus C) \; &=
			\; \mu\left( \left( B \setminus \bigcup_{n=1}^N B_n \right) \; \cup \; 
			\left( \bigcup_{n=1}^N B_n  \setminus C \right) \right) \\
			&\leq \; \mu \left( B \setminus \bigcup_{n=1}^N B_n \right) + 
			\sum_{n=1}^{N} \mu(B_n \setminus C_n) \\
			&<    \; \frac{\varepsilon}{2} + 
			\sum_{n=1}^{\infty} \frac{\varepsilon}{3^n} \; = \; \varepsilon \text{,}
		\end{align*}
		also ist $B$ schwach regulär von innen.
		
		Ferner existieren für alle $n$ offene Mengen $U_n \supseteq B_n$ 
		mit $\mu(U_n) - \mu(B_n) < \frac{\varepsilon}{2^n}$. Wir setzen 
		$U := \bigcup_{n \in \N} U_n$ und berechnen  
		$$\mu(U) - \mu(B) \; \leq \; \sum_{n=1}^\infty \mu(U_n \setminus B) \; \leq \; 
		\sum_{n=1}^\infty \mu(U_n \setminus B_n) \; < \; \varepsilon \text{.}$$
		Weil $U$ offen ist, folgt insgesamt die schwache Regularität von $\mu$.
	\end{proof}
	
	Ausgestattet mit den Hilfssätzen \ref{lem:opensets} und \ref{lem:sigmaalg} kann nun, 
	wie oben bereits angedeutet wurde, Satz~\ref{thm:weakregularity} bewiesen werden.
	
	\begin{proof}[Beweis von Satz~\ref{thm:weakregularity}]
		Es ist nun zu zeigen, dass für jedes endliche Borel-Maß $\mu$ auf $X$
		die Menge 
		$$\mathcal{S} \defby \setcomp{B \in \mathcal{B}(X)}{B \text{ schwach regulär bzgl. } \mu}$$
		bereits ganz $\mathcal{B}(X)$ ist. 
		Da $\mathcal{S}$ nach Hilfssatz~\ref{lem:sigmaalg} eine 
		$\sigma$-Algebra ist und 
		$\mathcal{B}(X)$ von den abgeschlossenen Mengen erzeugt wird, genügt es zu zeigen, 
		dass diese in $\mathcal{S}$ enthalten sind. 
		
		Sei $C \in \mathcal{B}(X)$. Dann ist $C$ sicherlich schwach regulär von innen. 
		Nun verwenden wir die offenen Mengen $A_n, \; n \in \N$ aus 
		Hilfssatz~\ref{lem:opensets}. 
		Wegen $A_n \convdown C$ und $\mu(X) < \infty$ folgt mit der 
		Maßstetigkeit von oben die Konvergenz $\mu(A_n) \convdown \mu(C)$,
		sodass $C$ auch schwach regulär von außen ist.
	\end{proof}

	\begin{Bemerkung}
		Handelt es sich in der Situation von Satz~\ref{thm:weakregularity} bei $X$ um einen polnischen Raum, 
		so werden wir zu einem späteren Zeitpunkt (in Folgerung~\ref{kor:polishregular}) noch sehen, dass hier tatsächlich sogar Regularität gilt.
	\end{Bemerkung}
	
	Schließlich möchten wir noch kurz eine direkte Folgerung aus Satz~\ref{thm:weakregularity} vorstellen, die etwas 
	schwächere hinreichende Bedingungen für die Gleichheit 
	zweier endlicher Maße auf metrischen Räumen bereitstellt. Im weiteren Verlauf 
	wird sich dies noch als nützlich erweisen.
	
	\begin{Satz}
		\label{thm:measureequality}
		Sei $(X,d)$ ein metrischer Raum und seien $\mu, \nu \in \Finitemeasures{X}$. 
		Dann sind die folgenden Aussagen äquivalent:
		\begin{equivalentthm}
			\item $\mu = \nu$.
			\item Für alle $f \in \Bduniffct{X}$ ist
			$\measureint{}{f}{\mu} = \measureint{}{f}{\nu}$.
			\item Für alle abgeschlossenen Mengen $C \subseteq X$ ist $\mu(C) = \nu(C)$.
		\end{equivalentthm}
	\end{Satz}
	
	\begin{proof}
		Die Implikation (i) $\Rightarrow$ (ii) ist klar und (iii) 
		$\Rightarrow$ (i) folgt aus Satz~\ref{thm:weakregularity}.
		
		(ii) $\Rightarrow$ (iii): Gelte (ii) und sei $C \subseteq X$ abgeschlossen. 
		Dann gilt für die Funktionen $f_n \in \Bduniffct{X}$ aus Hilfssatz~\ref{lem:opensets}
		$$\measureint{}{f_n}{\mu} = \measureint{}{f_n}{\nu}\text{,} \quad n \in \N \text{.}$$
		Wegen $| f_n | \leq 1$ und $f_n \convdown \indfct_C$ folgt mit dem Satz von Lebesgue 
		$$\measureint{}{f_n}{\mu} \; \to \; \mu(C) \quad \text{und} \quad \measureint{}{f_n}{\nu} 
		\; \to \; \nu(C)$$
		und damit gilt (iii).
	\end{proof}
	
\end{document}