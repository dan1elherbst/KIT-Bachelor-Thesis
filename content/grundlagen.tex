\documentclass[../main/main.tex]{subfiles}

\begin{document}
	
	\section{Grundlagen}
	\label{Grundlagen}
	
	In diesem Abschnitt möchten wir zunächst einige maßtheoretische Grundlagen vorstellen, 
	die wir später benötigen werden.

	\begin{Definition}[Polnischer Raum]
		Ein polnischer Raum ist ein separabler topologischer Raum $(X, \mathcal{O})$, 
		dessen Topologie von einer Metrik, bezüglich der $X$ vollständig ist, erzeugt wird.
	\end{Definition}

	\begin{Definition}[Borelsche $\sigma$-Algebra]
		\label{def:borel}
		Sei $(X, \mathcal{O})$ ein topologischer Raum. Dann definieren wir die Borelsche 
		$\sigma$-Algebra
		$$\mathcal{B} \defby \sigma(\mathcal{O})$$
		über $X$. Ferner sei
		$$\Probmeasures{X} \defby \setcomp{\fct{\mu}{\mathcal{B}}{[0,1]}}{\mu \; 
			\text{ist Wahrscheinlichkeitsmaß}}\text{.}$$
	\end{Definition}

	\begin{Definition}[Schwache Regularität von Maßen]
		\label{def:regularity}
		In der Situation von Definition~\ref{def:borel}, wobei zusätzlich $\mu$ ein 
		endliches Maß auf $\mathcal{B}$ sei, nennen wir $B \in \mathcal{B}$ schwach von 
		innen bzw. von außen regulär, falls
		$$\mu(B) = \sup_{\substack{C \subseteq B \\ C^\mathsf{c} \in \mathcal{O}}} \mu(C) 
		\quad \text{bzw.} \quad \mu(B) = \inf_{\substack{U \supseteq B \\ U \in \mathcal{O}}} 
		\mu(U)\text{.}$$
		Wir nennen $B \in \mathcal{B}$ schwach regulär, wenn $B$ schwach von innen und 
		von außen regulär ist. Sind alle $B \in \mathcal{B}$ schwach regulär, 
		so nennen wir $\mu$ ein schwach reguläres Maß.
	\end{Definition}

	\begin{Bemerkung}
		Ersetzen wir in der obigen Definition \enquote{abgeschlossen} durch \enquote{kompakt}, 
		so erhalten wir analog den Begriff der Regularität. (Schwache) Regularität ist eine 
		Approximationseigenschaft von Maßen, die häufig benutzt wird, um gewisse Aussagen 
		zunächst für abgeschlossene bzw. kompakte und für offene Mengen zu zeigen, um diese 
		anschließend auf ganz $\mathcal{B}$ auszuweiten.
	\end{Bemerkung}

	Im Falle von Wahrscheinlichkeitsmaßen auf metrischen Räumen kann schwache Regularität 
	recht einfach gezeigt werden, wie wir im Folgenden sehen werden.

	\begin{Satz}
		\label{thm:weakregularity}
		Ist $(X, d)$ ein metrischer Raum, so ist jedes $\mu \in \Probmeasures{X}$ schwach regulär.
	\end{Satz}

	Für den Beweis des Satzes benötigen wir noch zwei Hilfssätze. Hilfssatz~\ref{lem:sigmaalg} wird 
	uns zunächst ermöglichen, die schwache Regularität nur auf einem Erzeuger von $\mathcal{B}$ 
	zu zeigen (für den wir dann die abgeschlossenen Mengen wählen).
	
	\begin{Hilfssatz}
		\label{lem:sigmaalg}
		In der Situation von Definition~\ref{def:regularity} ist
		$$\mathcal{S} \defby \setcomp{B \in \mathcal{B}}{B \text{ schwach regulär bzgl. } \mu}$$
		eine $\sigma$-Algebra.
	\end{Hilfssatz}
	
	\begin{proof}
		$\emptyset, X \in \mathcal{S}$ ist offensichtlich. Jedes $B \in \mathcal{S}$ ist schwach regulär 
		von innen und von außen und daher gilt wegen $\mu(X) < \infty$
		$$\mu(B^\mathsf{c}) \, = \, \mu(X) - \mu(B) \, = \, \mu(X) - \sup_{\substack{C \subseteq B \\ 
				C^\mathsf{c} \in \mathcal{O}}} \mu(C) \, = \, \inf_{\substack{U \supseteq B^\mathsf{c} \\ 
				U \in \mathcal{O}}} \mu(U)$$
		sowie analog
		$$\mu(B^\mathsf{c}) \, = \, \mu(X) - \mu(B) \, = \, \mu(X) - \inf_{\substack{U \supseteq B \\ 
				U \in \mathcal{O}}} \mu(C) \, = \, \sup_{\substack{C \subseteq B^\mathsf{c} \\ C^\mathsf{c} 
				\in \mathcal{O}}} \mu(U) \text{,}$$
		womit die schwache Regularität von $B^\mathsf{c}$ folgt. 
		
		Es bleibt also zu zeigen, dass für $(B_n)_n \in \mathcal{S}^\N$ auch $B \defby 
		\bigcup_{n \in \N} B_n$ schwach regulär ist. 
		Hierfür zeigen zunächst die schwache Regularität von innen. 
		Sei dazu $\varepsilon > 0$ und wähle für $n \in \N$ jeweils abgeschlossene Mengen 
		$C_n \subseteq B_n$ mit $\mu(B_n) - \mu(C_n) < \frac{\varepsilon}{3^n}$.
		Wenn wir nun $N$ so groß wählen, dass $\mu\left( B \setminus \bigcup_{n=1}^N B_n \right) 
		< \frac{\varepsilon}{2}$ 
		(was wegen $\mu(B) < \infty$ immer geht), so gilt mit $C := \bigcup_{n=1}^N C_k$, dass 
		\begin{align*}
			\mu(B) - \mu(C) = \mu(B\setminus C) \; &=
			    \; \mu\left( \left( B \setminus \bigcup_{n=1}^N B_n \right) \; \cup \; 
			    \left( \bigcup_{n=1}^N B_n  \setminus C \right) \right) \\
			&\leq \; \mu \left( B \setminus \bigcup_{n=1}^N B_n \right) + 
			\sum_{n=1}^{N} \mu(B_n \setminus C_n) \\
			&<    \; \frac{\varepsilon}{2} + 
			\sum_{n=1}^{\infty} \frac{\varepsilon}{3^n} \; = \; \varepsilon
		\end{align*}
		und weil $C$ abgeschlossen ist, folgt, dass $B$ schwach regulär von innen ist.
		
		Wählen wir nun für alle $n$ offene Mengen $U_n \supseteq B_n$ so, 
		dass $\mu(U_n) - \mu(B_n) < \frac{\varepsilon}{2^n}$, und setzen 
		$U := \bigcup_{n \in \N} U_n$, so gilt 
		$$\mu(U) - \mu(B) \; \leq \; \sum_{n=1}^\infty \mu(U_n \setminus B) \; \leq \; 
		\sum_{n=1}^\infty \mu(U_n \setminus B_n) \; < \; \varepsilon$$
		und weil $U$ offen ist, folgt insgesamt die schwache Regularität von $\mu$.
	\end{proof}

	Der folgende Hilfssatz~\ref{lem:opensets} liefert uns noch eine Möglichkeit, 
	abgeschlossene Mengen von außen durch offene Mengen zu approximieren, womit wir im 
	Beweis von Satz~\ref{thm:weakregularity} die äußere Regularität von abgeschlossenen 
	Mengen zeigen werden können. Insbesondere die in Hilfssatz~\ref{lem:opensets} definierten 
	Funktionen $f_n$ werden  auch im weiteren Verlauf noch nützlich sein.
	
	\begin{Hilfssatz}
		\label{lem:opensets}
		Sei $(X, d)$ ein metrischer Raum und $C \subseteq X$ eine abgeschlossene 
		Teilmenge. Ferner definieren wir für $n \in \N$
		$$ A_n \defby \setcomp{y \in X}{d(y, C) < \frac{1}{n}} \quad \text{und} \quad 
		\fctmap{f_n}{X}{\R}{x}{\max \set{0, 1-n d(x, C)}} \text{,}$$
		wobei wir $d(y, C) \defby \inf_{x \in C} d(y, x)$ setzen.
		Dann gilt:
		\begin{enumeratethm}
			\item $A_n$ ist offen für alle $n \in \N$.
			\item $C = \bigcap_{n \in \N} A_n$, insbesondere ist $C$ also eine $G_\delta$-Menge.
			\item Für alle $n$ ist $\restr{f_n}{A_n^\mathsf{c}} = 0$ und $f_n$ ist gleichmäßig stetig.
			\item $f_n \convdown \indfct_C$.
		\end{enumeratethm}
	\end{Hilfssatz}

	\begin{proof}
		zu (a): Wir wählen $y \in A_n$, also $y \in X, \; d(y, C) < \frac{1}{n}$. 
		Mit $r_n \defby \frac{1}{n} - d(y, C) > 0$
		gilt dann für alle $x \in X, \; d(x, y) < r_n$:
		$$d(x, C) \leq d(x, y) + d(y, C) < \frac{1}{n} \text{,}$$
		und damit gilt $B_{r_n}(y) \subseteq A_n$ und $A_n$ ist offen.
		
		zu (b): Sicherlich ist $C \subseteq A_n$ für alle $n \in \N$ und damit 
		$C \subseteq \bigcap_{n \in \N} A_n$. 
		Umgekehrt ist für ein beliebiges $y \in \bigcap_{n \in \N} A_n$
		$$d(y, C) = \inf_{x \in C} d(y, x) = 0 \text{,}$$
		was impliziert, dass es eine Folge $(x_n)_n \in C^\N$ gibt mit $x_n \rightarrow y$, 
		und wegen der Abgeschlossenheit von $C$ ist damit $y \in C$, also auch 
		$\bigcap_{n \in \N} A_n \subseteq C$.
		
		zu (c): Für $x \in A_n^\mathsf{c}$ ist $d(x, C) \geq \frac{1}{n}$ und 
		damit $1 - nd(x, C) \leq 0$, also $\restr{f_n}{A_n^\mathsf{c}} = 0$. 
		Außerdem gilt für beliebige $x, y \in X$:
		\begin{align*}
			| f_n(x) - f_n(y) | \; &= \; | \max \set{0, 1-n d(x, C)} - \max \set{0, 1-n d(y, C)} | \\
								   &= \; \begin{cases}
								   	n | d(y, C) - d(x, C) | & \quad d(x, C) < \frac{1}{n} 
								   							\; \text{und} \; d(y, C) < \frac{1}{n} \\
								   	1 - nd(x, C)            & \quad d(x, C) < \frac{1}{n} 
								   							\; \text{und} \; d(y, C) \geq \frac{1}{n} \\
								   	1 - nd(y, C)            & \quad d(x, C) \geq \frac{1}{n} 
								   							\; \text{und} \; d(y, C) < \frac{1}{n} \\
								   	0                       & \quad d(x, C) \geq \frac{1}{n} 
								   							\; \text{und} \; d(y, C) \geq \frac{1}{n}
								   \end{cases} \\
							      &\leq \; n | d(y, C) - d(x, C) | \; \leq \; nd(x, y)
		\end{align*}
		und damit ist $f_n$ lipschitzstetig, also auch gleichmäßig stetig.
		
		zu (d): Offensichtlich ist $(f_n)_n$ fallend. Für $x \in C^\mathsf{c}$ ist $d(x, C) > 0$ und damit
		$$f_n(x) = \max \set{0, 1-n d(x, C)} \; \xrightarrow{n \to \infty} \; 0 \text{,}$$
		für $x \in C$ ist dagegen $f_n(x) = 1$ konstant, womit die Behauptung folgt.
	\end{proof}

	Ausgestattet mit den Hilfssätzen \ref{lem:sigmaalg} und \ref{lem:opensets} kann nun, 
	wie oben bereits skizziert, Satz~\ref{thm:weakregularity} bewiesen werden.

	\begin{proof}[Beweis von Satz~\ref{thm:weakregularity}]
		Es ist nun zu zeigen, dass für jedes $\mu \in \Probmeasures{X}$
		die Menge 
		$$\mathcal{S} \defby \setcomp{B \in \mathcal{B}}{B \text{ schwach regulär bzgl. } \mu}$$
		bereits ganz $\mathcal{B}$ ist. 
		Da nach Hilfssatz~\ref{lem:sigmaalg} $\mathcal{S}$ eine $\sigma$-Algebra ist und 
		$\mathcal{B}$ von den abgeschlossenen Mengen erzeugt wird, genügt es zu zeigen, 
		dass diese in $\mathcal{S}$ enthalten sind. 
		
		Jedes abgeschlossene $C \in \mathcal{B}$ ist sicherlich schwach regulär von innen. 
		Nun wählen wir $A_n, \; n \in \N$ wie in Hilfssatz~\ref{lem:opensets}. 
		Wegen $A_n \convdown C$ und $\mu(X) < \infty$ folgt mit der Maßstetigkeit von oben, dass
		$$\mu(A_n) \convdown \mu(C)$$
		und da alle $A_n$ offen sind, ist $C$ auch schwach regulär von außen.
		
		Insgesamt folgt also $\mathcal{S} = \mathcal{B}$ und damit die Behauptung.
	\end{proof}

	\begin{Satz}
		Sei $(X,d)$ ein metrischer Raum und seien $\mu, \nu \in \Probmeasures{X}$. Dann sind äquivalent:
		\begin{equivalentthm}
			\item $\mu = \nu$
			\item Für alle gleichmäßig stetigen Funktionen $\fct{f}{X}{\R}$ ist
					 $\measureint{}{f}{\mu} = \measureint{}{f}{\nu}$
			\item Für alle abgeschlossenen Mengen $C \in \mathcal{B}$ ist $\mu(C) = \nu(C)$.
		\end{equivalentthm}
	\end{Satz}

	\begin{proof}
		(i) $\Rightarrow$ (ii) ist klar.
		
		(ii) $\Rightarrow$ (iii): Gelte (ii) und sei $C \subseteq X$ abgeschlossen. 
		Dann gilt für die gleichmäßig stetigen Funktionen $f_n$ aus Hilfssatz~\ref{lem:opensets}
		$$\measureint{}{f_n}{\mu} = \measureint{}{f_n}{\nu}\text{,} \quad n \in \N \text{.}$$
		Wegen $| f_n | \leq 1$ und $f_n \convdown \indfct_C$ folgt mit dem Satz von Lebesgue 
		$$\measureint{}{f_n}{\mu} \; \to \; \mu(C) \quad \text{und} \quad \measureint{}{f_n}{\nu} 
			\; \to \; \nu(C)$$
		und damit gilt (iii).
		
		(iii) $\Rightarrow$ (i): Da $\mu$ und $\nu$ nach Satz~\ref{thm:weakregularity} 
		schwach regulär sind, folgt die Behauptung ebenfalls direkt.
	\end{proof}

	\begin{Definition}[Schwache Konvergenz von Maßen]
		Ist $(\mu_n)_n \in \Probmeasures{X}$ eine Folge, so sagen wir, dass $(\mu_n)_n$ 
		schwach gegen $\mu \in \Probmeasures{X}$ konvergiert, falls für alle $f \in C(X)$, 
		den stetigen beschränkten Funktionen, gilt, dass
		$$\measureint{}{f}{\mu_n} \; \xrightarrow{n \to \infty} \; \measureint{}{f}{\mu} \text{.}$$
		In diesem Fall schreiben wir
		$$\mu_n \xrightarrow{w} \mu \text{.}$$
	\end{Definition}

	Eine Charakterisierung der schwachen Konvergenz von Maßen liefert der folgende Satz:

	\begin{Satz}[Portmanteau]
		Sei $(X, d)$ ein metrischer Raum, so sind für $(\mu_n)_n \in \Probmeasures{X}^\N$ 
		und $\mu \in \Probmeasures{X}$ äquivalent:
		\begin{equivalentthm}
			\item $\mu_n \xrightarrow{w} \mu$
			\item Für alle abgeschlossenen Mengen $C \subseteq X$ gilt: 
				$$\limsup_{n \to \infty} \mu_n(C) \; \leq \; \mu(C)$$
			\item Für alle offenen Mengen $U \subseteq X$ gilt: 
				$$\liminf_{n \to \infty} \mu_n(U) \; \geq \; \mu(U)$$
			\item Für alle $B \in \mathcal{B}$ mit $\mu(\partial B) = 0$ 
				ist $$\lim_{n \to \infty} \mu_n(A) \; = \; \mu(A) \text{.}$$
		\end{equivalentthm}
	\end{Satz}

	\begin{proof}
		(i) $\Rightarrow$ (ii): Sei $C \subseteq X$ abgeschlossen und definiere 
		die zugehörigen $f_m, \; m \in \N$ aus Hilfssatz~\ref{lem:opensets}, 
		welche sowohl stetig als auch beschränkt sind.
		Dann gilt für alle $m \in \N$
		$$\mu_n(C) \; = \; \measureint{}{\indfct_C}{\mu_n} \; \leq \; 
			\measureint{}{f_m}{\mu_n} \; \xrightarrow{n \to \infty} 
			\measureint{}{f_m}{\mu} \text{,}$$
		also 
		$$\limsup_{n \to \infty} \mu_n(C) \; \leq \; 
			\measureint{}{f_m}{\mu} \quad \forall m \in \N$$
		und mit $f_m \convdown \indfct_C$ sowie $| f_m | \leq 1$ 
		liefert der Satz von Lebesgue, dass
		$$\measureint{}{f_m}{\mu} \; \xrightarrow{m \to \infty} 
			\measureint{}{\indfct_C}{\mu} \; = \; \mu(C) \text{,}$$
		woraus
		$$\limsup_{n \to \infty} \mu_n(C) \; \leq \; \mu(C)$$
		folgt.
		
		(ii) $\Leftrightarrow$ (iii): Sei $U$ offen, also 
		$C \defby U^\mathsf{c}$ abgeschlossen. Dann gilt
		$$\mu(X) - \liminf_{n \to \infty} \mu_n(U) \; = \; 
			\limsup_{n \to \infty} \mu_n(C) \; \leq \; 
			\mu(C) \; = \; \mu(X) - \mu(U)$$
		und damit 
		$$\liminf_{n \to \infty} \mu_n(U) \; \geq \; \mu(U) \text{.}$$
		Die andere Richtung geht vollkommen analog.
		
		(ii), (iii) $\Rightarrow$ (iv): Sei $A \in \mathcal{B}$ mit 
		$\mu(\partial A) = 0$. Dann gilt zunächst
		$$A^\mathsf{o} \subseteq A \subseteq \overline{A}, \quad 
			\partial A = \overline{A} \setminus A^\mathsf{o} \quad 
			\text{und damit } \quad \mu(A^\mathsf{o}) = 
			\mu(A) = \mu(\overline{A}) \text{,}$$
		und wegen 
		$$\limsup_{n \to \infty} \mu_n(\overline{A}) \; \leq \; 
			\mu(\overline{A}) \quad \text{und} \quad 
			\liminf_{n \to \infty} \mu_n(A^\mathsf{o}) \; \geq \; 
			\mu(A^\mathsf{o})$$
		folgt
		$$\lim_{n \to \infty} \mu_n(A) \; = \; \mu(A) \text{.}$$
		
		(iv) $\Rightarrow$ (i): Sei $f \in C(X)$ und $a < b \in \R$ 
		mit $a < f < b$. Setzen wir
		$$S \defby \setcomp{c \in (a, b)}{\mu(\set{f = c}) > 0} \text{,}$$
		so ist S abzählbar (denn für $n \in \N$ ist 
		$S_n \defby \setcomp{c \in (a, b)}{\mu(\set{f = c}) > \frac{1}{n}}$ 
		endlich und $S = \bigcup_{n \in \N} S_n$).
		Damit können wir für jedes $m \in \N$ 
		\[a = \underbrace{c_0^{(m)} < \dots < c_{2m}^{(m)}}_{\notin S} = 
			b \text{,} \qquad c_{j+1}^{(m)} - c_j^{(m)} \leq \frac{b-a}{m} 
			\label{eq:2.9.1} \tag{2.9.1}\]
		finden, wobei wir
		$$A_j^{(m)} \defby \set{c_j^{(m)} < f \leq c_{j+1}^{(m)}}\text{,}
			 \qquad j = 0,\dots,2m-1$$
		setzen. 
		Wegen der Stetigkeit von $f$ gilt dann 
		$\partial A_j^{(m)} \subseteq \set{f = c_j^{(m)}} \cup \set{f = c_{j+1}^{(m)}}$, 
		also 
		$$\mu(A_j^{(m)}) = 0 \text{,} \qquad j = 0,\dots,2m-1 \text{.}$$
		Setzen wir nun
		$$u_m \defby \sum_{j=0}^{2m-1} c_j^{(m)} \indfct_{A_j^{(m)}} 
			\text{,} \qquad m \in \N \text{,}$$
		so ist nach (iv)
		\[\measureint{}{u_m}{\mu_n} \; = \; \sum_{j=0}^{2m-1} c_j^{(m)} \mu_n(A_j^{(m)}) 
			\; \xrightarrow{n \to \infty} \; \sum_{j=0}^{2m-1} c_j^{(m)} \mu(A_j^{(m)}) \; = \; 
			\measureint{}{u_m}{\mu} \label{eq:2.9.2} \tag{2.9.2}\]
		und wegen \eqref{eq:2.9.1} gilt
		\[\left| \measureint{}{f}{\mu_n} - \measureint{}{u_m}{\mu_n} \right| \; \leq \; 
			\frac{b-a}{m} \quad \forall n \in \N \text{,} \qquad 
			\left| \measureint{}{f}{\mu} - \measureint{}{u_m}{\mu} \right| \; \leq \; 
			\frac{b-a}{m} \text{.} \label{eq:2.9.3} \tag{2.9.3}\]
		Damit gilt für alle $m, n \in \N$
		$$ \left| \measureint{}{f}{\mu} - \measureint{}{f}{\mu_n} \right| \; \leq \; 
			\left| \measureint{}{f}{\mu} - \measureint{}{u_m}{\mu} \right| + 
			\left| \measureint{}{u_m}{\mu} - \measureint{}{u_m}{\mu_n} \right| + 
			\left| \measureint{}{u_m}{\mu_n} - \measureint{}{f}{\mu_n} \right| \text{,}$$
		also wegen \eqref{eq:2.9.2} und \eqref{eq:2.9.3} für alle $m$
		$$ \limsup_{n \to \infty} \left| \measureint{}{f}{\mu} - \measureint{}{f}{\mu_n} \right|
			 \; \leq \; 2 \cdot \frac{b-a}{m} \quad 
			 \left(\xrightarrow{m \to \infty} \; 0\right) \text{,}$$
		was die Behauptung impliziert.
	\end{proof}

	
	
\end{document}

