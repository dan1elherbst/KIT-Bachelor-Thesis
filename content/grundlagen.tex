\documentclass[../main/main.tex]{subfiles}

\begin{document}
	
	\section{Grundlagen}
	\label{Grundlagen}
	
	In diesem Abschnitt möchten wir zunächst einige maßtheoretische Grundlagen vorstellen, 
	die wir später benötigen werden.

	\begin{Definition}[Polnischer Raum]
		Ein \emph{polnischer Raum} ist ein separabler und vollständig metrisierbarer topologischer Raum.
	\end{Definition}

	\begin{Definition}[Borelsche $\sigma$-Algebra]
		\label{def:borel}
		Sei $(X, \mathcal{O})$ ein topologischer Raum. Dann definieren wir die \emph{Borelsche 
		$\sigma$-Algebra}
		$$\mathcal{B} \defby \sigma(\mathcal{O})$$
		über $X$. Ferner sei
		$$\Probmeasures{X} \defby \setcomp{\fct{\mu}{\mathcal{B}}{[0,1]}}{\mu \; 
			\text{ist Wahrscheinlichkeitsmaß}}\text{.}$$
	\end{Definition}

	\begin{Definition}[Schwache Regularität von Maßen]
		\label{def:regularity}
		In der Situation von Definition~\ref{def:borel}, wobei zusätzlich $\mu$ ein 
		endliches Maß auf $\mathcal{B}$ sei, nennen wir $B \in \mathcal{B}$ \emph{schwach von 
		innen bzw. von außen regulär}, falls
		$$\mu(B) = \sup_{\substack{C \subseteq B \\ C^\mathsf{c} \in \mathcal{O}}} \mu(C) 
		\quad \text{bzw.} \quad \mu(B) = \inf_{\substack{U \supseteq B \\ U \in \mathcal{O}}} 
		\mu(U)$$
		gelten. Wir nennen $B \in \mathcal{B}$ \emph{schwach regulär}, wenn $B$ schwach von innen und 
		von außen regulär ist. Sind alle $B \in \mathcal{B}$ schwach regulär, 
		so nennen wir $\mu$ ein \emph{schwach reguläres Maß}.
	\end{Definition}

	\begin{Bemerkung}
		Ersetzen wir in der obigen Definition \enquote{abgeschlossen} durch \enquote{kompakt}, 
		so erhalten wir analog den Begriff der \emph{Regularität}. (Schwache) Regularität ist eine 
		Approximationseigenschaft von Maßen, die es uns etwa erlaubt, gewisse Aussagen 
		zunächst für abgeschlossene bzw. kompakte und für offene Mengen zu zeigen, um diese 
		anschließend auf ganz $\mathcal{B}$ auszuweiten.
	\end{Bemerkung}

	Offenbar ist eine abgeschlossene (bzw. offene) Menge schwach regulär von innen (bzw. außen).

	Im Falle von Wahrscheinlichkeitsmaßen auf metrischen Räumen kann schwache Regularität 
	recht einfach gezeigt werden, wie wir im Folgenden sehen werden.

	\begin{Satz}
		\label{thm:weakregularity}
		Ist $(X, d)$ ein metrischer Raum, so ist jedes $\mu \in \Probmeasures{X}$ schwach regulär.
	\end{Satz}

	Für den Beweis des Satzes benötigen wir noch zwei Hilfssätze. Hilfssatz~\ref{lem:sigmaalg} wird 
	uns zunächst ermöglichen, die schwache Regularität nur auf einem Erzeuger von $\mathcal{B}$ 
	zu zeigen (für den wir dann die abgeschlossenen Mengen wählen).
	
	\begin{Hilfssatz}
		\label{lem:sigmaalg}
		In der Situation von Definition~\ref{def:regularity} ist
		$$\mathcal{S} \defby \setcomp{B \in \mathcal{B}}{B \text{ schwach regulär bzgl. } \mu}$$
		eine $\sigma$-Algebra.
	\end{Hilfssatz}
	
	\begin{proof}
		Offensichtlich liegen $\emptyset$ und $X$ in $\mathcal{S}$. Sei $B \in \mathcal{S}$ und damit schwach regulär 
		von innen und von außen. Wegen $\mu(X) < \infty$ gilt dann
		$$\mu(B^\mathsf{c}) 
		\, = \, \mu(X) - \mu(B) 
		\, = \, \mu(X) - \sup_{\substack{C \subseteq B \\ C^\mathsf{c} \in \mathcal{O}}} \mu(C) 
		\, = \, \inf_{\substack{C \subseteq B \\ C^\mathsf{c} \in \mathcal{O}}} (\mu(X) - \mu(C))
		\, = \, \inf_{\substack{U \supseteq B^\mathsf{c} \\ U \in \mathcal{O}}} \mu(U)$$
		sowie analog
		$$\mu(B^\mathsf{c}) 
		\, = \, \mu(X) - \mu(B) 
		\, = \, \mu(X) - \inf_{\substack{U \supseteq B \\ U \in \mathcal{O}}} \mu(C)
		\, = \, \sup_{\substack{U \supseteq B \\ U \in \mathcal{O}}} (\mu(X) - \mu(U))
		\, = \, \sup_{\substack{C \subseteq B^\mathsf{c} \\ C^\mathsf{c} \in \mathcal{O}}} \mu(U) \text{.}$$
		Also ist auch $B^\mathsf{c}$ schwach regulär. 
		
		Es bleibt nun zu zeigen, dass für $(B_n)_n \in \mathcal{S}^\N$ auch $B \defby 
		\bigcup_{n \in \N} B_n$ schwach regulär ist. 
		Hierfür beweisen wir zunächst die schwache Regularität von innen. 
		Sei dazu $\varepsilon > 0$. Für $n \in \N$ gibt es jeweils abgeschlossene Mengen 
		$C_n \subseteq B_n$ mit $\mu(B_n) - \mu(C_n) < \frac{\varepsilon}{3^n}$.
		Wir wählen nun $N$ so groß, dass $\mu\left( B \setminus \bigcup_{n=1}^N B_n \right) 
		< \frac{\varepsilon}{2}$ ist (was wegen $\mu(B) < \infty$ immer geht). Für die abgeschlossene Menge $C := \bigcup_{n=1}^N C_k$ gilt dann die Ungleichung 
		\begin{align*}
			\mu(B) - \mu(C) = \mu(B\setminus C) \; &=
			    \; \mu\left( \left( B \setminus \bigcup_{n=1}^N B_n \right) \; \cup \; 
			    \left( \bigcup_{n=1}^N B_n  \setminus C \right) \right) \\
			&\leq \; \mu \left( B \setminus \bigcup_{n=1}^N B_n \right) + 
			\sum_{n=1}^{N} \mu(B_n \setminus C_n) \\
			&<    \; \frac{\varepsilon}{2} + 
			\sum_{n=1}^{\infty} \frac{\varepsilon}{3^n} \; = \; \varepsilon \text{,}
		\end{align*}
		also ist $B$ schwach regulär von innen.
		
		Ferner existieren für alle $n$ offene Mengen $U_n \supseteq B_n$ mit $\mu(U_n) - \mu(B_n) < \frac{\varepsilon}{2^n}$. Wir setzen 
		$U := \bigcup_{n \in \N} U_n$ und berechnen  
		$$\mu(U) - \mu(B) \; \leq \; \sum_{n=1}^\infty \mu(U_n \setminus B) \; \leq \; 
		\sum_{n=1}^\infty \mu(U_n \setminus B_n) \; < \; \varepsilon \text{.}$$
		Weil $U$ offen ist, folgt insgesamt die schwache Regularität von $\mu$.
	\end{proof}

	\begin{Bemerkung}
		Sofern $X$ kompakt ist, bleibt der obige Hilfssatz~\ref{lem:sigmaalg} gültig, wenn man \enquote{schwach regulär} durch \enquote{regulär} ersetzt. Der hier vorgestellte Beweis ist eine Anpassung von \cite[Lemma 4.5.5]{Simon.2015}, wo die Aussage für kompakte $X$ bewiesen wird.
	\end{Bemerkung}

	Der folgende Hilfssatz~\ref{lem:opensets} liefert uns noch eine Möglichkeit, 
	abgeschlossene Mengen von außen durch offene Mengen zu approximieren, womit wir im 
	Beweis von Satz~\ref{thm:weakregularity} die äußere Regularität von abgeschlossenen 
	Mengen zeigen werden können. Insbesondere die in Hilfssatz~\ref{lem:opensets} definierten 
	Funktionen $f_n$ werden auch noch im weiteren Verlauf nützlich sein.
	
	\begin{Hilfssatz}
		\label{lem:opensets}
		Sei $(X, d)$ ein metrischer Raum und $C \subseteq X$ eine abgeschlossene 
		Teilmenge. Ferner definieren wir für $n \in \N$
		$$ A_n \defby \setcomp{y \in X}{d(y, C) < \frac{1}{n}} \quad \text{und} \quad 
		\fctmap{f_n}{X}{\R}{x}{\max \set{0, 1-n d(x, C)}} \text{,}$$
		wobei wir $d(y, C) \defby \inf_{x \in C} d(y, x)$ setzen.
		Dann gilt:
		\begin{enumeratethm}
			\item $A_n$ ist offen für alle $n \in \N$.
			\item $C = \bigcap_{n \in \N} A_n$, insbesondere ist $C$ also eine $G_\delta$-Menge.
			\item Für alle $n$ ist $\restr{f_n}{A_n^\mathsf{c}} = 0$ und $f_n$ ist lipschitzstetig.
			\item $f_n \convdown \indfct_C$.
		\end{enumeratethm}
	\end{Hilfssatz}

	\begin{proof}
		Aussage (a) folgt aus der Stetigkeit von $y \mapsto d(y, C)$.
		
		Weiter ist $C \subseteq A_n$ für alle $n \in \N$ und damit 
		$C \subseteq \bigcap_{n \in \N} A_n$. 
		Umgekehrt gibt es für ein beliebiges $y \in \bigcap_{n \in \N} A_n$
		eine Folge $(x_n)_n \in C^\N$ mit $x_n \rightarrow y$. 
		Wegen der Abgeschlossenheit von $C$ liegt $y$ damit in $C$, sodass (b) gezeigt ist.
		
		Aussage (c) ist klar ($f_n$ ist als Komposition lipschitzstetiger Funktionen selbst lipschitzstetig).
		
		Schließlich fällt $f_n$ und für $x \in C$ gilt $f_n(x) = 1$. Für $x \in C^\mathsf{c}$ ist $d(x, C) > 0$ und damit
		$$f_n(x) = \max \set{0, 1-n d(x, C)} \to 0 \text{,} \quad n \to \infty \text{,}$$
		womit die Behauptung folgt.
	\end{proof}

	Ausgestattet mit den Hilfssätzen \ref{lem:sigmaalg} und \ref{lem:opensets} kann nun, 
	wie oben bereits angedeutet wurde, Satz~\ref{thm:weakregularity} bewiesen werden.

	\begin{proof}[Beweis von Satz~\ref{thm:weakregularity}]
		Es ist nun zu zeigen, dass für jedes $\mu \in \Probmeasures{X}$
		die Menge 
		$$\mathcal{S} \defby \setcomp{B \in \mathcal{B}}{B \text{ schwach regulär bzgl. } \mu}$$
		bereits ganz $\mathcal{B}$ ist. 
		Da $\mathcal{S}$ nach Hilfssatz~\ref{lem:sigmaalg} eine $\sigma$-Algebra ist und 
		$\mathcal{B}$ von den abgeschlossenen Mengen erzeugt wird, genügt es zu zeigen, 
		dass diese in $\mathcal{S}$ enthalten sind. 
		
		Sei $C \in \mathcal{B}$. Dann ist $C$ sicherlich schwach regulär von innen. 
		Nun verwenden wir die offenen Mengen $A_n, \; n \in \N$ aus Hilfssatz~\ref{lem:opensets}. 
		Wegen $A_n \convdown C$ und $\mu(X) < \infty$ folgt mit der Maßstetigkeit von oben die Konvergenz $\mu(A_n) \convdown \mu(C)$,
		sodass $C$ auch schwach regulär von außen ist.
	\end{proof}

	\begin{Satz}
		Sei $(X,d)$ ein metrischer Raum und seien $\mu, \nu \in \Probmeasures{X}$. Dann sind die folgenden Aussagen äquivalent:
		\begin{equivalentthm}
			\item $\mu = \nu$.
			\item Für alle gleichmäßig stetigen Funktionen $\fct{f}{X}{\R}$ ist
					 $\measureint{}{f}{\mu} = \measureint{}{f}{\nu}$.
			\item Für alle abgeschlossenen Mengen $C \in \mathcal{B}$ ist $\mu(C) = \nu(C)$.
		\end{equivalentthm}
	\end{Satz}

	\begin{proof}
		Die Implikation (i) $\Rightarrow$ (ii) ist klar und (iii) $\Rightarrow$ (i) folgt aus Satz~\ref{thm:weakregularity}.
		
		(ii) $\Rightarrow$ (iii): Gelte (ii) und sei $C \subseteq X$ abgeschlossen. 
		Dann gilt für die gleichmäßig stetigen Funktionen $f_n$ aus Hilfssatz~\ref{lem:opensets}
		$$\measureint{}{f_n}{\mu} = \measureint{}{f_n}{\nu}\text{,} \quad n \in \N \text{.}$$
		Wegen $| f_n | \leq 1$ und $f_n \convdown \indfct_C$ folgt mit dem Satz von Lebesgue 
		$$\measureint{}{f_n}{\mu} \; \to \; \mu(C) \quad \text{und} \quad \measureint{}{f_n}{\nu} 
			\; \to \; \nu(C)$$
		und damit gilt (iii).
	\end{proof}

	\tobechanged{Hier steht ein Kommentar.} $\Bdcontfct{X}$ bezeichne im Folgenden die Menge aller stetigen beschränkten Funktionen von $X$ nach $\R$.

	\begin{Definition}[Schwache Konvergenz von Maßen]
		Sei $(\mu_n)_n \in \Probmeasures{X}^\N$ eine Folge von Wahrscheinlichkeitsmaßen.
		Dann sagen wir, dass $(\mu_n)_n$ 
		\emph{schwach} gegen $\mu \in \Probmeasures{X}$ \emph{konvergiert}, falls für alle $f \in \Bdcontfct{X}$
		$$\measureint{}{f}{\mu_n} \to \measureint{}{f}{\mu}\text{,} \quad n \to \infty $$
		gilt. In diesem Fall schreiben wir
		$$\mu_n \xrightarrow{w} \mu \text{.}$$
	\end{Definition}

	Eine Charakterisierung der schwachen Konvergenz von Maßen liefert der folgende Satz:

	\begin{Satz}[Portmanteau]
		Sei $(X, d)$ ein metrischer Raum, $(\mu_n)_n \in \Probmeasures{X}^\N$ 
		und $\mu \in \Probmeasures{X}$. Dann sind die folgenden Aussagen äquivalent:
		\begin{equivalentthm}
			\item $\mu_n \xrightarrow{w} \mu$.
			\item Für alle abgeschlossenen Mengen $C \subseteq X$ gilt 
				$$\limsup_{n \to \infty} \mu_n(C) \; \leq \; \mu(C) \text{.}$$
			\item Für alle offenen Mengen $U \subseteq X$ gilt 
				$$\liminf_{n \to \infty} \mu_n(U) \; \geq \; \mu(U) \text{.}$$
			\item Für alle $B \in \mathcal{B}$ mit $\mu(\partial B) = 0$ 
				ist $$\lim_{n \to \infty} \mu_n(A) \; = \; \mu(A) \text{.}$$
		\end{equivalentthm}
	\end{Satz}

	\begin{proof}
		(i) $\Rightarrow$ (ii): Sei $C \subseteq X$ abgeschlossen und seien 
		$f_m, \; m \in \N$ die Funktionen aus Hilfssatz~\ref{lem:opensets}. 
		Diese sind stetig und beschränkt.
		Dann gilt für alle $m \in \N$
		$$\mu_n(C) \; = \; \measureint{}{\indfct_C}{\mu_n} \; \leq \; 
			\measureint{}{f_m}{\mu_n} \; \to \;
			\measureint{}{f_m}{\mu} \text{,} \quad n \to \infty \text{,}$$
		also 
		$$\limsup_{n \to \infty} \mu_n(C) \; \leq \; 
			\measureint{}{f_m}{\mu} \text{.}$$
		Wegen $f_m \convdown \indfct_C$ und $| f_m | \leq 1$ 
		liefert der Satz von Lebesgue die Konvergenz
		$$\measureint{}{f_m}{\mu} \; \to \;
			\measureint{}{\indfct_C}{\mu} \; = \; \mu(C) \text{,} \quad m \to \infty \text{,}$$
		woraus
		$$\limsup_{n \to \infty} \mu_n(C) \; \leq \; \mu(C)$$
		folgt.
		
		(ii) $\Leftrightarrow$ (iii): Es gelte (ii). Sei $U$ offen, also 
		$C \defby U^\mathsf{c}$ abgeschlossen. Dann erhalten wir
		$$\mu(X) - \liminf_{n \to \infty} \mu_n(U) \; = \; 
			\limsup_{n \to \infty} \mu_n(C) \; \leq \; 
			\mu(C) \; = \; \mu(X) - \mu(U)$$
		und damit (iii). Die andere Richtung zeigt man analog.
		
		(ii), (iii) $\Rightarrow$ (iv): Sei $A \in \mathcal{B}$ mit 
		$\mu(\partial A) = 0$. Wegen
		$A^\mathsf{o} \subseteq A \subseteq \overline{A}$ und 
			$\partial A = \overline{A} \setminus A^\mathsf{o}$ gilt $\mu(A^\mathsf{o}) = 
			\mu(A) = \mu(\overline{A}) \text{.}$
		Ferner liefern die Annahmen
		$$\limsup_{n \to \infty} \mu_n(\overline{A}) \; \leq \; 
			\mu(\overline{A}) \quad \text{und} \quad 
			\liminf_{n \to \infty} \mu_n(A^\mathsf{o}) \; \geq \; 
			\mu(A^\mathsf{o}) \text{.}$$
		Daraus folgt
		$$\limsup_{n \to \infty} \mu_n(A) \; \leq \; 
		\mu(A) \; \leq \; \liminf_{n \to \infty} \mu_n(A) \text{,}$$
		also insgesamt
		$$\lim_{n \to \infty} \mu_n(A) \; = \; \mu(A) \text{.}$$
		
		(iv) $\Rightarrow$ (i): Sei $f \in \Bdcontfct{X}$ und $a < b \in \R$ 
		mit $a < f < b$. Die Menge
		$$S \defby \setcomp{c \in (a, b)}{\mu(\set{f = c}) > 0} \text{,}$$
		ist abzählbar, da 
		$S_n \defby \setcomp{c \in (a, b)}{\mu(\set{f = c}) > \frac{1}{n}}$ 
		für jedes $n \in \N$ endlich ist und $S = \bigcup_{n \in \N} S_n$ gilt.
		Damit können wir für jedes $m \in \N$ Zahlen $c_j^{(m)} \notin S$ mit
		\[a = c_0^{(m)} < \dots < c_{2m}^{(m)} = 
			b \text{,} \qquad c_{j+1}^{(m)} - c_j^{(m)} \leq \frac{b-a}{m} 
			\label{eq:2.1} \tag{2.1}\]
		finden. Wir setzen
		$$A_j^{(m)} \defby \set{c_j^{(m)} < f \leq c_{j+1}^{(m)}}\text{,}
			 \qquad j \in \set{0,\dots,2m-1} \text{.}$$ 
		Die Stetigkeit von $f$ impliziert 
		$\partial A_j^{(m)} \subseteq \set{f = c_j^{(m)}} \cup \set{f = c_{j+1}^{(m)}}$, 
		woraus sich 
		$$\mu(\partial A_j^{(m)}) = 0 \text{,} \qquad j \in \set{0,\dots,2m-1}$$
		ergibt. Für $m \in \N$ schreiben wir
		$$u_m \defby \sum_{j=0}^{2m-1} c_j^{(m)} \indfct_{A_j^{(m)}}$$
		und Aussage (iv) führt dann auf
		\[\measureint{}{u_m}{\mu_n} \; = \; \sum_{j=0}^{2m-1} c_j^{(m)} \mu_n(A_j^{(m)}) 
			\; \to \; \sum_{j=0}^{2m-1} c_j^{(m)} \mu(A_j^{(m)}) \; = \; 
			\measureint{}{u_m}{\mu} \text{,} \quad n \to \infty \text{.} \label{eq:2.2} \tag{2.2}\]
		Außerdem folgen aus \eqref{eq:2.1} die Ungleichungen
		\[\left| \measureint{}{f}{\mu_n} - \measureint{}{u_m}{\mu_n} \right| \; \leq \; 
			\frac{b-a}{m} \text{,} \qquad 
			\left| \measureint{}{f}{\mu} - \measureint{}{u_m}{\mu} \right| \; \leq \; 
			\frac{b-a}{m} \label{eq:2.3} \tag{2.3}\]
		für $n \in \N$.
		Mit \eqref{eq:2.3} gilt nun für alle $m, n \in \N$
		\begin{align*}
			\left| \measureint{}{f}{\mu} - \measureint{}{f}{\mu_n} \right| \; &\leq \; 
			\left| \measureint{}{f}{\mu} - \measureint{}{u_m}{\mu} \right| + 
			\left| \measureint{}{u_m}{\mu} - \measureint{}{u_m}{\mu_n} \right| + 
			\left| \measureint{}{u_m}{\mu_n} - \measureint{}{f}{\mu_n} \right| \\
			&\leq \; 2 \cdot \frac{b-a}{m} + \left| \measureint{}{u_m}{\mu_n} - \measureint{}{f}{\mu_n} \right| \label{eq:2.4} \tag{2.4}\text{.}
		\end{align*}
		\eqref{eq:2.2} und \eqref{eq:2.4} liefern also letztendlich für alle $m$
		\begin{align*}
			\limsup_{n \to \infty} \left| \measureint{}{f}{\mu} - \measureint{}{f}{\mu_n} \right|
			\; &\leq \; 2 \cdot \frac{b-a}{m} + \limsup_{n \to \infty} \left| \measureint{}{u_m}{\mu_n} - \measureint{}{f}{\mu_n} \right| \\
			&= \; 2 \cdot \frac{b-a}{m} \; \to \; 0 \text{,} \quad m \to \infty \text{,}
		\end{align*}
		was (i) impliziert.
	\end{proof} 

	\begin{Definition}[Hilbertwürfel]
		Der \emph{Hilbertwürfel} ist der topologische Raum $H = [0, 1]^\N$ 
		ausgestattet mit der Produkttopologie, also der kleinsten Topologie, 
		bezüglich der alle Projektionen $\fctmap{\pi_n}{H}{[0, 1]}{x}{x_n}$ stetig sind.
	\end{Definition}

	\begin{Hilfssatz}
		\label{lem:hilbertcube}
		Für den Hilbertwürfel $H$ gelten folgende Aussagen:
		\begin{enumeratethm}
			\item Eine Folge $(x^{(k)})_k \in H^\N$ konvergiert genau dann gegen 
			ein $x \in H$, wenn alle Komponenten konvergieren, also, wenn
			$\lim_{n \to \infty} x_n^{(k)} = x_n$ für alle $n \in \N$ gilt.
			\item Setzen wir für $x, y \in H$
			$$d(x, y) \defby \max_{n \in \N} \frac{|x_n - y_n|}{2^n} \text{,}$$
			so definiert $d$ eine Metrik, die $H$ metrisiert.
		\end{enumeratethm}
	\end{Hilfssatz}

	\begin{proof}
		Die Hinrichtung von (a) folgt aus der Stetigkeit der Projektionen 
		$\pi_n, \; n \in \N$. Für die Rückrichtung bemerke man zunächst, dass Mengen der Form 
		\[U = \prod_{n=1}^{\infty} U_n\text{,} \quad U_n \subseteq [0, 1] \text{ offen, }
		 \quad U_n = [0, 1] \text{ für fast alle } n \label{eq:2.5} \tag{2.5}\]
		eine Basis der Topologie von $H$ bilden. 
		Konvergiert nun also $(x^{(k)})_k \in H^\N$ komponentenweise gegen $x \in H$, 
		so enthält jede offene Umgebung von $x$ von der Form \eqref{eq:2.5} auch fast 
		alle Folgenglieder $x^{(k)}, \; k \in \N$ und damit konvergiert 
		auch $(x^{(k)})_k$ gegen $x$.
		
		Offenbar wird durch $d$ aus (b) eine Metrik auf $H$ definiert. 
		Es genügt also zu zeigen, dass $d$ die Topologie von $H$ induziert. 
		Für $x \in H$ und $r > 0$ ist $B_r(x) = \prod_{n=1}^{\infty} B_{2^n r}(x_n) \subseteq H$ 
		offen. Ist nun $U = \prod_{n=1}^{\infty} U_n$ wie in \eqref{eq:2.5}, 
		so lässt sich leicht einsehen, dass es für jedes $x \in U$ ein $r > 0$ 
		gibt mit $B_r(x) \subseteq U$, also ist $U$ offen bezüglich der von $d$ 
		erzeugten Topologie und insgesamt wird $H$ von $d$ metrisiert.
	\end{proof}

	\begin{Bemerkung}
		Der aus der Topologie bekannte \emph{Satz von Tychonoff} liefert 
		unmittelbar, dass $H$ als Produkt kompakter topologischer Räume 
		ebenfalls kompakt ist.
		Außerdem lässt sich unter Ausnutzung von Aussage (a) aus 
		Hilfssatz~\ref{lem:hilbertcube} leicht zeigen, dass $(H, d)$ 
		vollständig ist. Weil Mengen der Form \eqref{eq:2.5}, bei denen 
		$U_n$ ausschließlich rationale Intervalle sind, eine abzählbare 
		Basis der Topologie von $H$ bilden, ist $H$ sogar separabel und 
		damit insgesamt ein kompakter polnischer Raum.
	\end{Bemerkung}

	\begin{Satz}[Charakterisierung polnischer Räume]
		Ein topologischer Raum $(X, \mathcal{O})$ ist genau dann polnisch, 
		wenn es eine $G_\delta$-Teilmenge $Y \subseteq H$ gibt, sodass $X$ 
		zu $Y$ homöomorph ist. 
		
		Darüber hinaus entsprechen die kompakten polnischen Räume bis auf 
		Homöomorphie den abgeschlossenen Teilmengen von $H$.
	\end{Satz}

	\begin{Hilfssatz}
		Sei $(X, d)$ ein separabler metrischer Raum und sei 
		$D \defby \setcomp{x_n}{n \in \N} \subseteq X$ eine abzählbare 
		dichte Teilmenge. Außerdem setzen wir
		\[\fctmap{\varphi}{X}{H}{x}{(\min(1, d(x, x_n)))_n} \text{.} \label{eq:2.6} \tag{2.6}\]
		Dann gilt das Folgende:
		\begin{enumeratethm}
			\item $\varphi$ ein Homöomorphismus zwischen $X$ und $\varphi(X)$.
			\item Ist $(X, d)$ vollständig, so ist $\varphi(X) \subseteq H$ eine $G_\delta$-Teilmenge.
		\end{enumeratethm}
	\end{Hilfssatz}

	\begin{proof}
		Wir zeigen zunächst, dass $\varphi$ injektiv ist. 
		Dazu seien $y, z \in H$ mit $\varphi(y) = \varphi(z)$, also gilt
		\[\min(1, d(y, x_n)) = \min(1, d(z, x_n)) \label{eq:2.7} \tag{2.7}\]
		für alle $n \in \N$. Weil es eine Folge $(y_n)_n \in D^\N$ mit 
		$y_n \to y$ gibt und diese wegen \eqref{eq:2.7} auch gegen $z$ 
		konvergiert, folgt die Gleichheit von $y$ und $z$.
		
		Die Stetigkeit von $\varphi$ folgt direkt, weil die 
		Komponentenfunktionen $x \mapsto \min(1, d(x, x_n))$ jeweils stetig sind. 
		Seien nun $(y_k)_k \in X^\N$ und $y \in X$ mit $\varphi(y_k) \to \varphi(y)$, 
		also
		\[\min(1, d(y_k, x_n)) \; \to \; \min(1, d(y, x_n)), 
			\quad k \to \infty \label{2.8} \tag{2.8}\]
		für alle $n \in \N$. Wählt man nun $\varepsilon < 1$ beliebig, 
		so existiert ein $n \in \N$ mit $d(y, x_n) \leq \varepsilon$. 
		Wegen \eqref{2.8} ist dann auch
		$d(y_k, x_n) \to d(y, x_k), \; k \to \infty$. Es gilt also die Ungleichung
		$$\limsup_{k \to \infty} d(y_k, y) \; \leq \; 
			\limsup_{k \to \infty} d(y_k, x_n) + d(y, x_n) \; \leq \; 2\varepsilon$$
		und mit $\varepsilon \to 0$ folgt die Stetigkeit von $\varphi^{-1}$, 
		womit Aussage (a) gezeigt ist.
	\end{proof}

	
	
\end{document}

