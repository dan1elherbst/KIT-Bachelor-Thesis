\documentclass[../main/main.tex]{subfiles}

\begin{document}
	
	\section{Wahrscheinlichkeitsmaße auf metrischen Räumen}
	
	Im folgenden Kapitel möchten wir einige Eigenschaften von Borel-Wahrscheinlichkeitsmaßen auf metrischen Räumen
	festhalten. Außerdem werden wir allgemeiner Wahrscheinlichkeitsmaße auf topologischen Räumen selbst 
	mit einer Topologie bzw. einem Konvergenzbegriff versehen, dem der sogenannten \emph{schwachen Konvergenz}, 
	welche die \emph{Verteilungskonvergenz} aus der Stochastik verallgemeinert. 
	Erst die Einschränkung auf die Klasse der polnischen Räume wird uns einige nichttriviale Erkenntnisse über die 
	Topologie dieses Raumes der Wahrscheinlichkeitsmaße ermöglichen.
	
	Zunächst müssen wir noch ein paar wenige Begriffe definieren.
	Sei $X$ ein topologischer Raum mit offenen Mengen $\mathcal{O}$. Dann definieren wir die
	\emph{Borel-$\sigma$-Algebra} 
	$$\mathcal{B}(X) = \mathcal{B} \defby \sigma(\mathcal{O}) \text{.}$$
	Maße auf $\mathcal{B}(X)$ nennen wir \emph{Borel-Maße} und wir führen außerdem die Schreibweise
	$$\Probmeasures{X} \defby \setcomp{\fct{\mu}{\mathcal{B}}{[0,1]}}{\mu \; 
		\text{ist Wahrscheinlichkeitsmaß}}$$
	für die Menge aller \emph{Borel-Wahrscheinlichkeitsmaße} auf $X$ ein.
	
	\begin{Definition}[Schwache Regularität von Maßen]
		\label{def:regularity}
		Sei $X$ ein topologischer Raum und $\mu$ ein endliches Borel-Maß auf $X$. 
		Dann nennen wir $B \in \mathcal{B}$ \emph{schwach von 
			innen bzw. von außen regulär}, falls
		$$\mu(B) = \sup_{\substack{C \subseteq B \\ C^\mathsf{c} \in \mathcal{O}}} \mu(C) 
		\quad \text{bzw.} \quad \mu(B) = \inf_{\substack{U \supseteq B \\ U \in \mathcal{O}}} 
		\mu(U)$$
		gelten. Wir nennen $B \in \mathcal{B}$ \emph{schwach regulär}, wenn $B$ schwach von innen und 
		von außen regulär ist. Sind alle $B \in \mathcal{B}$ schwach regulär, 
		so nennen wir $\mu$ ein \emph{schwach reguläres Maß}.
	\end{Definition}

	Ersetzen wir in der obigen Definition \enquote{abgeschlossen} durch \enquote{kompakt}, 
	so erhalten wir analog den Begriff der \emph{Regularität}. (Schwache) Regularität ist eine 
	Approximationseigenschaft von Maßen, die es uns etwa erlaubt, gewisse Aussagen 
	zunächst für abgeschlossene bzw. kompakte und für offene Mengen zu zeigen, um diese 
	anschließend auf ganz $\mathcal{B}$ auszuweiten.
	
	Offenbar ist eine abgeschlossene (bzw. offene) Menge schwach regulär von innen (bzw. außen).
	
	Im Falle von Wahrscheinlichkeitsmaßen auf metrischen Räumen kann schwache Regularität 
	recht einfach gezeigt werden, wie wir im Folgenden sehen werden.
	
	\begin{Satz}
		\label{thm:weakregularity}
		Ist $(X, d)$ ein metrischer Raum, so ist jedes $\mu \in \Probmeasures{X}$ schwach regulär.
	\end{Satz}
	
	Für den Beweis des Satzes benötigen wir noch Hilfssatz~\ref{lem:sigmaalg}. Dieser wird 
	uns ermöglichen, die schwache Regularität nur auf einem Erzeuger von $\mathcal{B}$ 
	zu zeigen (für den wir dann die abgeschlossenen Mengen wählen).
	
	\begin{Hilfssatz}
		\label{lem:sigmaalg}
		In der Situation von Definition~\ref{def:regularity} ist
		$$\mathcal{S} \defby \setcomp{B \in \mathcal{B}}{B \text{ schwach regulär bzgl. } \mu}$$
		eine $\sigma$-Algebra.
	\end{Hilfssatz}
	
	\begin{proof}
		Offensichtlich liegen $\emptyset$ und $X$ in $\mathcal{S}$. Sei 
		$B \in \mathcal{S}$ und damit schwach regulär 
		von innen und von außen. Wegen $\mu(X) < \infty$ gilt dann
		$$\mu(B^\mathsf{c}) 
		\, = \, \mu(X) - \mu(B) 
		\, = \, \mu(X) - \sup_{\substack{C \subseteq B \\ C^\mathsf{c} \in \mathcal{O}}} \mu(C) 
		\, = \, \inf_{\substack{C \subseteq B \\ C^\mathsf{c} \in \mathcal{O}}} (\mu(X) - \mu(C))
		\, = \, \inf_{\substack{U \supseteq B^\mathsf{c} \\ U \in \mathcal{O}}} \mu(U)$$
		sowie analog
		$$\mu(B^\mathsf{c}) 
		\, = \, \mu(X) - \mu(B) 
		\, = \, \mu(X) - \inf_{\substack{U \supseteq B \\ U \in \mathcal{O}}} \mu(C)
		\, = \, \sup_{\substack{U \supseteq B \\ U \in \mathcal{O}}} (\mu(X) - \mu(U))
		\, = \, \sup_{\substack{C \subseteq B^\mathsf{c} \\ C^\mathsf{c} \in 
				\mathcal{O}}} \mu(U) \text{.}$$
		Also ist auch $B^\mathsf{c}$ schwach regulär. 
		
		Es bleibt nun zu zeigen, dass für $(B_n)_n \in \mathcal{S}^\N$ auch $B \defby 
		\bigcup_{n \in \N} B_n$ schwach regulär ist. 
		Hierfür beweisen wir zunächst die schwache Regularität von innen. 
		Sei dazu $\varepsilon > 0$. Für $n \in \N$ gibt es jeweils abgeschlossene Mengen 
		$C_n \subseteq B_n$ mit $\mu(B_n) - \mu(C_n) < \frac{\varepsilon}{3^n}$.
		Wir wählen nun $N$ so groß, dass $\mu\left( B \setminus \bigcup_{n=1}^N B_n \right) 
		< \frac{\varepsilon}{2}$ ist (was wegen $\mu(B) < \infty$ immer geht). 
		Für die abgeschlossene Menge $C := \bigcup_{n=1}^N C_k$ gilt dann die Ungleichung 
		\begin{align*}
			\mu(B) - \mu(C) = \mu(B\setminus C) \; &=
			\; \mu\left( \left( B \setminus \bigcup_{n=1}^N B_n \right) \; \cup \; 
			\left( \bigcup_{n=1}^N B_n  \setminus C \right) \right) \\
			&\leq \; \mu \left( B \setminus \bigcup_{n=1}^N B_n \right) + 
			\sum_{n=1}^{N} \mu(B_n \setminus C_n) \\
			&<    \; \frac{\varepsilon}{2} + 
			\sum_{n=1}^{\infty} \frac{\varepsilon}{3^n} \; = \; \varepsilon \text{,}
		\end{align*}
		also ist $B$ schwach regulär von innen.
		
		Ferner existieren für alle $n$ offene Mengen $U_n \supseteq B_n$ 
		mit $\mu(U_n) - \mu(B_n) < \frac{\varepsilon}{2^n}$. Wir setzen 
		$U := \bigcup_{n \in \N} U_n$ und berechnen  
		$$\mu(U) - \mu(B) \; \leq \; \sum_{n=1}^\infty \mu(U_n \setminus B) \; \leq \; 
		\sum_{n=1}^\infty \mu(U_n \setminus B_n) \; < \; \varepsilon \text{.}$$
		Weil $U$ offen ist, folgt insgesamt die schwache Regularität von $\mu$.
	\end{proof}
	
	\begin{Bemerkung}
		Sofern $X$ kompakt ist, bleibt der obige Hilfssatz~\ref{lem:sigmaalg} 
		gültig, wenn man \enquote{schwach regulär} durch \enquote{regulär} 
		ersetzt. Der hier vorgestellte Beweis ist eine Anpassung von 
		\cite[Lemma 4.5.5]{Simon.2015}, wo die Aussage für kompakte $X$ bewiesen wird.
	\end{Bemerkung}
	
	Ausgestattet mit den Hilfssätzen \ref{lem:opensets} und \ref{lem:sigmaalg} kann nun, 
	wie oben bereits angedeutet wurde, Satz~\ref{thm:weakregularity} bewiesen werden.
	
	\begin{proof}[Beweis von Satz~\ref{thm:weakregularity}]
		Es ist nun zu zeigen, dass für jedes $\mu \in \Probmeasures{X}$
		die Menge 
		$$\mathcal{S} \defby \setcomp{B \in \mathcal{B}}{B \text{ schwach regulär bzgl. } \mu}$$
		bereits ganz $\mathcal{B}$ ist. 
		Da $\mathcal{S}$ nach Hilfssatz~\ref{lem:sigmaalg} eine 
		$\sigma$-Algebra ist und 
		$\mathcal{B}$ von den abgeschlossenen Mengen erzeugt wird, genügt es zu zeigen, 
		dass diese in $\mathcal{S}$ enthalten sind. 
		
		Sei $C \in \mathcal{B}$. Dann ist $C$ sicherlich schwach regulär von innen. 
		Nun verwenden wir die offenen Mengen $A_n, \; n \in \N$ aus 
		Hilfssatz~\ref{lem:opensets}. 
		Wegen $A_n \convdown C$ und $\mu(X) < \infty$ folgt mit der 
		Maßstetigkeit von oben die Konvergenz $\mu(A_n) \convdown \mu(C)$,
		sodass $C$ auch schwach regulär von außen ist.
	\end{proof}

\tobechanged{Ab hier fehlen Kommentare.}
	
	\begin{Satz}
		Sei $(X,d)$ ein metrischer Raum und seien $\mu, \nu \in \Probmeasures{X}$. 
		Dann sind die folgenden Aussagen äquivalent:
		\begin{equivalentthm}
			\item $\mu = \nu$.
			\item Für alle gleichmäßig stetigen Funktionen $\fct{f}{X}{\R}$ ist
			$\measureint{}{f}{\mu} = \measureint{}{f}{\nu}$.
			\item Für alle abgeschlossenen Mengen $C \in \mathcal{B}$ ist $\mu(C) = \nu(C)$.
		\end{equivalentthm}
	\end{Satz}
	
	\begin{proof}
		Die Implikation (i) $\Rightarrow$ (ii) ist klar und (iii) 
		$\Rightarrow$ (i) folgt aus Satz~\ref{thm:weakregularity}.
		
		(ii) $\Rightarrow$ (iii): Gelte (ii) und sei $C \subseteq X$ abgeschlossen. 
		Dann gilt für die gleichmäßig stetigen Funktionen $f_n$ aus Hilfssatz~\ref{lem:opensets}
		$$\measureint{}{f_n}{\mu} = \measureint{}{f_n}{\nu}\text{,} \quad n \in \N \text{.}$$
		Wegen $| f_n | \leq 1$ und $f_n \convdown \indfct_C$ folgt mit dem Satz von Lebesgue 
		$$\measureint{}{f_n}{\mu} \; \to \; \mu(C) \quad \text{und} \quad \measureint{}{f_n}{\nu} 
		\; \to \; \nu(C)$$
		und damit gilt (iii).
	\end{proof}
	
	$\Bdcontfct{X}$ bezeichne 
	im Folgenden die Menge aller stetigen beschränkten Funktionen von $X$ nach $\R$.
	
	\begin{Definition}[Schwache Konvergenz von Maßen]
		\label{def:weakconvergence}
		Sei $(\mu_n)_n \in \Probmeasures{X}^\N$ eine Folge von Wahrscheinlichkeitsmaßen.
		Dann sagen wir, dass $(\mu_n)_n$ 
		\emph{schwach} gegen $\mu \in \Probmeasures{X}$ \emph{konvergiert}, 
		falls für alle $f \in \Bdcontfct{X}$
		$$\measureint{}{f}{\mu_n} \to \measureint{}{f}{\mu}\text{,} 
		\quad n \to \infty $$
		gilt. In diesem Fall schreiben wir
		$$\mu_n \xrightarrow{w} \mu \text{.}$$
	\end{Definition}
	
	Eine Charakterisierung der schwachen Konvergenz von Maßen liefert der folgende Satz:
	
	\begin{Satz}[Portmanteau]
		\label{thm:portmanteau}
		Sei $(X, d)$ ein metrischer Raum, $(\mu_n)_n \in \Probmeasures{X}^\N$ 
		und $\mu \in \Probmeasures{X}$. Dann sind die folgenden Aussagen äquivalent:
		\begin{equivalentthm}
			\item $\mu_n \xrightarrow{w} \mu$.
			\item Für alle abgeschlossenen Mengen $C \subseteq X$ gilt 
			$$\limsup_{n \to \infty} \mu_n(C) \; \leq \; \mu(C) \text{.}$$
			\item Für alle offenen Mengen $U \subseteq X$ gilt 
			$$\liminf_{n \to \infty} \mu_n(U) \; \geq \; \mu(U) \text{.}$$
			\item Für alle $B \in \mathcal{B}$ mit $\mu(\partial B) = 0$ 
			ist $$\lim_{n \to \infty} \mu_n(A) \; = \; \mu(A) \text{.}$$
		\end{equivalentthm}
	\end{Satz}
	
	\begin{proof}
		(i) $\Rightarrow$ (ii): Sei $C \subseteq X$ abgeschlossen und seien 
		$f_m, \; m \in \N$ die Funktionen aus Hilfssatz~\ref{lem:opensets}. 
		Diese sind stetig und beschränkt.
		Dann gilt für alle $m \in \N$
		$$\mu_n(C) \; = \; \measureint{}{\indfct_C}{\mu_n} \; \leq \; 
		\measureint{}{f_m}{\mu_n} \; \to \;
		\measureint{}{f_m}{\mu} \text{,} \quad n \to \infty \text{,}$$
		also 
		$$\limsup_{n \to \infty} \mu_n(C) \; \leq \; 
		\measureint{}{f_m}{\mu} \text{.}$$
		Wegen $f_m \convdown \indfct_C$ und $| f_m | \leq 1$ 
		liefert der Satz von Lebesgue die Konvergenz
		$$\measureint{}{f_m}{\mu} \; \to \;
		\measureint{}{\indfct_C}{\mu} \; = \; \mu(C) \text{,} 
		\quad m \to \infty \text{,}$$
		woraus
		$$\limsup_{n \to \infty} \mu_n(C) \; \leq \; \mu(C)$$
		folgt.
		
		(ii) $\Leftrightarrow$ (iii): Es gelte (ii). Sei $U$ offen, also 
		$C \defby U^\mathsf{c}$ abgeschlossen. Dann erhalten wir
		$$\mu(X) - \liminf_{n \to \infty} \mu_n(U) \; = \; 
		\limsup_{n \to \infty} \mu_n(C) \; \leq \; 
		\mu(C) \; = \; \mu(X) - \mu(U)$$
		und damit (iii). Die andere Richtung zeigt man analog.
		
		(ii), (iii) $\Rightarrow$ (iv): Sei $A \in \mathcal{B}$ mit 
		$\mu(\partial A) = 0$. Wegen
		$A^\mathsf{o} \subseteq A \subseteq \overline{A}$ und 
		$\partial A = \overline{A} \setminus A^\mathsf{o}$ gilt $\mu(A^\mathsf{o}) = 
		\mu(A) = \mu(\overline{A}) \text{.}$
		Ferner liefern die Annahmen
		$$\limsup_{n \to \infty} \mu_n(\overline{A}) \; \leq \; 
		\mu(\overline{A}) \quad \text{und} \quad 
		\liminf_{n \to \infty} \mu_n(A^\mathsf{o}) \; \geq \; 
		\mu(A^\mathsf{o}) \text{.}$$
		Daraus folgt
		$$\limsup_{n \to \infty} \mu_n(A) \; \leq \; 
		\mu(A) \; \leq \; \liminf_{n \to \infty} \mu_n(A) \text{,}$$
		also insgesamt
		$$\lim_{n \to \infty} \mu_n(A) \; = \; \mu(A) \text{.}$$
		
		(iv) $\Rightarrow$ (i): Sei $f \in \Bdcontfct{X}$ und $a < b \in \R$ 
		mit $a < f < b$. Die Menge
		$$S \defby \setcomp{c \in (a, b)}{\mu(\set{f = c}) > 0} \text{,}$$
		ist abzählbar, da 
		$S_n \defby \setcomp{c \in (a, b)}{\mu(\set{f = c}) > \frac{1}{n}}$ 
		für jedes $n \in \N$ endlich ist und $S = \bigcup_{n \in \N} S_n$ gilt.
		Damit können wir für jedes $m \in \N$ Zahlen $c_j^{(m)} \notin S$ mit
		\[a = c_0^{(m)} < \dots < c_{2m}^{(m)} = 
		b \text{,} \qquad c_{j+1}^{(m)} - c_j^{(m)} \leq \frac{b-a}{m} 
		\label{eq:2.1} \tag{2.1}\]
		finden. Wir setzen
		$$A_j^{(m)} \defby \set{c_j^{(m)} < f \leq c_{j+1}^{(m)}}\text{,}
		\qquad j \in \set{0,\dots,2m-1} \text{.}$$ 
		Die Stetigkeit von $f$ impliziert 
		$\partial A_j^{(m)} \subseteq \set{f = c_j^{(m)}} \cup \set{f = c_{j+1}^{(m)}}$, 
		woraus sich 
		$$\mu(\partial A_j^{(m)}) = 0 \text{,} \qquad j \in \set{0,\dots,2m-1}$$
		ergibt. Für $m \in \N$ schreiben wir
		$$u_m \defby \sum_{j=0}^{2m-1} c_j^{(m)} \indfct_{A_j^{(m)}}$$
		und Aussage (iv) führt dann auf
		\[\measureint{}{u_m}{\mu_n} \; = \; \sum_{j=0}^{2m-1} c_j^{(m)} \mu_n(A_j^{(m)}) 
		\; \to \; \sum_{j=0}^{2m-1} c_j^{(m)} \mu(A_j^{(m)}) \; = \; 
		\measureint{}{u_m}{\mu} \text{,} \quad n \to \infty \text{.} 
		\label{eq:2.2} \tag{2.2}\]
		Außerdem folgen aus \eqref{eq:2.1} die Ungleichungen
		\[\left| \measureint{}{f}{\mu_n} - \measureint{}{u_m}{\mu_n} \right| \; \leq \; 
		\frac{b-a}{m} \text{,} \qquad 
		\left| \measureint{}{f}{\mu} - \measureint{}{u_m}{\mu} \right| \; \leq \; 
		\frac{b-a}{m} \label{eq:2.3} \tag{2.3}\]
		für $n \in \N$.
		Mit \eqref{eq:2.3} gilt nun für alle $m, n \in \N$
		\begin{align*}
			\left| \measureint{}{f}{\mu} - \measureint{}{f}{\mu_n} \right| \; &\leq \; 
			\left| \measureint{}{f}{\mu} - \measureint{}{u_m}{\mu} \right| + 
			\left| \measureint{}{u_m}{\mu} - \measureint{}{u_m}{\mu_n} \right| + 
			\left| \measureint{}{u_m}{\mu_n} - \measureint{}{f}{\mu_n} \right| \\
			&\leq \; 2 \cdot \frac{b-a}{m} + \left| \measureint{}{u_m}{\mu_n} - 
			\measureint{}{f}{\mu_n} \right| \label{eq:2.4} \tag{2.4}\text{.}
		\end{align*}
		\eqref{eq:2.2} und \eqref{eq:2.4} liefern also letztendlich für alle $m$
		\begin{align*}
			\limsup_{n \to \infty} \left| \measureint{}{f}{\mu} - 
			\measureint{}{f}{\mu_n} \right|
			\; &\leq \; 2 \cdot \frac{b-a}{m} + 
			\limsup_{n \to \infty} \left| \measureint{}{u_m}{\mu_n} - 
			\measureint{}{f}{\mu_n} \right| \\
			&= \; 2 \cdot \frac{b-a}{m} \; \to \; 0 \text{,} 
			\quad m \to \infty \text{,}
		\end{align*}
		was (i) impliziert.
	\end{proof}
	
\end{document}