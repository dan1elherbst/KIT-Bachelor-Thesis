\documentclass[../main/main.tex]{subfiles}

\begin{document}
	
	\section{Maße auf metrischen Räumen}
	
	Im folgenden Kapitel möchten wir einige Eigenschaften von endlichen Maßen auf topologischen bzw. metrischen Räumen
	festhalten und diese selbst mit einem Konvergenzbegriff sowie einer zugehörigen Topologie versehen.
	
	Hierzu müssen wir zunächst noch ein paar wenige Begriffe definieren.
	Sei $X$ ein topologischer Raum mit der Topologie $\mathcal{O}$. Dann definieren wir die
	\emph{Borel-$\sigma$-Algebra} 
	$$\mathcal{B}(X) = \mathcal{B} \defby \sigma(\mathcal{O}) \text{.}$$
	Maße auf $\mathcal{B}(X)$ nennen wir \emph{Borel-Maße} und wir führen außerdem die Schreibweisen
	\begin{align*}
		\Finitemeasures{X}  \; &\defby \; \setcomp{\fct{\mu}{\mathcal{B}(X)}{[0,1]}}{\mu \; 
			\text{ist endliches Maß}} \\
		\Probmeasures{X} \; &\defby \; \setcomp{\fct{\mu}{\mathcal{B}(X)}{[0,1]}}{\mu \; 
			\text{ist Wahrscheinlichkeitsmaß}}
	\end{align*}
	für die Menge aller \emph{endlichen Borel-Maße} beziehungsweise \emph{Borel-Wahrscheinlichkeitsmaße} auf $X$ ein.
	
	\subsection{(Schwache) Regularität}
	
	(Schwache) Regularität ist eine 
	Approximationseigenschaft von Maßen, die es uns etwa erlaubt, gewisse Aussagen 
	zunächst für abgeschlossene bzw. kompakte und für offene Mengen zu zeigen, um diese 
	anschließend auf ganz $\mathcal{B}(X)$ auszuweiten. 
	
	\begin{Definition}[Schwache Regularität von Maßen]
		\label{def:regularity}
		Sei $X$ ein topologischer Raum und $\mu$ ein endliches Borel-Maß auf $X$. 
		Dann nennen wir $B \in \mathcal{B}(X)$ \emph{schwach von 
			innen bzw. von außen regulär}, falls
		$$\mu(B) = \sup_{\substack{C \subseteq B \\ C \; \text{abgeschlossen}}} \mu(C) 
		\quad \text{bzw.} \quad \mu(B) = \inf_{\substack{U \supseteq B \\ U \; \text{offen}}} 
		\mu(U)$$
		gelten. Wir nennen $B \in \mathcal{B}(X)$ \emph{schwach regulär}, wenn $B$ schwach von innen und 
		von außen regulär ist. Sind alle $B \in \mathcal{B}(X)$ schwach regulär, 
		so nennen wir $\mu$ ein \emph{schwach reguläres Maß}.
	\end{Definition}

	\begin{Bemerkung}
		Ersetzen wir in der obigen Definition \enquote{abgeschlossen} durch \enquote{kompakt}, 
		so erhalten wir analog den Begriff der \emph{Regularität}. 
		
		Offenbar ist eine abgeschlossene (bzw. offene) Menge schwach regulär von innen (bzw. außen).
	\end{Bemerkung}
	
	Im Falle von endlichen Maßen auf metrischen Räumen kann schwache Regularität 
	recht leicht gezeigt werden, wie wir im Folgenden sehen werden.
	
	\begin{Satz}
		\label{thm:weakregularity}
		Ist $(X, d)$ ein metrischer Raum, so ist jedes endliche Borel-Maß $\mu$ auf $X$ schwach regulär.
	\end{Satz}
	
	Für den Beweis des Satzes benötigen wir noch einen Hilfssatz, welcher 
	uns ermöglichen wird, die schwache Regularität nur auf einem Erzeuger von $\mathcal{B}(X)$ 
	zu zeigen (für den wir dann die abgeschlossenen Mengen wählen).
	
	\begin{Hilfssatz}
		\label{lem:sigmaalg}
		In der Situation von Definition~\ref{def:regularity} ist
		$$\mathcal{S} \defby \setcomp{B \in \mathcal{B}(X)}{B \text{ schwach regulär bzgl. } \mu}$$
		eine $\sigma$-Algebra.
	\end{Hilfssatz}
	
	\begin{proof}
		Offensichtlich liegen $\emptyset$ und $X$ in $\mathcal{S}$. Sei 
		$B \in \mathcal{S}$ und damit schwach regulär 
		von innen und von außen. Wegen $\mu(X) < \infty$ gilt dann
		$$\mu(B^\mathsf{c}) 
		\, = \, \mu(X) - \mu(B) 
		\, = \, \mu(X) - \sup_{\substack{C \subseteq B \\ C^\mathsf{c} \in \mathcal{O}}} \mu(C) 
		\, = \, \inf_{\substack{C \subseteq B \\ C^\mathsf{c} \in \mathcal{O}}} (\mu(X) - \mu(C))
		\, = \, \inf_{\substack{U \supseteq B^\mathsf{c} \\ U \in \mathcal{O}}} \mu(U)$$
		sowie analog
		$$\mu(B^\mathsf{c}) 
		\, = \, \mu(X) - \mu(B) 
		\, = \, \mu(X) - \inf_{\substack{U \supseteq B \\ U \in \mathcal{O}}} \mu(C)
		\, = \, \sup_{\substack{U \supseteq B \\ U \in \mathcal{O}}} (\mu(X) - \mu(U))
		\, = \, \sup_{\substack{C \subseteq B^\mathsf{c} \\ C^\mathsf{c} \in 
				\mathcal{O}}} \mu(U) \text{.}$$
		Also ist auch $B^\mathsf{c}$ schwach regulär. 
		
		Es bleibt nun zu zeigen, dass für $(B_n)_n \in \mathcal{S}^\N$ auch $B \defby 
		\bigcup_{n \in \N} B_n$ schwach regulär ist. 
		Hierfür beweisen wir zunächst die schwache Regularität von innen. 
		Sei dazu $\varepsilon > 0$. Für $n \in \N$ gibt es jeweils abgeschlossene Mengen 
		$C_n \subseteq B_n$ mit $\mu(B_n) - \mu(C_n) < \frac{\varepsilon}{3^n}$.
		Wir wählen nun $N$ so groß, dass $\mu\left( B \setminus \bigcup_{n=1}^N B_n \right) 
		< \frac{\varepsilon}{2}$ ist (was wegen $\mu(B) < \infty$ immer geht). 
		Für die abgeschlossene Menge $C := \bigcup_{n=1}^N C_k$ gilt dann die Ungleichung 
		\begin{align*}
			\mu(B) - \mu(C) = \mu(B\setminus C) \; &=
			\; \mu\left( \left( B \setminus \bigcup_{n=1}^N B_n \right) \; \cup \; 
			\left( \bigcup_{n=1}^N B_n  \setminus C \right) \right) \\
			&\leq \; \mu \left( B \setminus \bigcup_{n=1}^N B_n \right) + 
			\sum_{n=1}^{N} \mu(B_n \setminus C_n) \\
			&<    \; \frac{\varepsilon}{2} + 
			\sum_{n=1}^{\infty} \frac{\varepsilon}{3^n} \; = \; \varepsilon \text{,}
		\end{align*}
		also ist $B$ schwach regulär von innen.
		
		Ferner existieren für alle $n$ offene Mengen $U_n \supseteq B_n$ 
		mit $\mu(U_n) - \mu(B_n) < \frac{\varepsilon}{2^n}$. Wir setzen 
		$U := \bigcup_{n \in \N} U_n$ und berechnen  
		$$\mu(U) - \mu(B) \; \leq \; \sum_{n=1}^\infty \mu(U_n \setminus B) \; \leq \; 
		\sum_{n=1}^\infty \mu(U_n \setminus B_n) \; < \; \varepsilon \text{.}$$
		Weil $U$ offen ist, folgt insgesamt die schwache Regularität von $\mu$.
	\end{proof}
	
	\begin{Bemerkung}
		Sofern $X$ kompakt ist, bleibt der obige Hilfssatz~\ref{lem:sigmaalg} 
		gültig, wenn man \enquote{schwach regulär} durch \enquote{regulär} 
		ersetzt. Der hier vorgestellte Beweis ist eine Anpassung von 
		\cite[Lemma 4.5.5]{Simon.2015}, wo die Aussage für kompakte $X$ bewiesen wird.
	\end{Bemerkung}
	
	Ausgestattet mit den Hilfssätzen \ref{lem:opensets} und \ref{lem:sigmaalg} kann nun, 
	wie oben bereits angedeutet wurde, Satz~\ref{thm:weakregularity} bewiesen werden.
	
	\begin{proof}[Beweis von Satz~\ref{thm:weakregularity}]
		Es ist nun zu zeigen, dass für jedes endliche Borel-Maß $\mu$ auf $X$
		die Menge 
		$$\mathcal{S} \defby \setcomp{B \in \mathcal{B}(X)}{B \text{ schwach regulär bzgl. } \mu}$$
		bereits ganz $\mathcal{B}$ ist. 
		Da $\mathcal{S}$ nach Hilfssatz~\ref{lem:sigmaalg} eine 
		$\sigma$-Algebra ist und 
		$\mathcal{B}(X)$ von den abgeschlossenen Mengen erzeugt wird, genügt es zu zeigen, 
		dass diese in $\mathcal{S}$ enthalten sind. 
		
		Sei $C \in \mathcal{B}(X)$. Dann ist $C$ sicherlich schwach regulär von innen. 
		Nun verwenden wir die offenen Mengen $A_n, \; n \in \N$ aus 
		Hilfssatz~\ref{lem:opensets}. 
		Wegen $A_n \convdown C$ und $\mu(X) < \infty$ folgt mit der 
		Maßstetigkeit von oben die Konvergenz $\mu(A_n) \convdown \mu(C)$,
		sodass $C$ auch schwach regulär von außen ist.
	\end{proof}

	\subsection{Schwache Konvergenz}
	
	Wie zuvor angekündigt werden wir nun endliche Borel-Maße auf topologischen Räumen selbst
	mit einer Topologie bzw. einem Konvergenzbegriff versehen, dem der sogenannten \emph{schwachen Konvergenz}, 
	welche die aus der Stochastik bekannte \emph{Verteilungskonvergenz} auf andere Räume als $\R$ verallgemeinert. 
	Erst die Einschränkung auf die Klasse der polnischen Räume als Grundraum wird uns einige nichttriviale Erkenntnisse 
	über die Topologie dieses Raumes der Wahrscheinlichkeitsmaße ermöglichen.
	
	$\Bdcontfct{X}$ bezeichne 
	im Folgenden die Menge aller stetigen beschränkten Funktionen von $X$ nach $\R$.
	
	\begin{Definition}[Schwache Konvergenz von Maßen]
		\label{def:weakconvergence}
		Sei $(\mu_n)_n \in \Finitemeasures{X}^\N$ und $\mu \in \Finitemeasures{X}$. 
		Dann sagen wir, dass $(\mu_n)_n$ 
		\emph{schwach} gegen $\mu \in \Probmeasures{X}$ \emph{konvergiert}, 
		falls für alle $f \in \Bdcontfct{X}$
		$$\measureint{}{f}{\mu_n} \to \measureint{}{f}{\mu}\text{,} 
		\quad n \to \infty $$
		gilt. In diesem Fall schreiben wir
		$$\mu_n \xrightarrow{w} \mu \text{.}$$
	\end{Definition}

	\begin{Definition}[Topologie der schwachen Konvergenz]
		Für $f \in \Bdcontfct{X}$ sei
		$$\fctmap{I_f}{\Finitemeasures{X}}{\R}{\mu}{\measureint{}{f}{\mu}} \text{.}$$
		Bezeichnen wir nun mit $\mathcal{O}_w$ die kleinste Topologie auf $\Finitemeasures{X}$, bezüglich der alle
		$I_f$ für $f \in \Bdcontfct{X}$ stetig sind, so entspricht Konvergenz bezüglich dieser Topologie exakt
		der schwachen Konvergenz.
	\end{Definition}

	\begin{Bemerkung}
		Möchte man ausschließlich Wahrscheinlichkeitsmaße betrachten, so definiert man die Topologie der schwachen Konvergenz
		auf $\Probmeasures{X}$ völlig analog. Offenbar entspricht diese dann gerade der Teilraumtopologie von 
		$\Probmeasures{X} \subseteq (\Finitemeasures{X}, \mathcal{O}_w)$.
	\end{Bemerkung}
	
	Im Falle metrischer Räume liefert der folgende Satz eine umfangreiche Charakterisierung der schwachen Konvergenz.
	
	\begin{Satz}[Portmanteau]
		\label{thm:portmanteau}
		Sei $(X, d)$ ein metrischer Raum, $(\mu_n)_n \in \Finitemeasures{X}^\N$ 
		und $\mu \in \Finitemeasures{X}$. Dann sind die folgenden Aussagen äquivalent:
		\begin{equivalentthm}
			\item $\mu_n \xrightarrow{w} \mu$.
			\item Es ist 
			$\lim_{n \to \infty} \mu_n(X) = \mu(X)$
			und für alle abgeschlossenen Mengen $C \subseteq X$ gilt 
			$$\limsup_{n \to \infty} \mu_n(C) \; \leq \; \mu(C) \text{.}$$
			\item Es ist 
			$\lim_{n \to \infty} \mu_n(X) = \mu(X)$
			und für alle offenen Mengen $U \subseteq X$ gilt 
			$$\liminf_{n \to \infty} \mu_n(U) \; \geq \; \mu(U) \text{.}$$
			\item Für alle $B \in \mathcal{B}$ mit $\mu(\partial B) = 0$ 
			ist $$\lim_{n \to \infty} \mu_n(A) \; = \; \mu(A) \text{.}$$
		\end{equivalentthm}
	\end{Satz}
	
	\begin{proof}
		(i) $\Rightarrow$ (ii): Sei $C \subseteq X$ abgeschlossen und seien 
		$f_m, \; m \in \N$ die Funktionen aus Hilfssatz~\ref{lem:opensets}. 
		Diese sind stetig und beschränkt.
		Dann gilt für alle $m \in \N$
		$$\mu_n(C) \; = \; \measureint{}{\indfct_C}{\mu_n} \; \leq \; 
		\measureint{}{f_m}{\mu_n} \; \to \;
		\measureint{}{f_m}{\mu} \text{,} \quad n \to \infty \text{,}$$
		also 
		$$\limsup_{n \to \infty} \mu_n(C) \; \leq \; 
		\measureint{}{f_m}{\mu} \text{.}$$
		Wegen $f_m \convdown \indfct_C$ und $| f_m | \leq 1$ 
		liefert der Satz von Lebesgue die Konvergenz
		$$\measureint{}{f_m}{\mu} \; \to \;
		\measureint{}{\indfct_C}{\mu} \; = \; \mu(C) \text{,} 
		\quad m \to \infty \text{,}$$
		woraus
		$$\limsup_{n \to \infty} \mu_n(C) \; \leq \; \mu(C)$$
		folgt. Außerdem ist 
		$$\mu_n(X) \; = \; \measureint{}{\indfct_{X}}{\mu_n} \; \to \; \measureint{}{\indfct_{X}}{\mu} \; = \; \mu(X) \text{,} 
		\quad n \to \infty \text{.}$$
		
		(ii) $\Leftrightarrow$ (iii): Es gelte (ii). Sei $U$ offen, also 
		$C \defby U^\mathsf{c}$ abgeschlossen. Dann erhalten wir
		\begin{align*}
			\liminf_{n \to \infty} \mu_n(U) \; &= \; \liminf_{n \to \infty} (\mu_n(X) - \mu_n(C)) \; = \;
			\mu(X) - \limsup_{n \to \infty} \mu_n(C) \\
			&\geq \; 
			\mu(X) - \mu(C) \; = \; \mu(U)
		\end{align*}
		und damit (iii). Die andere Richtung zeigt man analog.
		
		(ii), (iii) $\Rightarrow$ (iv): Sei $A \in \mathcal{B}$ mit 
		$\mu(\partial A) = 0$. Wegen
		$A^\mathsf{o} \subseteq A \subseteq \overline{A}$ und 
		$\partial A = \overline{A} \setminus A^\mathsf{o}$ gilt $\mu(A^\mathsf{o}) = 
		\mu(A) = \mu(\overline{A}) \text{.}$
		Ferner liefern die Annahmen
		$$\limsup_{n \to \infty} \mu_n(\overline{A}) \; \leq \; 
		\mu(\overline{A}) \quad \text{und} \quad 
		\liminf_{n \to \infty} \mu_n(A^\mathsf{o}) \; \geq \; 
		\mu(A^\mathsf{o}) \text{.}$$
		Daraus folgt
		$$\limsup_{n \to \infty} \mu_n(A) \; \leq \; 
		\mu(A) \; \leq \; \liminf_{n \to \infty} \mu_n(A) \text{,}$$
		also insgesamt
		$$\lim_{n \to \infty} \mu_n(A) \; = \; \mu(A) \text{.}$$
		
		(iv) $\Rightarrow$ (i): Sei $f \in \Bdcontfct{X}$ und $a < b \in \R$ 
		mit $a < f < b$. Die Menge
		$$S \defby \setcomp{c \in (a, b)}{\mu(\set{f = c}) > 0} \text{,}$$
		ist abzählbar, da 
		$S_n \defby \setcomp{c \in (a, b)}{\mu(\set{f = c}) > \frac{1}{n}}$ 
		für jedes $n \in \N$ endlich ist und $S = \bigcup_{n \in \N} S_n$ gilt.
		Damit können wir für jedes $m \in \N$ Zahlen $c_j^{(m)} \notin S$ mit
		\[a = c_0^{(m)} < \dots < c_{2m}^{(m)} = 
		b \text{,} \qquad c_{j+1}^{(m)} - c_j^{(m)} \leq \frac{b-a}{m} 
		\label{eq:3.1} \tag{3.1}\]
		finden. Wir setzen
		$$A_j^{(m)} \defby \set{c_j^{(m)} < f \leq c_{j+1}^{(m)}}\text{,}
		\qquad j \in \set{0,\dots,2m-1} \text{.}$$ 
		Die Stetigkeit von $f$ impliziert 
		$\partial A_j^{(m)} \subseteq \set{f = c_j^{(m)}} \cup \set{f = c_{j+1}^{(m)}}$, 
		woraus sich 
		$$\mu(\partial A_j^{(m)}) = 0 \text{,} \qquad j \in \set{0,\dots,2m-1}$$
		ergibt. Für $m \in \N$ schreiben wir
		$$u_m \defby \sum_{j=0}^{2m-1} c_j^{(m)} \indfct_{A_j^{(m)}}$$
		und Aussage (iv) führt dann auf
		\[\measureint{}{u_m}{\mu_n} \; = \; \sum_{j=0}^{2m-1} c_j^{(m)} \mu_n(A_j^{(m)}) 
		\; \to \; \sum_{j=0}^{2m-1} c_j^{(m)} \mu(A_j^{(m)}) \; = \; 
		\measureint{}{u_m}{\mu} \text{,} \quad n \to \infty \text{.} 
		\label{eq:3.2} \tag{3.2}\]
		Außerdem folgen aus \eqref{eq:3.1} die Ungleichungen
		\[\left| \measureint{}{f}{\mu_n} - \measureint{}{u_m}{\mu_n} \right| \; \leq \; 
		\frac{b-a}{m} \text{,} \qquad 
		\left| \measureint{}{f}{\mu} - \measureint{}{u_m}{\mu} \right| \; \leq \; 
		\frac{b-a}{m} \label{eq:3.3} \tag{3.3}\]
		für $n \in \N$.
		Mit \eqref{eq:3.3} gilt nun für alle $m, n \in \N$
		\begin{align*}
			\left| \measureint{}{f}{\mu} - \measureint{}{f}{\mu_n} \right| \; &\leq \; 
			\left| \measureint{}{f}{\mu} - \measureint{}{u_m}{\mu} \right| + 
			\left| \measureint{}{u_m}{\mu} - \measureint{}{u_m}{\mu_n} \right| + 
			\left| \measureint{}{u_m}{\mu_n} - \measureint{}{f}{\mu_n} \right| \\
			&\leq \; 2 \cdot \frac{b-a}{m} + \left| \measureint{}{u_m}{\mu_n} - 
			\measureint{}{f}{\mu_n} \right| \label{eq:3.4} \tag{3.4}\text{.}
		\end{align*}
		\eqref{eq:3.2} und \eqref{eq:3.4} liefern also letztendlich für alle $m$
		\begin{align*}
			\limsup_{n \to \infty} \left| \measureint{}{f}{\mu} - 
			\measureint{}{f}{\mu_n} \right|
			\; &\leq \; 2 \cdot \frac{b-a}{m} + 
			\limsup_{n \to \infty} \left| \measureint{}{u_m}{\mu_n} - 
			\measureint{}{f}{\mu_n} \right| \\
			&= \; 2 \cdot \frac{b-a}{m} \; \to \; 0 \text{,} 
			\quad m \to \infty \text{,}
		\end{align*}
		was (i) impliziert.
	\end{proof}

	In erster Linie stellt der nächste Satz etwas schwächere hinreichende Bedingungen für die Gleichheit 
	zweier endlicher Maße auf metrischen Räumen bereit, was sich im weiteren Verlauf 
	noch als nützlich erweisen wird.
	
	\begin{Satz}
		\label{thm:measureequality}
		Sei $(X,d)$ ein metrischer Raum und seien $\mu, \nu \in \Finitemeasures{X}$. 
		Dann sind die folgenden Aussagen äquivalent:
		\begin{equivalentthm}
			\item $\mu = \nu$.
			\item Für alle lipschitzstetigen Funktionen $f \in \Bdcontfct{X}$ ist
			$\measureint{}{f}{\mu} = \measureint{}{f}{\nu}$.
			\item Für alle abgeschlossenen Mengen $C \in \mathcal{B}$ ist $\mu(C) = \nu(C)$.
		\end{equivalentthm}
	\end{Satz}
	
	\begin{proof}
		Die Implikation (i) $\Rightarrow$ (ii) ist klar und (iii) 
		$\Rightarrow$ (i) folgt aus Satz~\ref{thm:weakregularity}.
		
		(ii) $\Rightarrow$ (iii): Gelte (ii) und sei $C \subseteq X$ abgeschlossen. 
		Dann gilt für die lipschitzstetigen Funktionen $f_n \in \Bdcontfct{X}$ aus Hilfssatz~\ref{lem:opensets}
		$$\measureint{}{f_n}{\mu} = \measureint{}{f_n}{\nu}\text{,} \quad n \in \N \text{.}$$
		Wegen $| f_n | \leq 1$ und $f_n \convdown \indfct_C$ folgt mit dem Satz von Lebesgue 
		$$\measureint{}{f_n}{\mu} \; \to \; \mu(C) \quad \text{und} \quad \measureint{}{f_n}{\nu} 
		\; \to \; \nu(C)$$
		und damit gilt (iii).
	\end{proof}

	\begin{Bemerkung}
		Insbesondere folgt aus Satz~\ref{thm:measureequality} unmittelbar, dass für einen 
		metrischen Raum $(X, d)$ der Raum aller endlichen Maße $\Finitemeasures{X}$, 
		ausgestattet mit der Topologie der schwachen Konvergenz, Hausdorffsch ist. 
		Denn für $\mu \neq \nu \in \Finitemeasures{X}$ gibt es nach Satz~\ref{thm:measureequality} (ii) 
		ein $f \in C_b(X)$ mit $\measureint{}{f}{\mu} \neq \measureint{}{f}{\nu}$. 
		Nun lassen sich disjunkte Umgebungen von $\measureint{}{f}{\mu}$ und $\measureint{}{f}{\nu} \in \R$ 
		finden, deren Urbilder bezüglich $I_f$ disjunkte Umgebungen von $\mu$ und $\nu$ 
		in $\Finitemeasures{X}$ sind.
	\end{Bemerkung}

	\begin{Bemerkung}[Schwach-$\ast$-Konvergenz]
		Ist $X$ ein topologischer Raum, so ist $(C_b(X), \norm{\cdot}_\infty)$ ein Banachraum. 
		Wir betrachten nun die Abbildung
		\[\fctmap{\Phi}{\Finitemeasures{X}}{C_b(X)^\ast}{\mu}
			{\left[\fctmap{l_\mu}{\Bdcontfct{X}}{\R}{f}{\measureint{}{f}{\mu}}\right]} \text{,} \label{3.5} \tag{3.5}\]
		die wegen $\abs{\measureint{}{f}{\mu}} \leq \mu(X) \norm{f}_\infty$ und $\mu(X) < \infty$ wohldefiniert ist 
		und einem endlichen Maß $\mu \in \Finitemeasures{X}$ ein zugehöriges lineares Funktional aus $C_b(X)^\ast$ zuordnet. 
		Sind $\Finitemeasures{X}$ bzw. $\Bdcontfct{X}^\ast$ nun mit der Topologie der schwachen Konvergenz bzw. mit der 
		Schwach-$\ast$-Topologie ausgestattet, so ist schwache Konvergenz in $\Finitemeasures{X}$ äquivalent 
		zur Schwach-$\ast$-Konvergenz der Bilder unter $\Phi$.
		
		Handelt es sich bei $X$ nun um einen metrischen Raum, so ist $\Phi$ sogar injektiv und damit ein 
		Folgen-Homöomorphismus auf sein Bild: Satz~\ref{thm:measureequality} (ii) ermöglicht es uns, 
		für $\mu, \nu \in \Finitemeasures{X}$ aus $l_\mu = l_\nu$ die Gleichheit $\mu = \nu$ zu schließen. 
		
		Unter etwas spezielleren Voraussetzungen können nun mittels dieser Identifikation Aussagen aus der 
		Funktionalanalysis auch für $\Finitemeasures{X}$ fruchtbar gemacht werden, indem von topologischen Eigenschaften von
		$\Bdcontfct{X}^\ast$ mit der Schwach-$\ast$-Konvergenz auf die entsprechenden Eigenschaften von 
		$\Finitemeasures{X}$ mit der Topologie der schwachen Konvergenz geschlossen wird. 
		Ein konkretes Beispiel hierfür werden wir noch im Beweis von Satz~\ref{thm:compactequivalence} sehen.
	\end{Bemerkung}
	
\end{document}