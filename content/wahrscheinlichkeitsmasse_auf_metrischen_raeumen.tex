\documentclass[../main/main.tex]{subfiles}

\begin{document}
	
	\section{Maße auf metrischen Räumen}
	
	Wir müssen zunächst noch ein paar wenige Begriffe definieren.
	Sei $X$ ein topologischer Raum mit Topologie $\mathcal{O}$. Dann definieren wir die
	\emph{Borel-$\sigma$-Algebra} 
	$$\mathcal{B}(X) = \mathcal{B} \defby \sigma(\mathcal{O}) \text{.}$$
	Maße auf $\mathcal{B}(X)$ nennen wir \emph{Borel-Maße} und wir führen außerdem die Schreibweisen
	\begin{align*}
		\Finitemeasures{X}  \; &\defby \; \setcomp{\fct{\mu}{\mathcal{B}(X)}{[0,\infty)}}{\mu \; 
			\text{ist endliches Maß}} \\
		\Probmeasures{X} \; &\defby \; \setcomp{\fct{\mu}{\mathcal{B}(X)}{[0,1]}}{\mu \; 
			\text{ist Wahrscheinlichkeitsmaß}}
	\end{align*}
	für die Menge aller \emph{endlichen Borel-Maße} beziehungsweise \emph{Borel-Wahrscheinlichkeitsmaße} auf $X$ ein.
	
	Ferner bezeichnen wir mit $\Bdcontfct{X}$ die Menge aller stetigen und beschränkten Funktionen von $X$ nach $\R$. 
	Ist $X$ ein metrischer Raum, so schreiben wir $\Bduniffct{X}$ für die Menge aller gleichmäßig stetigen und beschränkten Funktionen
	von $X$ nach $\R$. Offensichtlich ist dann $\Bduniffct{X} \subseteq \Bdcontfct{X}$, allerdings hängt $\Bduniffct{X}$ im Gegensatz 
	zu $\Bdcontfct{X}$ von der Metrik selbst und nicht ausschließlich von der Topologie ab.
	
	
	(Schwache) Regularität ist eine 
	Approximationseigenschaft von Maßen, die es uns etwa erlaubt, gewisse Aussagen 
	zunächst für abgeschlossene bzw. kompakte und für offene Mengen zu zeigen, um diese 
	anschließend auf ganz $\mathcal{B}(X)$ auszuweiten. 
	
	\begin{Definition}[Schwache Regularität von Maßen]
		\label{def:regularity}
		Sei $X$ ein topologischer Raum und $\mu$ ein endliches Borel-Maß auf $X$. 
		Dann nennen wir $B \in \mathcal{B}(X)$ \emph{schwach von 
			innen bzw. von außen regulär}, falls
		$$\mu(B) = \sup_{\substack{C \subseteq B \\ C \; \text{abgeschlossen}}} \mu(C) 
		\quad \text{bzw.} \quad \mu(B) = \inf_{\substack{U \supseteq B \\ U \; \text{offen}}} 
		\mu(U)$$
		gelten. Wir nennen $B \in \mathcal{B}(X)$ \emph{schwach regulär}, wenn $B$ schwach von innen und 
		von außen regulär ist. Sind alle $B \in \mathcal{B}(X)$ schwach regulär, 
		so nennen wir $\mu$ ein \emph{schwach reguläres Maß}.
	\end{Definition}

	\begin{Bemerkung}
		Ersetzen wir in der obigen Definition \enquote{abgeschlossen} durch \enquote{kompakt}, 
		so erhalten wir analog den Begriff der \emph{Regularität}. 
		
		Offenbar ist eine abgeschlossene (bzw. offene) Menge schwach regulär von innen (bzw. außen).
	\end{Bemerkung}
	
	Im Falle von endlichen Maßen auf metrischen Räumen kann schwache Regularität 
	recht leicht gezeigt werden, wie wir im Folgenden sehen werden.
	
	\begin{Satz}
		\label{thm:weakregularity}
		Ist $(X, d)$ ein metrischer Raum, so ist jedes endliche Borel-Maß $\mu$ auf $X$ schwach regulär.
	\end{Satz}
	
	Für den Beweis des Satzes benötigen wir noch einen Hilfssatz, welcher 
	uns ermöglichen wird, die schwache Regularität nur auf einem Erzeuger von $\mathcal{B}(X)$ 
	zu zeigen (für den wir dann die abgeschlossenen Mengen wählen).
	
	\begin{Hilfssatz}
		\label{lem:sigmaalg}
		In der Situation von Definition~\ref{def:regularity} ist
		$$\mathcal{S} \defby \setcomp{B \in \mathcal{B}(X)}{B \text{ schwach regulär bzgl. } \mu}$$
		eine $\sigma$-Algebra.
	\end{Hilfssatz}
	
	\begin{proof}
		Offensichtlich liegen $\emptyset$ und $X$ in $\mathcal{S}$. Sei 
		$B \in \mathcal{S}$ und damit schwach regulär 
		von innen und von außen. Wegen $\mu(X) < \infty$ gilt dann
		$$\mu(B^\mathsf{c}) 
		\, = \, \mu(X) - \mu(B) 
		\, = \, \mu(X) - \sup_{\substack{C \subseteq B \\ C^\mathsf{c} \in \mathcal{O}}} \mu(C) 
		\, = \, \inf_{\substack{C \subseteq B \\ C^\mathsf{c} \in \mathcal{O}}} (\mu(X) - \mu(C))
		\, = \, \inf_{\substack{U \supseteq B^\mathsf{c} \\ U \in \mathcal{O}}} \mu(U)$$
		sowie analog
		$$\mu(B^\mathsf{c}) 
		\, = \, \mu(X) - \mu(B) 
		\, = \, \mu(X) - \inf_{\substack{U \supseteq B \\ U \in \mathcal{O}}} \mu(C)
		\, = \, \sup_{\substack{U \supseteq B \\ U \in \mathcal{O}}} (\mu(X) - \mu(U))
		\, = \, \sup_{\substack{C \subseteq B^\mathsf{c} \\ C^\mathsf{c} \in 
				\mathcal{O}}} \mu(U) \text{.}$$
		Also ist auch $B^\mathsf{c}$ schwach regulär. 
		
		Es bleibt nun zu zeigen, dass für $(B_n)_n \in \mathcal{S}^\N$ auch $B \defby 
		\bigcup_{n \in \N} B_n$ schwach regulär ist. 
		Hierfür beweisen wir zunächst die schwache Regularität von innen. 
		Sei dazu $\varepsilon > 0$. Für $n \in \N$ gibt es jeweils abgeschlossene Mengen 
		$C_n \subseteq B_n$ mit $\mu(B_n) - \mu(C_n) < \frac{\varepsilon}{3^n}$.
		Wir wählen nun $N$ so groß, dass $\mu\left( B \setminus \bigcup_{n=1}^N B_n \right) 
		< \frac{\varepsilon}{2}$ ist (was wegen $\mu(B) < \infty$ immer geht). 
		Für die abgeschlossene Menge $C := \bigcup_{n=1}^N C_k$ gilt dann die Ungleichung 
		\begin{align*}
			\mu(B) - \mu(C) = \mu(B\setminus C) \; &=
			\; \mu\left( \left( B \setminus \bigcup_{n=1}^N B_n \right) \; \cup \; 
			\left( \bigcup_{n=1}^N B_n  \setminus C \right) \right) \\
			&\leq \; \mu \left( B \setminus \bigcup_{n=1}^N B_n \right) + 
			\sum_{n=1}^{N} \mu(B_n \setminus C_n) \\
			&<    \; \frac{\varepsilon}{2} + 
			\sum_{n=1}^{\infty} \frac{\varepsilon}{3^n} \; = \; \varepsilon \text{,}
		\end{align*}
		also ist $B$ schwach regulär von innen.
		
		Ferner existieren für alle $n$ offene Mengen $U_n \supseteq B_n$ 
		mit $\mu(U_n) - \mu(B_n) < \frac{\varepsilon}{2^n}$. Wir setzen 
		$U := \bigcup_{n \in \N} U_n$ und berechnen  
		$$\mu(U) - \mu(B) \; \leq \; \sum_{n=1}^\infty \mu(U_n \setminus B) \; \leq \; 
		\sum_{n=1}^\infty \mu(U_n \setminus B_n) \; < \; \varepsilon \text{.}$$
		Weil $U$ offen ist, folgt insgesamt die schwache Regularität von $\mu$.
	\end{proof}
	
	\begin{Bemerkung}
		Sofern $X$ kompakt ist, bleibt der obige Hilfssatz~\ref{lem:sigmaalg} 
		gültig, wenn man \enquote{schwach regulär} durch \enquote{regulär} 
		ersetzt. Der hier vorgestellte Beweis ist eine Anpassung von 
		\cite[Lemma 4.5.5]{Simon.2015}, wo die Aussage für kompakte $X$ bewiesen wird.
	\end{Bemerkung}
	
	Ausgestattet mit den Hilfssätzen \ref{lem:opensets} und \ref{lem:sigmaalg} kann nun, 
	wie oben bereits angedeutet wurde, Satz~\ref{thm:weakregularity} bewiesen werden.
	
	\begin{proof}[Beweis von Satz~\ref{thm:weakregularity}]
		Es ist nun zu zeigen, dass für jedes endliche Borel-Maß $\mu$ auf $X$
		die Menge 
		$$\mathcal{S} \defby \setcomp{B \in \mathcal{B}(X)}{B \text{ schwach regulär bzgl. } \mu}$$
		bereits ganz $\mathcal{B}$ ist. 
		Da $\mathcal{S}$ nach Hilfssatz~\ref{lem:sigmaalg} eine 
		$\sigma$-Algebra ist und 
		$\mathcal{B}(X)$ von den abgeschlossenen Mengen erzeugt wird, genügt es zu zeigen, 
		dass diese in $\mathcal{S}$ enthalten sind. 
		
		Sei $C \in \mathcal{B}(X)$. Dann ist $C$ sicherlich schwach regulär von innen. 
		Nun verwenden wir die offenen Mengen $A_n, \; n \in \N$ aus 
		Hilfssatz~\ref{lem:opensets}. 
		Wegen $A_n \convdown C$ und $\mu(X) < \infty$ folgt mit der 
		Maßstetigkeit von oben die Konvergenz $\mu(A_n) \convdown \mu(C)$,
		sodass $C$ auch schwach regulär von außen ist.
	\end{proof}

	Wir möchten kurz eine direkte Folgerung aus Satz~\ref{thm:weakregularity} vorstellen, die etwas 
	schwächere hinreichende Bedingungen für die Gleichheit 
	zweier endlicher Maße auf metrischen Räumen bereitstellt. Im weiteren Verlauf 
	wird sich dies noch als nützlich erweisen.
	
	\begin{Satz}
		\label{thm:measureequality}
		Sei $(X,d)$ ein metrischer Raum und seien $\mu, \nu \in \Finitemeasures{X}$. 
		Dann sind die folgenden Aussagen äquivalent:
		\begin{equivalentthm}
			\item $\mu = \nu$.
			\item Für alle gleichmäßig stetigen Funktionen $f \in \Bdcontfct{X}$ ist
			$\measureint{}{f}{\mu} = \measureint{}{f}{\nu}$.
			\item Für alle abgeschlossenen Mengen $C \in \mathcal{B}$ ist $\mu(C) = \nu(C)$.
		\end{equivalentthm}
	\end{Satz}
	
	\begin{proof}
		Die Implikation (i) $\Rightarrow$ (ii) ist klar und (iii) 
		$\Rightarrow$ (i) folgt aus Satz~\ref{thm:weakregularity}.
		
		(ii) $\Rightarrow$ (iii): Gelte (ii) und sei $C \subseteq X$ abgeschlossen. 
		Dann gilt für die gleichmäßig stetigen Funktionen $f_n \in \Bdcontfct{X}$ aus Hilfssatz~\ref{lem:opensets}
		$$\measureint{}{f_n}{\mu} = \measureint{}{f_n}{\nu}\text{,} \quad n \in \N \text{.}$$
		Wegen $| f_n | \leq 1$ und $f_n \convdown \indfct_C$ folgt mit dem Satz von Lebesgue 
		$$\measureint{}{f_n}{\mu} \; \to \; \mu(C) \quad \text{und} \quad \measureint{}{f_n}{\nu} 
		\; \to \; \nu(C)$$
		und damit gilt (iii).
	\end{proof}
	
\end{document}