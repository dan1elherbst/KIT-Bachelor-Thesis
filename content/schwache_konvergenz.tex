\documentclass[../main/main.tex]{subfiles}

\begin{document}
	
	\section{Schwache Konvergenz von Maßen}
	
	In diesem Kapitel werden wir endliche Borel-Maße auf topologischen Räumen selbst
	mit einer Topologie bzw. einem Konvergenzbegriff versehen, dem der sogenannten \emph{schwachen Konvergenz}, 
	welche die aus der Stochastik bekannte \emph{Verteilungskonvergenz} auf allgemeinere Räume als $\R$ verallgemeinert. 
	Erst die Einschränkung auf die Klasse der polnischen Räume als Grundraum wird uns einige nichttriviale Erkenntnisse 
	über die Topologie dieses Raumes der Wahrscheinlichkeitsmaße ermöglichen.
	
	Ist $X$ ein topologischer Raum, so möchten wir im Folgenden die Menge aller stetigen beschränkten Funktionen von $X$ 
	nach $\R$ mit $\Bdcontfct{X}$ bezeichnen.
	
	\subsection{Netze}
	
	In allgemeinen topologischen Räumen beschreiben Folgen und deren Konvergenz die Eigenschaften eines Raumes nur unzureichend:
	Beispielsweise ist Folgenstetigkeit nicht notwendigerweise äquivalent zu Stetigkeit und zwei unterschiedliche Topologien auf einer 
	gegebenen Menge können generell dieselben konvergenten Folgen haben. Um solche Probleme im Folgenden zu umgehen, führen wir das
	Konzept des Netzes aus der mengentheoretischen Topologie ein. Netze verallgemeinern Folgen und können in vielerlei Hinsicht die 
	Eigenschaften eines topologischen Raumes besser erfassen.
	
	\begin{Definition}[Gerichtete Menge]
		Sei $I$ eine Menge. Eine Relation $\preceq$ auf $I$ mit
		\begin{enumeratethm}
			\item $\forall \iota \in I: \; \iota \preceq \iota$
			\item $\forall \iota, \kappa, \lambda \in I: \; \iota \preceq \kappa$ und $\kappa \preceq \lambda$ impliziert $\iota \preceq \lambda$
			\item $\forall \iota, \kappa \in I: \; \exists \lambda \in I: \; \iota \preceq \lambda$ und $\kappa \preceq \lambda$
		\end{enumeratethm}
		heißt \emph{gerichtet}. In diesem Fall nennen wir das Paar $(I, \preceq)$ auch eine \emph{gerichtete Menge}.
	\end{Definition}
	
	\begin{Definition}[Netz]
		Sei $X$ eine Menge und $(I, \preceq)$ eine gerichtete Menge. Ein \emph{Netz} in $X$ ist dann eine Abbildung $\fct{x}{I}{X}$. In diesem
		Fall schreiben wir auch $x = (x_\iota)_{\iota \in I}$.
	\end{Definition}

	\begin{Definition}[Konvergenz von Netzen]
		Sei $X$ ein topologischer Raum und sei $x = (x_\iota)_{\iota \in I}$ ein Netz in $X$. Wir sagen, dass $x$ gegen $z \in X$ konvergiert, falls
		es für jede Umgebung $U$ von $z$ ein derartiges $\iota_0 \in I$ gibt, dass $x_\iota \in U$ für alle $\iota_0 \preceq \iota \in I$ erfüllt ist.
		In diesem Fall schreiben wir auch $\lim_{\iota \in I} x_\iota = z$ oder $x_\iota \to z$.
	\end{Definition}

	\begin{Bemerkung}
		Offenbar sind Folgen einfach Netze, bei denen $(I, \preceq) \defby (\N, \leq)$ gewählt wird. Der Konvergenzbegriff von Netzen stimmt
		dann auch mit dem von uns bekannten Konvergenzbegriff in topologischen Räumen überein.
	\end{Bemerkung}

	Wie bereits angekündigt lassen sich topologische Räume im Allgemeinen viel besser über Netze charakterisieren als lediglich über Folgen. Der nächste Satz
	konkretisiert dies.
	
	\begin{Satz}
		\label{thm:netconvergence}
		Seien $X$ und $Y$ topologische Räume. Dann gelten die folgenden Aussagen:
		\begin{enumeratethm}
			\item Eine Teilmenge $C \subseteq X$ ist genau dann abgeschlossen, wenn der Grenzwert $z \in X$ eines jeden konvergenten Netzes 
			$x = (x_\iota)_{\iota \in I}$ in $X$ bereits in $A$ liegt.
			\item Eine Funktion $\fct{f}{X}{Y}$ ist genau dann stetig, wenn für jedes konvergente Netz $x = (x_\iota)_{\iota \in I}$ in $X$ 
			\[ \lim_{\iota \in I} f(x_\iota) \; = \; f(\lim_{\iota \in I} x_\iota) \]
			erfüllt ist.
			\item Zwei Topologien auf $X$ sind genau dann gleich, wenn alle Netze in $X$ bezüglich beiden Topologien dasselbe Konvergenzverhalten aufweisen.
		\end{enumeratethm}
	\end{Satz}

	\begin{proof}
		Ohne Beweis.
	\end{proof}

	\begin{Bemerkung}
		Es sei an dieser Stelle angemerkt, dass im Allgemeinen keine der drei Aussagen gelten, wenn man \enquote{Netz} durch \enquote{Folge} ersetzt.
	\end{Bemerkung}

	\subsection{Definition der schwachen Topologie}
	
	\begin{Definition}[Schwache Topologie]
		\label{def:weaktopology}
		Sei $X$ ein topologischer Raum. Dann ist die \emph{schwache Topologie} auf $\Finitemeasures{X}$ definiert als die 
		Initialtopologie bezüglich der Abbildungen
		\[ \fctmap{I_f}{\Finitemeasures{X}}{\R}{\mu}{\measureint{}{f}{\mu}}, \quad f \in \Bdcontfct{X} \text{.} \]
	\end{Definition}

	Falls nicht explizit angegeben, sei $\Finitemeasures{X}$ im Folgenden immer mit der schwachen Topologie versehen.

	\begin{Definition}[Schwache Konvergenz von Maßen]
		In der Situation von Definition~\ref{def:weaktopology} heißt ein Netz $(\mu_\iota)_{\iota \in I}$ von Maßen in $\Finitemeasures{X}$ genau dann 
		\emph{schwach konvergent} gegen ein Maß $\mu \in \Finitemeasures{X}$, wenn $\lim_{\iota \in I} \mu_\iota = \mu$ bezüglich der schwachen Topologie gilt.
		In diesem Fall schreiben wir auch
		\[ \mu_\iota \xrightarrow{w} \mu \text{.} \]
	\end{Definition}

	\begin{Hilfssatz}
		\label{lem:generalweakconvergence}
		Sei $X$ ein topologischer Raum. Ein Netz $(\mu_\iota)_{\iota \in I}$ in $\Finitemeasures{X}$ konvergiert genau dann schwach gegen $\mu \in \Finitemeasures{X}$, wenn
		\[ \lim_{\iota \in I} \measureint{}{f}{\mu_\iota} \; = \; \measureint{}{f}{\mu} \]
		für alle $f \in \Bdcontfct{X}$ gilt.
	\end{Hilfssatz}

	\begin{proof}
		Dies ist eine grundlegende Eigenschaft der Initialtopologie.
	\end{proof}

	\begin{Bemerkung}
		Häufig werden wir nur Teilräume von $\Finitemeasures{X}$ wie etwa $\Probmeasures{X}$ betrachten. Mit der schwachen Topologie auf solchen Teilräumen bezeichnen wir 
		dann einfach die Teilraumtopologie. Damit behält Hilfssatz~\ref{lem:generalweakconvergence} offenbar auch seine Gültigkeit, wenn man $\Finitemeasures{X}$ durch den 
		entsprechenden Teilraum ersetzt.
	\end{Bemerkung}

	\subsection{Beziehung zur Schwach-*-Konvergenz}
	
	Angesichts des Namens der \emph{schwachen Konvergenz} stellt sich unmittelbar die Frage, ob diese in irgendeiner Beziehung zur schwachen Konvergenz in Banachräumen aus der 
	Funktionalanalysis steht. Tatsächlich lässt sich dies zumindest bedingt bejahen, wie wir im Folgenden sehen werden. 
	
	Sei dazu zunächst $X$ ein topologischer Raum. Dann ist $(\Bdcontfct{X}, \norm{\cdot}_\infty)$ ein Banachraum und wir betrachten nun die Abbildung
	\[\fctmap{\Phi}{\Finitemeasures{X}}{C_b(X)^\ast}{\mu}
	{\left[\fctmap{l_\mu}{\Bdcontfct{X}}{\R}{f}{\measureint{}{f}{\mu}}\right]} \text{,} \label{5.1} \tag{5.1}\]
	die wegen $\abs{\measureint{}{f}{\mu}} \leq \mu(X) \norm{f}_\infty$ und $\mu(X) < \infty$ wohldefiniert ist 
	und einem endlichen Maß $\mu \in \Finitemeasures{X}$ ein zugehöriges lineares Funktional aus $C_b(X)^\ast$ zuordnet. 
	Ist $\Bdcontfct{X}^\ast$ mit der 
	Schwach-$\ast$-Topologie ausgestattet, so entspricht schwache Konvergenz in $\Finitemeasures{X}$ trivialerweise der 
	Schwach-$\ast$-Konvergenz der jeweiligen Bilder unter $\Phi$.
	
	Ist $X$ etwa metrisierbar, so können wir sogar noch eine stärkere Aussage formulieren:
	
	\begin{Hilfssatz}
		Sei $X$ ein metrisierbarer topologischer Raum. Dann ist die Abbildung $\Phi$ aus \eqref{5.1} ein Homöomorphismus auf sein Bild.
	\end{Hilfssatz}

	\begin{proof}
		$\Phi$ ist injektiv, denn für $\mu, \nu \in \Finitemeasures{X}$ folgt aus $l_\mu = l_\nu$ wegen Satz~\ref{thm:measureequality} die Gleichheit $\mu = \nu$. 
		Da schwache Konvergenz in $\Finitemeasures{X}$ exakt der Schwach-$\ast$-Konvergenz der jeweiligen Bilder unter $\Phi$ entspricht, liefert Satz~\ref{thm:netconvergence}
		die Behauptung.
	\end{proof}

	Unter diesen Voraussetzungen lässt sich $\Finitemeasures{X}$ also in $\Bdcontfct{X}$ einbetten. Mittels dieser Identifikation können nun Aussagen aus der 
	Funktionalanalysis auch für $\Finitemeasures{X}$ fruchtbar gemacht werden, indem von topologischen Eigenschaften von
	$\Bdcontfct{X}^\ast$ mit der Schwach-$\ast$-Konvergenz auf die entsprechenden Eigenschaften von 
	$\Finitemeasures{X}$ geschlossen wird. 
	Ein konkretes Beispiel hierfür können wir bereits an dieser Stelle vorführen:
	
	\begin{Satz}
		Sei $X$ ein metrisierbarer topologischer Raum. Dann ist $X$ genau dann kompakt, wenn $\Probmeasures{X}$ kompakt ist.
	\end{Satz}
	
	\begin{proof}
		Sei zunächst $X$ kompakt. Wähle eine Metrik $d$, die $X$ metrisiert.
		Zunächst setzen wir
		\[ A \; \defby \; \setcomp{l \in \Bdcontfct{X}^\ast}{\norm{l}_{\Bdcontfct{X}^\ast} \leq 1, \; l(\indfct_{X}) = 1, \; 
			\forall f \in \Bdcontfct{X} \text{ mit } f \geq 0 : l(f) \geq 0}\]
		und analog zu \eqref{5.1}
		\[ \fctmap{\Phi}{\Probmeasures{X}}{A}{\mu}{\left[\fctmap{l_\mu}{\Bdcontfct{X}}{\R}{f}{\measureint{}{f}{\mu}}\right]} \text{.} \]
		Der Darstellungssatz von Riesz-Markov liefert die Bijektivität von $\Phi$, also ist $\Phi$ ein 
		Homöomorphismus zwischen $\Probmeasures{X}$ und $A \subseteq \Bdcontfct{X}^\ast$ 
		mit der Schwach-$\ast$-Topologie. Mit dem Satz von Banach-Alaoglu folgt nun, dass $A$ als abgeschlossene Teilmenge der kompakten Menge 
		$\setcomp{l \in C_b(X)^\ast}{\norm{l}_{\Bdcontfct{X}^\ast} \leq 1}$ selbst kompakt bezüglich der Schwach-$\ast$-Topologie ist, 
		was schließlich die Kompaktheit von $\Probmeasures{X}$ nach sich zieht.
	\end{proof}

	\tobechanged{FEHLT: Rückrichtung}
	
	\subsection{Charakterisierungen schwacher Konvergenz}
	
	Im Falle metrisierbarer topologischer Räume liefert der folgende Satz eine umfassende Charakterisierung der schwachen Konvergenz.
	
	\begin{Satz}[Portmanteau]
		Sei $X$ ein metrisierbarer topologischer Raum, $(\mu_\iota)_{\iota \in I}$ ein Netz in $\Finitemeasures{X}$
		und $\mu \in \Finitemeasures{X}$. Dann sind die folgenden Aussagen äquivalent:
		\begin{equivalentthm}
			\item $\mu_\iota \xrightarrow{w} \mu$.
			\item Es ist 
			$\lim_{\iota \in I} \mu_\iota(X) = \mu(X)$
			und für alle abgeschlossenen Mengen $C \subseteq X$ gilt 
			$$\limsup_{\iota \in I} \mu_\iota(C) \; \leq \; \mu(C) \text{.}$$
			\item Es ist 
			$\lim_{\iota \in I} \mu_\iota(X) = \mu(X)$
			und für alle offenen Mengen $U \subseteq X$ gilt 
			$$\liminf_{\iota \in I} \mu_\iota(U) \; \geq \; \mu(U) \text{.}$$
			\item Für alle $B \in \mathcal{B}$ mit $\mu(\partial B) = 0$ 
			ist $$\lim_{\iota \in I} \mu_\iota(A) \; = \; \mu(A) \text{.}$$
		\end{equivalentthm}
	\end{Satz}

	\tobechanged{TODO: Limsup definieren und Beweis anpassen}
	
	\begin{proof}
		(i) $\Rightarrow$ (ii): Sei $C \subseteq X$ abgeschlossen und seien 
		$f_m, \; m \in \N$ die Funktionen aus Hilfssatz~\ref{lem:opensets}. 
		Diese sind stetig und beschränkt.
		Dann gilt für alle $m \in \N$
		$$\mu_n(C) \; = \; \measureint{}{\indfct_C}{\mu_n} \; \leq \; 
		\measureint{}{f_m}{\mu_n} \; \to \;
		\measureint{}{f_m}{\mu} \text{,} \quad n \to \infty \text{,}$$
		also 
		$$\limsup_{n \to \infty} \mu_n(C) \; \leq \; 
		\measureint{}{f_m}{\mu} \text{.}$$
		Wegen $f_m \convdown \indfct_C$ und $| f_m | \leq 1$ 
		liefert der Satz von Lebesgue die Konvergenz
		$$\measureint{}{f_m}{\mu} \; \to \;
		\measureint{}{\indfct_C}{\mu} \; = \; \mu(C) \text{,} 
		\quad m \to \infty \text{,}$$
		woraus
		$$\limsup_{n \to \infty} \mu_n(C) \; \leq \; \mu(C)$$
		folgt. Außerdem ist 
		$$\mu_n(X) \; = \; \measureint{}{\indfct_{X}}{\mu_n} \; \to \; \measureint{}{\indfct_{X}}{\mu} \; = \; \mu(X) \text{,} 
		\quad n \to \infty \text{.}$$
		
		(ii) $\Leftrightarrow$ (iii): Es gelte (ii). Sei $U$ offen, also 
		$C \defby U^\mathsf{c}$ abgeschlossen. Dann erhalten wir
		\begin{align*}
			\liminf_{n \to \infty} \mu_n(U) \; &= \; \liminf_{n \to \infty} (\mu_n(X) - \mu_n(C)) \; = \;
			\mu(X) - \limsup_{n \to \infty} \mu_n(C) \\
			&\geq \; 
			\mu(X) - \mu(C) \; = \; \mu(U)
		\end{align*}
		und damit (iii). Die andere Richtung zeigt man analog.
		
		(ii), (iii) $\Rightarrow$ (iv): Sei $A \in \mathcal{B}$ mit 
		$\mu(\partial A) = 0$. Wegen
		$A^\mathsf{o} \subseteq A \subseteq \overline{A}$ und 
		$\partial A = \overline{A} \setminus A^\mathsf{o}$ gilt $\mu(A^\mathsf{o}) = 
		\mu(A) = \mu(\overline{A}) \text{.}$
		Ferner liefern die Annahmen
		$$\limsup_{n \to \infty} \mu_n(\overline{A}) \; \leq \; 
		\mu(\overline{A}) \quad \text{und} \quad 
		\liminf_{n \to \infty} \mu_n(A^\mathsf{o}) \; \geq \; 
		\mu(A^\mathsf{o}) \text{.}$$
		Daraus folgt
		$$\limsup_{n \to \infty} \mu_n(A) \; \leq \; 
		\mu(A) \; \leq \; \liminf_{n \to \infty} \mu_n(A) \text{,}$$
		also insgesamt
		$$\lim_{n \to \infty} \mu_n(A) \; = \; \mu(A) \text{.}$$
		
		(iv) $\Rightarrow$ (i): Sei $f \in \Bdcontfct{X}$ und $a < b \in \R$ 
		mit $a < f < b$. Die Menge
		$$S \defby \setcomp{c \in (a, b)}{\mu(\set{f = c}) > 0} \text{,}$$
		ist abzählbar, da 
		$S_n \defby \setcomp{c \in (a, b)}{\mu(\set{f = c}) > \frac{1}{n}}$ 
		für jedes $n \in \N$ endlich ist und $S = \bigcup_{n \in \N} S_n$ gilt.
		Damit können wir für jedes $m \in \N$ Zahlen $c_j^{(m)} \notin S$ mit
		\[a = c_0^{(m)} < \dots < c_{2m}^{(m)} = 
		b \text{,} \qquad c_{j+1}^{(m)} - c_j^{(m)} \leq \frac{b-a}{m} 
		\label{eq:3.1} \tag{3.1}\]
		finden. Wir setzen
		$$A_j^{(m)} \defby \set{c_j^{(m)} < f \leq c_{j+1}^{(m)}}\text{,}
		\qquad j \in \set{0,\dots,2m-1} \text{.}$$ 
		Die Stetigkeit von $f$ impliziert 
		$\partial A_j^{(m)} \subseteq \set{f = c_j^{(m)}} \cup \set{f = c_{j+1}^{(m)}}$, 
		woraus sich 
		$$\mu(\partial A_j^{(m)}) = 0 \text{,} \qquad j \in \set{0,\dots,2m-1}$$
		ergibt. Für $m \in \N$ schreiben wir
		$$u_m \defby \sum_{j=0}^{2m-1} c_j^{(m)} \indfct_{A_j^{(m)}}$$
		und Aussage (iv) führt dann auf
		\[\measureint{}{u_m}{\mu_n} \; = \; \sum_{j=0}^{2m-1} c_j^{(m)} \mu_n(A_j^{(m)}) 
		\; \to \; \sum_{j=0}^{2m-1} c_j^{(m)} \mu(A_j^{(m)}) \; = \; 
		\measureint{}{u_m}{\mu} \text{,} \quad n \to \infty \text{.} 
		\label{eq:3.2} \tag{3.2}\]
		Außerdem folgen aus \eqref{eq:3.1} die Ungleichungen
		\[\left| \measureint{}{f}{\mu_n} - \measureint{}{u_m}{\mu_n} \right| \; \leq \; 
		\frac{b-a}{m} \text{,} \qquad 
		\left| \measureint{}{f}{\mu} - \measureint{}{u_m}{\mu} \right| \; \leq \; 
		\frac{b-a}{m} \label{eq:3.3} \tag{3.3}\]
		für $n \in \N$.
		Mit \eqref{eq:3.3} gilt nun für alle $m, n \in \N$
		\begin{align*}
			\left| \measureint{}{f}{\mu} - \measureint{}{f}{\mu_n} \right| \; &\leq \; 
			\left| \measureint{}{f}{\mu} - \measureint{}{u_m}{\mu} \right| + 
			\left| \measureint{}{u_m}{\mu} - \measureint{}{u_m}{\mu_n} \right| \\ & \qquad + 
			\left| \measureint{}{u_m}{\mu_n} - \measureint{}{f}{\mu_n} \right| \\
			&\leq \; 2 \cdot \frac{b-a}{m} + \left| \measureint{}{u_m}{\mu_n} - 
			\measureint{}{f}{\mu_n} \right| \label{eq:3.4} \tag{3.4}\text{.}
		\end{align*}
		\eqref{eq:3.2} und \eqref{eq:3.4} liefern also letztendlich für alle $m$
		\begin{align*}
			\limsup_{n \to \infty} \left| \measureint{}{f}{\mu} - 
			\measureint{}{f}{\mu_n} \right|
			\; &\leq \; 2 \cdot \frac{b-a}{m} + 
			\limsup_{n \to \infty} \left| \measureint{}{u_m}{\mu_n} - 
			\measureint{}{f}{\mu_n} \right| \\
			&= \; 2 \cdot \frac{b-a}{m} \; \to \; 0 \text{,} 
			\quad m \to \infty \text{,}
		\end{align*}
		was (i) impliziert.
	\end{proof}
	
\end{document}