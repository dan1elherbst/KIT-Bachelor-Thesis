\documentclass[../main/main.tex]{subfiles}

\begin{document}
	
	\section{Schwache Topologie und schwache Konvergenz}
	
	In diesem Kapitel werden wir endliche Borel-Maße auf topologischen Räumen selbst
	mit einer Topologie bzw. einem Konvergenzbegriff versehen, dem der sogenannten \emph{schwachen Konvergenz}, 
	welche die aus der Stochastik bekannte \emph{Verteilungskonvergenz} auf allgemeinere Räume als $\R$ verallgemeinert. 
	Erst die Einschränkung auf die Klasse der polnischen Räume als Grundraum wird uns einige nichttriviale Erkenntnisse 
	über die Topologie dieses Raumes der Wahrscheinlichkeitsmaße ermöglichen.
	
	Die grundlegenden Definitionen und Bemerkungen in den Abschnitten \ref{subsec:Definition} und \ref{subsec:BeziehungSchwachStern} folgen 
	\cite[Abschnitte 1-3]{Varadarajan.1958}, während wir uns bei Satz~\ref{thm:portmanteau} in Abschnitt \ref{subsec:CharakterisierungenSchwacherKonvergenz}
	an \cite[Satz 4.14.4]{Simon.2015} orientieren.

	\subsection{Definitionen}
	\label{subsec:Definition}
	
	\begin{Definition}[Schwache Topologie]
		\label{def:weaktopology}
		Sei $X$ ein topologischer Raum. Dann ist die \emph{schwache Topologie} auf $\Finitemeasures{X}$ definiert als die 
		Initialtopologie bezüglich der Abbildungen
		\[ \fctmap{I_f}{\Finitemeasures{X}}{\R}{\mu}{\measureint{}{f}{\mu}}, \quad f \in \Bdcontfct{X} \text{.} \]
	\end{Definition}

	Falls nicht explizit angegeben, sei $\Finitemeasures{X}$ im Folgenden immer mit der schwachen Topologie versehen.

	\begin{Definition}[Schwache Konvergenz von Maßen]
		In der Situation von Definition~\ref{def:weaktopology} heißt ein Netz $(\mu_\iota)_{\iota \in I}$ von Maßen in $\Finitemeasures{X}$ genau dann 
		\emph{schwach konvergent} gegen ein Maß $\mu \in \Finitemeasures{X}$, wenn $\lim_{\iota \to \infty} \mu_\iota = \mu$ bezüglich der schwachen Topologie gilt.
		In diesem Fall schreiben wir auch
		\[ \mu_\iota \xrightarrow{w} \mu \text{.} \]
	\end{Definition}

	\begin{Hilfssatz}
		\label{lem:generalweakconvergence}
		Sei $X$ ein topologischer Raum. Ein Netz $(\mu_\iota)_{\iota \in I}$ in $\Finitemeasures{X}$ konvergiert genau dann schwach gegen $\mu \in \Finitemeasures{X}$, wenn
		\[ \lim_{\iota \to \infty} \measureint{}{f}{\mu_\iota} \; = \; \measureint{}{f}{\mu} \]
		für alle $f \in \Bdcontfct{X}$ gilt.
	\end{Hilfssatz}

	\begin{proof}
		Dies folgt direkt aus Hilfssatz~\ref{lem: convergenceinitialtopology}.
	\end{proof}

	\begin{Bemerkung}
		Häufig betrachten wir nur Teilräume von $\Finitemeasures{X}$ wie etwa $\Probmeasures{X}$. Mit der schwachen Topologie auf solchen Teilräumen bezeichnen wir 
		dann einfach die entsprechende Teilraumtopologie. Damit behält Hilfssatz~\ref{lem:generalweakconvergence} offenbar auch seine Gültigkeit, wenn man $\Finitemeasures{X}$ durch den 
		jeweiligen Teilraum ersetzt.
	\end{Bemerkung}

	\subsection{Beziehung zur Schwach-*-Konvergenz}
	\label{subsec:BeziehungSchwachStern}
	
	Angesichts des Namens der \emph{schwachen Konvergenz} stellt sich unmittelbar die Frage, ob diese in irgendeiner Beziehung zur schwachen Konvergenz in Banachräumen aus der 
	Funktionalanalysis steht. Tatsächlich lässt sich dies zumindest bedingt bejahen, wie wir im Folgenden sehen werden. 
	
	Sei dazu zunächst $X$ ein topologischer Raum. Dann ist $(\Bdcontfct{X}, \norm{\cdot}_\infty)$ ein Banachraum und wir betrachten nun die Abbildung
	\[\fctmap{\Phi}{\Finitemeasures{X}}{C_b(X)^\ast}{\mu}
	{\left[\fctmap{l_\mu}{\Bdcontfct{X}}{\R}{f}{\measureint{}{f}{\mu}}\right]} \text{,} \label{3.1} \tag{3.1}\]
	die wegen $\abs{\measureint{}{f}{\mu}} \leq \mu(X) \norm{f}_\infty$ und $\mu(X) < \infty$ wohldefiniert ist 
	und einem endlichen Maß $\mu \in \Finitemeasures{X}$ ein zugehöriges lineares Funktional aus $C_b(X)^\ast$ zuordnet. 
	Ist $\Bdcontfct{X}^\ast$ mit der 
	Schwach-$\ast$-Topologie ausgestattet, so entspricht schwache Konvergenz in $\Finitemeasures{X}$ trivialerweise der 
	Schwach-$\ast$-Konvergenz der jeweiligen Bilder unter $\Phi$.
	
	Ist $X$ etwa metrisierbar, so können wir sogar noch eine stärkere Aussage formulieren:
	
	\begin{Hilfssatz}
		Sei $X$ ein metrisierbarer topologischer Raum. Dann ist die Abbildung $\Phi$ aus \eqref{3.1} ein Homöomorphismus auf sein Bild.
	\end{Hilfssatz}

	\begin{proof}
		$\Phi$ ist injektiv, denn für $\mu, \nu \in \Finitemeasures{X}$ folgt aus $l_\mu = l_\nu$ wegen Satz~\ref{thm:measureequality} die Gleichheit $\mu = \nu$. 
		Da schwache Konvergenz in $\Finitemeasures{X}$ exakt der Schwach-$\ast$-Konvergenz der jeweiligen Bilder unter $\Phi$ entspricht, liefert Satz~\ref{thm:netconvergence}
		die Behauptung.
	\end{proof}

	Unter obigen Voraussetzungen lässt sich $\Finitemeasures{X}$ also in $\Bdcontfct{X}^\ast$ einbetten. Mittels dieser Identifikation können nun Aussagen aus der 
	Funktionalanalysis auch für $\Finitemeasures{X}$ fruchtbar gemacht werden, indem von topologischen Eigenschaften von
	$\Bdcontfct{X}^\ast$ mit der Schwach-$\ast$-Konvergenz auf die entsprechenden Eigenschaften von 
	$\Finitemeasures{X}$ geschlossen wird. 
	
	Eine konkrete Anwendung hiervon werden wir schon zu Beginn des nächsten Abschnitts im Beweis von Satz~\ref{thm:compactimpliescompactmeasures} sehen.
	
	\subsection{Charakterisierungen schwacher Konvergenz}
	\label{subsec:CharakterisierungenSchwacherKonvergenz}
	
	Im Falle metrisierbarer topologischer Räume liefert der folgende Satz eine umfassende Charakterisierung der schwachen Konvergenz.
	
	\begin{Satz}[Portmanteau]
		\label{thm:portmanteau}
		Sei $(X, d)$ ein metrischer Raum, $(\mu_\iota)_{\iota \in I}$ ein Netz in $\Finitemeasures{X}$
		und $\mu \in \Finitemeasures{X}$. Dann sind die folgenden Aussagen äquivalent:
		\begin{equivalentthm}
			\item $\mu_\iota \xrightarrow{w} \mu$.
			\item Für alle $f \in \Bduniffct{X}$ gilt $\lim_{\iota \to \infty} \measureint{}{f}{\mu_\iota} = \measureint{}{f}{\mu}$.
			\item Es ist 
			$\lim_{\iota \to \infty} \mu_\iota(X) = \mu(X)$
			und für alle abgeschlossenen Mengen $C \subseteq X$ gilt 
			$$\limsup_{\iota \to \infty} \mu_\iota(C) \; \leq \; \mu(C) \text{.}$$
			\item Es ist 
			$\lim_{\iota \to \infty} \mu_\iota(X) = \mu(X)$
			und für alle offenen Mengen $U \subseteq X$ gilt 
			$$\liminf_{\iota \to \infty} \mu_\iota(U) \; \geq \; \mu(U) \text{.}$$
			\item Für alle $B \in \mathcal{B}(X)$ mit $\mu(\partial B) = 0$ 
			ist $$\lim_{\iota \to \infty} \mu_\iota(A) \; = \; \mu(A) \text{.}$$
		\end{equivalentthm}
	\end{Satz}
	
	\begin{proof}
		Wir präsentieren hier eine Verallgemeinerung des Beweises von Satz 4.14.4 in \cite[Kapitel 4.14]{Simon.2015} mit endlichen Maßen 
		anstelle Wahrscheinlichkeitsmaßen und Netzen. Die Implikation \enquote{(i) $\Rightarrow$ (ii)} folgt 
		direkt aus Hilfssatz~\ref{lem:generalweakconvergence}. 
		
		(ii) $\Rightarrow$ (iii): Sei $C \subseteq X$ abgeschlossen und seien 
		$f_m, \; m \in \N$ die Funktionen aus Hilfssatz~\ref{lem:opensets}. 
		Diese sind gleichmäßig stetig und beschränkt.
		Dann gilt für alle $m \in \N$
		$$\mu_\iota(C) \; = \; \measureint{}{\indfct_C}{\mu_\iota} \; \leq \; 
		\measureint{}{f_m}{\mu_\iota} \; \to \;
		\measureint{}{f_m}{\mu} \text{,} \quad \iota \to \infty \text{,}$$
		also 
		$$\limsup_{\iota \to \infty} \mu_\iota(C) \; \leq \; 
		\measureint{}{f_m}{\mu} \text{.}$$
		Wegen $f_m \convdown \indfct_C$ und $| f_m | \leq 1$ 
		liefert der Satz von Lebesgue die Konvergenz
		$$\measureint{}{f_m}{\mu} \; \to \;
		\measureint{}{\indfct_C}{\mu} \; = \; \mu(C) \text{,} 
		\quad m \to \infty \text{,}$$
		woraus
		$$\limsup_{\iota \to \infty} \mu_\iota(C) \; \leq \; \mu(C)$$
		folgt. Außerdem ist 
		$$\mu_\iota(X) \; = \; \measureint{}{\indfct_{X}}{\mu_\iota} \; \to \; \measureint{}{\indfct_{X}}{\mu} \; = \; \mu(X) \text{,} 
		\quad \iota \to \infty \text{.}$$
		
		(iii) $\Leftrightarrow$ (iv): Es gelte (iii). Sei $U$ offen, also 
		$C \defby U^\mathsf{c}$ abgeschlossen. Dann erhalten wir
		\begin{align*}
			\liminf_{\iota \to \infty} \mu_\iota(U) \; &= \; \liminf_{\iota \to \infty} (\mu_\iota(X) - \mu_\iota(C)) \; = \;
			\mu(X) - \limsup_{\iota \to \infty} \mu_\iota(C) \\
			&\geq \; 
			\mu(X) - \mu(C) \; = \; \mu(U)
		\end{align*}
		und damit (iv). Die andere Richtung zeigt man analog.
		
		(iii), (iv) $\Rightarrow$ (v): Sei $A \in \mathcal{B}(X)$ mit 
		$\mu(\partial A) = 0$. Wegen
		$A^\mathsf{o} \subseteq A \subseteq \overline{A}$ und 
		$\partial A = \overline{A} \setminus A^\mathsf{o}$ gilt $\mu(A^\mathsf{o}) = 
		\mu(A) = \mu(\overline{A}) \text{.}$
		Ferner liefern die Annahmen
		$$\limsup_{\iota \to \infty} \mu_\iota(\overline{A}) \; \leq \; 
		\mu(\overline{A}) \quad \text{und} \quad 
		\liminf_{\iota \to \infty} \mu_\iota(A^\mathsf{o}) \; \geq \; 
		\mu(A^\mathsf{o}) \text{.}$$
		Daraus folgt
		$$\limsup_{\iota \to \infty} \mu_\iota(A) \; \leq \; 
		\mu(A) \; \leq \; \liminf_{\iota \to \infty} \mu_\iota(A) \text{,}$$
		also insgesamt
		$$\lim_{\iota \to \infty} \mu_\iota(A) \; = \; \mu(A) \text{.}$$
		
		(v) $\Rightarrow$ (i): Sei $f \in \Bdcontfct{X}$ und $a < b \in \R$ 
		mit $a < f < b$. Die Menge
		$$S \defby \setcomp{c \in (a, b)}{\mu(\set{f = c}) > 0} \text{,}$$
		ist abzählbar, da 
		$S_n \defby \setcomp{c \in (a, b)}{\mu(\set{f = c}) > \frac{1}{n}}$ 
		für jedes $n \in \N$ endlich ist und $S = \bigcup_{n \in \N} S_n$ gilt.
		Damit können wir für jedes $m \in \N$ Zahlen $c_j^{(m)} \notin S$ mit
		\[a = c_0^{(m)} < \dots < c_{2m}^{(m)} = 
		b \text{,} \qquad c_{j+1}^{(m)} - c_j^{(m)} \leq \frac{b-a}{m} 
		\label{eq:3.2} \tag{3.2}\]
		finden. Wir setzen
		$$A_j^{(m)} \defby \set{c_j^{(m)} < f \leq c_{j+1}^{(m)}}\text{,}
		\qquad j \in \set{0,\dots,2m-1} \text{.}$$ 
		Die Stetigkeit von $f$ impliziert 
		$\partial A_j^{(m)} \subseteq \set{f = c_j^{(m)}} \cup \set{f = c_{j+1}^{(m)}}$, 
		woraus sich 
		$$\mu(\partial A_j^{(m)}) = 0 \text{,} \qquad j \in \set{0,\dots,2m-1}$$
		ergibt. Für $m \in \N$ schreiben wir
		$$u_m \defby \sum_{j=0}^{2m-1} c_j^{(m)} \indfct_{A_j^{(m)}}$$
		und Aussage (v) führt dann auf
		\[\measureint{}{u_m}{\mu_\iota} \; = \; \sum_{j=0}^{2m-1} c_j^{(m)} \mu_\iota(A_j^{(m)}) 
		\; \to \; \sum_{j=0}^{2m-1} c_j^{(m)} \mu(A_j^{(m)}) \; = \; 
		\measureint{}{u_m}{\mu} \text{,} \quad \iota \to \infty \text{.} 
		\label{eq:3.3} \tag{3.3}\]
		Außerdem folgen aus \eqref{eq:3.2} die Ungleichungen
		\[\left| \measureint{}{f}{\mu_\iota} - \measureint{}{u_m}{\mu_\iota} \right| \; \leq \; 
		\frac{b-a}{m} \text{,} \qquad 
		\left| \measureint{}{f}{\mu} - \measureint{}{u_m}{\mu} \right| \; \leq \; 
		\frac{b-a}{m} \label{eq:3.4} \tag{3.4}\]
		für $\iota \in I$.
		Mit \eqref{eq:3.4} gilt nun für alle $m \in \N, \; \iota \in I$
		\begin{align*}
			\left| \measureint{}{f}{\mu} - \measureint{}{f}{\mu_\iota} \right| \; &\leq \; 
			\left| \measureint{}{f}{\mu} - \measureint{}{u_m}{\mu} \right| + 
			\left| \measureint{}{u_m}{\mu} - \measureint{}{u_m}{\mu_\iota} \right| \\ & \qquad + 
			\left| \measureint{}{u_m}{\mu_\iota} - \measureint{}{f}{\mu_\iota} \right| \\
			&\leq \; 2 \cdot \frac{b-a}{m} + \left| \measureint{}{u_m}{\mu_\iota} - 
			\measureint{}{u_m}{\mu} \right| \label{eq:3.5} \tag{3.5}\text{.}
		\end{align*}
		\eqref{eq:3.3} und \eqref{eq:3.5} liefern also letztendlich für alle $m$
		\begin{align*}
			\limsup_{\iota \to \infty} \left| \measureint{}{f}{\mu} - 
			\measureint{}{f}{\mu_\iota} \right|
			\; &\leq \; 2 \cdot \frac{b-a}{m} + 
			\limsup_{\iota \to \infty} \left| \measureint{}{u_m}{\mu_\iota} - 
			\measureint{}{u_m}{\mu} \right| \\
			&= \; 2 \cdot \frac{b-a}{m} \; \to \; 0 \text{,} 
			\quad m \to \infty \text{,}
		\end{align*}
		was wegen Hilfssatz~\ref{lem:generalweakconvergence} die schwache Konvergenz $\mu_\iota \xrightarrow{w} \mu$ impliziert.
	\end{proof}
	
\end{document}