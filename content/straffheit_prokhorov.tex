\documentclass[../main/main.tex]{subfiles}

\begin{document}
	
	\section{Straffheit und der Satz von Prokhorov}
	
	Das zentrale Resultat dieses Abschnitts stellt der \emph{Satz von Prokhorov} (Satz~\ref{thm:prokhorov}) dar, der für einen polnischen Raum $X$ eine einfache Charakterisierung der Kompaktheit von Teilmengen von $\Probmeasures{X}$ bereitstellt. Hierfür werden wir zunächst das Konzept der \emph{Straffheit} einführen.
	
	\begin{Definition}
		Sei $X$ ein polnischer Raum und $\mu \in \Finitemeasures{X}$. Dann nennen wir 
		eine Teilmenge $A \subseteq \Finitemeasures{X}$ \emph{straff}, falls es für jedes $\varepsilon > 0$ eine kompakte Teilmenge
		$K_\varepsilon \subseteq X$ gibt mit 
		\[ \mu(K_\varepsilon) \geq \mu(X)  - \varepsilon \]
		für alle $\mu \in A$.
		
		Ferner heißt ein Maß $\mu \in \Finitemeasures{X}$ \emph{straff}, falls $\set{\mu}$ straff ist.
	\end{Definition}

	Im folgenden Satz werden wir sehen, dass wir es tatsächlich sehr häufig mit straffen Maßen zu tun haben.
	
	\begin{Satz}
		\label{thm:tightness}
		Sei $X$ ein polnischer Raum und $\mu \in \Finitemeasures{X}$. Dann ist $\mu$ straff.
	\end{Satz}
	
	\begin{proof}
		Sei $d$ eine Metrik, bezüglich der $(X, d)$ vollständig ist und sei 
		$\mathcal{D} \defby \setcomp{x_m}{m \in \N}$ eine abzählbare dichte Teilmenge von $X$. 
		Für $k \in \N$ gilt also $\bigcup_{m \in \N} B_{1/k}(x_m) = X$ und Maßstetigkeit 
		von unten impliziert 
		$\lim_{M \to \infty} \bigcup_{m=1}^{M} B_{1/k}(x_m) = \mu(X)$.
		
		Sei nun $\varepsilon > 0$. Dann finden wir natürliche Zahlen $M_1 \leq M_2 \leq \dots$ mit
		\[ \mu\left( \bigcup_{m=1}^{M_k} B_{1/k}(x_m) \right) \; \geq \; \mu(X) - \frac{\varepsilon}{2^k} \]
		für alle $k \in \N$. Wir setzen nun
		\[ S_\varepsilon 
		\; \defby \; \bigcap_{k \in \N} \left( \bigcup_{m=1}^{M_k} B_{1/k}(x_m) \right) 
		\quad \text{und} \quad K_\varepsilon \defby \overline{S_\varepsilon} \text{.} \]
		Offenbar ist $S_\varepsilon$ totalbeschränkt und damit ist $K_\varepsilon = \overline{S_\varepsilon}$ 
		kompakt (dies wird etwa in \cite[Satz 2.3.8]{Simon.2015} bewiesen, 
		hier geht die Vollständigkeit von $(X, d)$ ein).
		Außerdem berechnen wir
		\begin{align*}
			\mu(K_\varepsilon) \; &\geq \; \mu(S_\varepsilon) 
			\; \geq \; \mu(X) - \sum_{k=1}^{\infty} \left( \mu(X) - \mu\left( \bigcup_{m=1}^{M_k} B_{1/k}(x_m) \right) \right) \\
			\; &\geq \; \mu(X) - \sum_{k=1}^{\infty} \frac{\varepsilon}{2^k} \; = \; \mu(X) - \varepsilon \text{.}
		\end{align*} 
		Weil $\varepsilon > 0$ beliebig gewählt war, ist $\mu$ also straff.
	\end{proof}
	
	Ohne große Mühe liefert uns der vorige Satz die Aussage, dass endliche Borel-Maße auf polnischen Räumen regulär sind:
	
	\begin{Folgerung}
		\label{kor:polishregular}
		Ist $X$ ein polnischer Raum, so ist jedes $\mu \in \Finitemeasures{X}$ regulär.
	\end{Folgerung}
	
	\begin{proof}
		Sei $\mu \in \Probmeasures{X}$ und $B \in \mathcal{B}(X)$. 
		Wegen \ref{thm:weakregularity} wissen wir bereits, dass 
		\[\mu(B) = \sup_{\substack{C \subseteq B \\ C \text{ abgeschlossen}}} \mu(C) 
		\quad \text{bzw.} \quad \mu(B) = \inf_{\substack{U \supseteq B \\ U \text{ offen}}} 
		\mu(U)\]
		gilt. Sei nun $\varepsilon > 0$. Dann gibt es eine abgeschlossene Teilmenge 
		$C \subseteq B \subseteq X$ mit $\mu(B \setminus C) < \varepsilon$. 
		Außerdem gibt es nach Satz~\ref{thm:tightness} eine kompakte Menge 
		$\tilde{K} \subseteq X$ mit $\mu(\tilde{K}) \geq 1 - \varepsilon$. Setzen 
		wir nun $K \defby \tilde{K} \cap C$, so ist $K$ ebenfalls kompakt und 
		$K \subseteq B$ mit 
		\[ \mu(B \setminus K) 
		\; = \; \mu((B \setminus C) \cap (B \setminus \tilde{K})) 
		\; \leq \; \mu(B \setminus C) + \mu(K^{\ast \mathsf{c}}) 
		\; \leq \; 2 \varepsilon \text{.} \]
		Daher folgt 
		\[\mu(B) 
		= \sup_{\substack{K \subseteq B \\ K \text{ kompakt}}} \mu(K)\] 
		und insgesamt $\mu$ ist regulär.
	\end{proof}

	Wir möchten nun zum zentralen Satz dieses Abschnitts gehen, der uns eine interessante Charakterisierung der Kompaktheit von Teilmengen von $\Probmeasures{X}$ liefert, sofern wir $X$ als polnisch voraussetzen.
	
	\begin{Satz}[Prokhorov]
		\label{thm:prokhorov}
		Sei $X$ ein polnischer Raum und $A \subseteq \Probmeasures{X}$, wobei $\Probmeasures{X}$ mit der 
		Topologie der schwachen Konvergenz versehen sei. Dann sind folgende Aussagen äquivalent:
		\begin{equivalentthm}
			\item $A$ ist straff.
			\item $\overline{A} \subseteq \Probmeasures{X}$ ist kompakt.
		\end{equivalentthm}
	\end{Satz}

	\begin{Bemerkung}
		Es sei an dieser Stelle angemerkt, dass die Aussage des Satzes von Prokhorov die von Satz~\ref{thm:compactimpliescompactmeasures} beinhaltet. Denn ist $X$ kompakt und metrisierbar (und damit insbesondere polnisch), so ist $\Probmeasures{X}$ trivialerweise straff und damit nach dem Satz von Prokhorov auch kompakt.
	\end{Bemerkung}
	
	\begin{proof}
		Sei zunächst $A \subseteq \Probmeasures{X}$ straff. Wir möchten nun beweisen, dass $\overline{A}$ 
		kompakt ist, was wegen der Metrisierbarkeit von $\Probmeasures{X}$ äquivalent dazu ist,
		dass jede Folge $(\mu_n)_n \in A^\N$ eine schwach konvergente Teilfolge besitzt. 
		
		Aufgrund der Straffheit von $A$ gibt es für jedes $m \in \N$ ein Kompaktum $K^{(m)} \subseteq X$, 
		das für alle $n \in \N$
		\[ \mu_n(K^{(m)}) \geq 1 - \frac{1}{m+1} \label{5.1} \tag{5.1} \]
		erfüllt. Ohne Einschränkung dürfen wir außerdem annehmen, dass 
		\[ K^{(m)} \subseteq K^{(m+1)} \label{5.2} \tag{5.2} \] 
		für alle $m \in \N$ gilt.
		
		Sei nun also $(\mu_n)_n \in A^\N$.
		In Analogie zu Satz~\ref{thm:compactimpliescompactmeasures} lässt sich nun auch einsehen, dass 
		\[ \Unitmeasures{K^{(m)}} \; \defby \; \setcomp{\fct{\mu}{\mathcal{B}(K^{(m)})}{[0,1]}}{\mu \; 
			\text{ist endliches Maß und } \mu(K^{(m)}) \leq 1} \]
		für alle $m \in \N$ kompakt bezüglich der Topologie der schwachen Konvergenz ist. Damit existiert ein Maß 
		$\nu^{(1)} \in \Unitmeasures{K^{(1)}}$ und eine Teilfolge $(\mu_{n_j}^{(1)})_k$ von $(\mu_n)_n$ mit
		\[ \restr{\mu_{n_j}^{(1)}}{K^{(1)}} \; \xrightarrow{w} \; \nu^{(1)} \text{.} \]
		Auf dieselbe Art und Weise findet man nun für alle $m\geq 2$ sukzessive Maße 
		$\nu^{(m)} \in \Unitmeasures{K^{(m)}}$ und Teilfolgen $(\mu_{n_j}^{(m)})_j$ von $(\mu_{n_j}^{(m-1)})_j$ mit
		\[ \restr{\mu_{n_j}^{(m)}}{K^{(m)}} \; \xrightarrow{w} \; \nu^{(m)}, \quad j \to \infty \text{.} \]
		Nun diagonalisieren wir und erhalten mit $\mu_{n_j} \; \defby \; \mu_{n_j}^{(j)}$ eine Teilfolge 
		$(\mu_{n_j})_j$ von $(\mu_n)_n$, die
		\[ \restr{\mu_{n_j}}{K^{(m)}} \; \xrightarrow{w} \; \nu^{(m)}, \quad j \to \infty \]
		für alle $m \in \N$ erfüllt.
		
		Außerdem können Maße $\nu^{(m)}, \; m \in \N$ via 
		$\nu^{(m)}(B) \defby \nu^{(m)}(B \cap K^{(m)}), \; B \in \mathcal{B}(X)$ einfach auf $X$ fortgesetzt werden. 
		Mit dieser Konvention können wir nun beweisen, dass die Maße
		$\nu^{(m)}, \; m \in \N$ in dem Sinne monoton wachsen, dass 
		\[ \nu^{(m)}(B) \leq \nu^{(m+1)}(B) \label{5.3} \tag{5.3} \]
		für alle $B \in \mathcal{B}(X)$ und $m \in \N$ gilt:
		
		Wegen schwacher Regularität (vgl. Satz~\ref{thm:weakregularity}) genügt es, die Ungleichung \eqref{5.3} 
		für abgeschlossene Mengen zu zeigen. Sei also $m \in \N$ fixiert, $C \subseteq X$ abgeschlossen 
		und seien $\fct{f_k}{X}{\R}, \; k \in \N$ die gleichmäßig stetigen, beschränkten Funktionen aus 
		Hilfssatz~\ref{lem:opensets}. Unter Ausnutzung von \eqref{5.2} folgt nun 
		\[ \measureint{K^{(m)}}{f_k}{\mu_{n_j}} \; \leq \; \measureint{K^{(m+1)}}{f_k}{\mu_{n_j}} \]
		für alle $k, j \in \N$. Wegen 
		\[ \measureint{K^{(m)}}{f_k}{\mu_{n_j}} \to \ \measureint{}{f_k}{\nu^{(m)}} \quad \text{und} 
		\quad \measureint{K^{(m+1)}}{f_k}{\mu_{n_j}} \to \ \measureint{}{f_k}{\nu^{(m+1)}}, \quad j \to \infty \]
		ist damit auch 
		\[ \measureint{}{f_k}{\nu^{(m)}} \; \leq \; \measureint{}{f_k}{\nu^{(m+1)}} \]
		für alle $k \in \N$, was wiederum aufgrund der Konvergenz $f_k \convdown \indfct_C$ und dem Satz von Lebesgue Ungleichung 
		\eqref{5.3} impliziert.
		
		Schlussendlich möchten wir nun nachweisen, dass durch
		\[ \fctmap{\nu}{\mathcal{B}(X)}{[0, 1]}{B}{\sup_{m \in \N} \nu^{(m)}(B)} \]
		ein Wahrscheinlichkeitsmaß auf $X$ definiert wird, gegen das $(\mu_{n_j})_j$ schwach konvergiert.
		
		Zunächst verifizieren wir, dass $\nu$ ein Wahrscheinlichkeitsmaß ist. $\nu(\emptyset) = 0$ ist klar. 
		Weil $\nu^{(m)}$ in $\Unitmeasures{K^{(m)}}$ liegt und wir \eqref{5.1} angenommen haben, gilt 
		\[\nu^{(m)}(X) = \nu^{(m)}(K^{(m)}) \in [1 - \frac{1}{m+1}, 1]\] 
		für alle $m \in \N$ und damit $\nu(X) = 1$. Für die $\sigma$-Additivität seien 
		$B_k \in \mathcal{B}(X), \; k \in \N$ paarweise disjunkt. Dann gilt
		\begin{align*}
			\nu \left( \bigcupdot_{k \in \N} B_k \right) \; &= \; \sup_{m \in \N} \; \nu^{(m)} \left( \bigcupdot_{k \in \N} B_k \right) 
			\; = \; \sup_{m \in \N} \; \sum_{k=1}^{\infty} \nu^{(m)}(B_k) \\
			&\stackrel{\eqref{5.3}}{=} \; \sum_{k=1}^{\infty} \; \sup_{m \in \N} \nu^{(m)}(B_k) 
			\; = \; \sum_{k=1}^{\infty} \nu(B_k) \text{,}
		\end{align*}
		wobei in der vorletzten Gleichung der Satz von Beppo Levi verwendet wurde. 
		Insgesamt haben wir also $\nu \in \Probmeasures{X}$ nachgewiesen.
		
		Jetzt bleibt lediglich zu zeigen dass $(\mu_{n_j})_j$ tatsächlich schwach gegen $\nu$ konvergiert. 
		Hierfür möchten wir die Charakterisierung (iii) des Portmanteau-Theorems (Satz~\ref{thm:portmanteau}) verwenden.
		Sei dazu $C \subseteq X$ abgeschlossen. Zunächst liefert ebendieser Satz für alle $m \in \N$ die 
		Abschätzung $\limsup_{j \to \infty} \mu_{n_j}(C \cap K^{(m)}) \leq \nu^{(m)}(C)$ und wegen \eqref{5.1} ist 
		zudem $\mu_{n_j}(C \cap K^{(m) \mathsf{c}}) \leq  \frac{1}{m+1}$ für alle $j, m \in \N$. Insgesamt folgt daraus 
		\begin{align*}
			\limsup_{j \to \infty} \mu_{n_j}(C) \; &=    \; \lim_{m \to \infty} \limsup_{j \to \infty} 
			\left( \mu_{n_j}(C \cap K^{(m)}) + \mu_{n_j}(C \cap K^{(m) \mathsf{c}}) \right) \\
			&\leq \; \lim_{m \to \infty} \left( \nu^{(m)}(C) + \frac{1}{m+1} \right) 
			\; = \; \nu(C) \text{,}
		\end{align*}
		was den Beweis der Hinrichtung von Satz~\ref{thm:prokhorov} abschließt.
		
		Für die Rückrichtung sei nun $\overline{A}$ kompakt. Außerdem sei $d$ eine Metrik, die $X$ vollständig metrisiert 
		und $\mathcal{D} \defby \setcomp{x_n}{n \in \N} \subseteq X$ eine abzählbare dichte Teilmenge.
		
		Nun behaupten wir, dass für alle $\delta > 0$ ein solches $M_\delta \in \N$ existiert, dass
		\[ \mu(\bigcup_{m=1}^{M_\delta} B_\delta(x_m)) \; > \; 1 - \delta \label{5.4} \tag{5.4} \]
		für alle $\mu \in A$ gilt. Denn falls es kein derartiges $M_\delta$ gibt, so lässt sich ein $\delta > 0$ finden, für das
		für alle $M \in \N$ ein $\mu_M \in A$ existiert mit
		\[ \mu_M(\bigcup_{m=1}^{M} B_\delta(x_m)) \; \leq \; 1 - \delta \text{.} \]
		Insbesondere bedeutet das natürlich auch, dass wir für alle $M \in \N$ und $N \geq M$ 
		\[ \mu_N(\bigcup_{m=1}^{M} B_\delta(x_m)) \; \leq \; 1 - \delta \]
		abschätzen dürfen. Aufgrund der Kompaktheit von $\overline{A}$ gibt es eine Teilfolge $(\mu_{N_j})_j$ von $(\mu_N)_N$, 
		die einen schwachen Grenzwert $\mu \in \Probmeasures{X}$ besitzt. Fixiere nun ein $M \in \N$. 
		Dann ist $\bigcup_{m=1}^{M} B_\delta(x_m)$
		offen und daher liefert das Portmanteau-Theorem (Satz~\ref{thm:portmanteau} (iv))
		\[ \mu(\bigcup_{m=1}^{M} B_\delta(x_m)) 
		\; \leq \; \liminf_{j \to \infty} \mu_{N_j}(\bigcup_{m=1}^{M} B_\delta(x_m)) 
		\; \leq \; 1 - \delta \text{.} \label{5.5} \tag{5.5} \]
		Wegen $\bigcup_{m=1}^{\infty} B_\delta(x_m) = X$ und Maßstetigkeit von unten 
		liefert uns \eqref{5.5} unmittelbar 
		\[ \mu(X) \; \leq \; 1 - \delta \; < \; 1 \text{,} \]
		was einen Widerspruch dazu darstellt, dass es sich bei $\mu$ um ein 
		Wahrscheinlichkeitsmaß auf $X$ handelt. Also muss unsere Behauptung in \eqref{5.4} tatsächlich gelten.
		
		Sei nun ein $\varepsilon > 0$ gegeben. Wir setzen 
		\[ S_\varepsilon 
		\; \defby \; \bigcap_{k \in \N} \bigcup_{m=1}^{M_{\varepsilon / 2^k}} 
		B_{\varepsilon / 2^k}(x_m) \]
		und verfahren nun ähnlich wie im Beweis von Satz~\ref{thm:tightness}: Offensichtlich ist $S_\varepsilon$ total beschränkt und damit auch 
		$ K_\varepsilon \defby \overline{S_\varepsilon}$ kompakt (vgl. \cite[Satz 2.3.8]{Simon.2015}). 
		Außerdem gilt für alle $\mu \in A$
		\[
		\mu(K_\varepsilon) \; \geq \; \mu(S_\varepsilon) \; = \; \mu\left(\bigcap_{k \in \N} \bigcup_{m=1}^{M_{\varepsilon / 2^k}} 
		B_{\varepsilon / 2^k}(x_m)\right)
		\; \geq \; 1 - \sum_{k=1}^{\infty} \frac{\varepsilon}{2^k} \; = \; 1 - \varepsilon
		\]
		und damit ist $A$ straff.
	\end{proof}
	
\end{document}