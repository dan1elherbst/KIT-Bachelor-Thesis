\documentclass[../thesis/thesis.tex]{subfiles}

\begin{document}
	
	\chapter{Polnische Räume}
	
	Eine Klasse topologischer Räume wird in den späteren Abschnitten dieser Arbeit eine zentrale Rolle spielen: die der polnischen Räume.
	
	\begin{Definition}[Polnischer Raum]
		Ein \emph{polnischer Raum} ist ein separabler und vollständig metrisierbarer topologischer Raum.
	\end{Definition}

	\begin{Bemerkung}
		Es sei hier anzumerken, dass Polnizität tatsächlich eine rein topologische Eigenschaft ist. Wir fordern für einen polnischen Raum $\X$ nämlich nur die Existenz einer Metrik,
		die die Topologie von $\X$ erzeugt und bezüglich der $\X$ vollständig ist, möchten uns aber die Flexibilität
		bewahren, diese Metrik nicht zu fixieren und bei Bedarf zwischen verschiedenen solchen Metriken
		zu wechseln. Es kann durchaus Metriken geben, die $\X$ zwar metrisieren, jedoch nicht vollständig.
		Beispielsweise ist $\R$ ausgestattet mit der Standardtopologie offensichtlich ein polnischer Raum,
		denn $\R$ ist separabel und die euklidische Metrik metrisiert $\R$ vollständig. 
		Allerdings ist $(0, 1)$ als Teilraum von $\R$ mit der euklidischen Metrik zwar separabel, aber nicht vollständig.
		Da $(0, 1)$ homöomorph zu $\R$ ist, ist $(0, 1)$ aber dennoch ein polnischer Raum.
	\end{Bemerkung}

	Polnische Räume sind insofern interessant, dass es sich hierbei um eine verhältnismäßig allgemeine und elementare Klasse topologischer Räume handelt, die -- wie wir in diesem Abschnitt sehen werden -- 
	etwa unter Bildung von $G_\delta$-Teilmengen oder abzählbaren Produkten abgeschlossen ist und mit der sich daher sehr leicht umgehen lässt. Dennoch erlaubt es die Klasse der polnischen Räume in vielen Bereichen, 
	nützliche und recht allgemeine Resultate zu erhalten sowie Pathologien zu vermeiden. Konkret werden wir insbesondere in Abschnitt~\ref{sec:eigenschaften_der_schwachen_topologie} Beispiele hierfür sehen.
	Dieser Abschnitt orientiert sich überwiegend an den relevanten Teilen von \cite[Kapitel 4.14]{Simon.2015}. 
	
	Zunächst widmen wir uns einer grundlegenden Abgeschlossenheitseigenschaft von polnischen Räumen.
	
	\begin{Satz}[Alexandroff]
		\label{satz:alexandroff}
		Sei $\X$ ein polnischer Raum. Dann ist $A \subseteq \X$ genau dann selbst ein polnischer Raum bezüglich der Teilraumtopologie, 
		wenn $A \subseteq \X$ eine $G_\delta$-Teilmenge ist.
	\end{Satz}
	
	\begin{proof}
		Der Beweis der Hinrichtung folgt \cite[Satz 7]{JordanBell.2014}, während der Beweis der Rückrichtung eine Anpassung des Beweises
		von Satz 4.14.6 aus \cite[Kapitel 4.14]{Simon.2015} ist.
		
		Für die Hinrichtung sei $A \subseteq \X$ eine $G_\delta$-Teilmenge und seien 
		$U_n \subseteq \X, \, n \in \N$ offene Mengen mit 
		$A = \bigcap_{n \in \N} U_n$, wobei wir außerdem 
		$C_n \defby U_n^{\mathsf{c}}$ schreiben. Ferner sei 
		$d$ eine Metrik, die $\X$ vollständig metrisiert. 
		Da $A$ als Teilmenge eines separablen metrischen Raums 
		selbst separabel ist, genügt es, die vollständige Metrisierbarkeit 
		von $A$ zu beweisen.
		
		Für $x, y \in A$ definieren wir
		\[\tilde{d}(x, y) \; \defby \; d(x, y) + \sum_{n=1}^{\infty} \min \left(
		\frac{1}{2^n}, \left| \frac{1}{d(x, C_n)} - \frac{1}{d(y, C_n)} \right|
		\right) \text{.}\]
		Offensichtlich ist $\tilde{d}$ eine Metrik auf $A$, 
		die dieselbe Topologie wie $d$ erzeugt. 
		
		Wir möchten nun zeigen, dass $(A, \tilde{d})$ vollständig ist. 
		Sei dazu $(x_k)_k \in A^\N$ eine Cauchyfolge. Wegen $d \leq \tilde{d}$ 
		ist $(x_k)_k$ dann auch bezüglich $d$ eine Cauchyfolge und die 
		Vollständigkeit von $(\X, d)$ liefert die Existenz eines Grenzwertes 
		$x \in \X$. Angenommen $x \notin A$, so gibt es ein $m \in \N$ mit 
		$x \in C_m$. Damit ist aber für alle $k \in \N$
		$$\sup_{l \geq k} \tilde{d}(x_k, x_l) \; \geq \; \sup_{l \geq k} 
		\, \sum_{n=1}^{\infty} \min \left(
		\frac{1}{2^n}, \left| \frac{1}{d(x_k, C_n)} -
		\frac{1}{d(x_l, C_n)} \right|
		\right) \; \geq \; \frac{1}{2^m} \; > \; 0 \text{,}$$
		was der Tatsache widerspricht, dass $(x_k)_k$ eine Cauchyfolge 
		bezüglich $\tilde{d}$ ist. Also muss $x \in A$ gelten. 
		Aus der Stetigkeit von $x \mapsto d(x, C_n)$ folgt mit dem 
		Satz von Lebesgue unmittelbar die Konvergenz 
		$\tilde{d}(x_k, x) \to 0$ für $k \to \infty$,
		also ist $(A, \tilde{d})$ vollständig und $A$ damit selbst ein
		polnischer Raum.
		
		Nun beweisen wir noch die Rückrichtung. Sei dazu $A \subseteq \X$ eine derartige Teilmenge, dass
		$A$ bezüglich der Teilraumtopologie selbst polnisch ist. $\tilde{d}$ metrisiere im Folgenden 
		$A$ vollständig.
		Sicherlich ist $B_{1/k}(x_m) \subseteq A$ für alle 
		$k, m \in \N$ relativ offen, also gibt es jeweils offene Mengen 
		$V_{k, m} \subseteq \X$ mit
		\[ B_{1/k}(x_m) \; = \; V_{k, m} 
		\cap A \text{.} \label{glg:3.1} \tag{3.1}\]
		Wir möchten jetzt zeigen, dass
		\[ A \; = \; \overline{A} \, \cap \, 
		\bigcap_{k \in \N} \left( \bigcup_{m \in \N} V_{k, m} \right) 
		\label{glg:3.2} \tag{3.2}\]
		gilt. 
		Weil $\overline{A}$ nach Hilfssatz~\ref{hilfssatz:offene_mengen} 
		als abgeschlossene Menge eine $G_\delta$-Menge ist, 
		folgt aus \eqref{glg:3.2} direkt die Aussage (b).
		
		Für jedes $m \in \N$ ist sicherlich $A \subseteq 
		\bigcup_{m \in \N} V_{k, m}$, weshalb \enquote{$\subseteq$} 
		in \eqref{glg:3.2} unmittelbar folgt.
		Wir nehmen nun an, dass $z$ in der rechten Seite von \eqref{glg:3.2} liegt. 
		Dann gibt es für jedes $k \in \N$ ein $m_k \in \N$ mit $z \in V_{k, m_k}$. 
		Ferner existiert wegen $z \in \overline{A}$ eine Folge $(y_k)_k \in A^\N$ 
		mit $y_k \to z$, wobei wir zusätzlich
		$y_k \in \bigcap_{j=1}^{k} V_{j, m_j}$
		fordern. Weil damit nach \eqref{glg:3.1} auch für alle $k \in \N$
		\[y_k \in \bigcap_{j=1}^{k} B_{1/k}(x_m)\]
		gilt, ist $(y_k)_k$ eine Cauchyfolge bezüglich $\tilde{d}$. Aufgrund der Vollständigkeit von 
		$(A, \tilde{d})$ gibt es also ein $y \in A$ mit $y_k \to y$, was 
		$z = y \in A$ und damit \enquote{$\supseteq$} 
		in \eqref{glg:3.2} impliziert.
	\end{proof}
	
	Mit dem \emph{Hilbertwürfel} möchten wir uns nun einen speziellen polnischen Raum ansehen, 
	der sich gleich in gewisser Weise als universell für alle polnischen Räume erweisen wird.
	
	\begin{Definition}[Hilbertwürfel]
		Der \emph{Hilbertwürfel} ist der topologische Raum $H = [0, 1]^\N$ 
		ausgestattet mit der Produkttopologie, also der Initialtopologie bezüglich 
		aller Projektionen $\fctmap{\pi_n}{H}{[0, 1]}{x}{x_n}, \; n \in \N$.
	\end{Definition}
	
	Einige fundamentale Eigenschaften des Hilbertwürfels sind im folgenden Hilfssatz zusammengefasst.
	
	\begin{Hilfssatz}
		\label{hilfssatz:hilbertwürfel}
		Für den Hilbertwürfel $H$ gelten folgende Aussagen:
		\begin{enumeratethm}
			\item Eine Folge $(x^{(k)})_k \in H^\N$ konvergiert genau dann gegen 
			ein $x \in H$, wenn alle Komponenten konvergieren, also, wenn
			$\lim_{n \to \infty} x_n^{(k)} = x_n$ für alle $n \in \N$ ist.
			\item Setzen wir für $x, y \in H$
			$$\rho(x, y) \defby \max_{n \in \N} \frac{|x_n - y_n|}{2^n} \text{,}$$
			so definiert $\rho$ eine Metrik, die $H$ metrisiert.
			\item $H$ ist kompakt.
		\end{enumeratethm}
	\end{Hilfssatz}
	
	\begin{proof}
		Aussage (a) folgt direkt aus Hilfssatz~\ref{hilfssatz:konvergenz_initialtopologie}.
		
		Für den Beweis von (b) bemerke man zunächst, dass Mengen der Form 
		\[U = \prod_{n=1}^{\infty} U_n\text{,} \quad U_n \subseteq [0, 1] \text{ offen, }
		\quad U_n = [0, 1] \text{ für fast alle } n \label{glg:3.3} \tag{3.3}\]
		eine Basis der Topologie von $H$ bilden.
		
		Offenbar wird durch $\rho$ aus (b) eine Metrik auf $H$ definiert. 
		Es genügt also zu zeigen, dass $\rho$ die Topologie von $H$ induziert. 
		Für $x \in H$ und $r > 0$ ist 
		$B_r(x) = \prod_{n=1}^{\infty} B_{2^n r}(x_n) \subseteq H$ 
		offen. Ist nun $U = \prod_{n=1}^{\infty} U_n$ wie in \eqref{glg:3.3}, 
		so lässt sich leicht einsehen, dass es für jedes $x \in U$ ein $r > 0$ 
		gibt mit $B_r(x) \subseteq U$, also ist $U$ offen bezüglich der von $\rho$ 
		erzeugten Topologie und insgesamt wird $H$ von $\rho$ metrisiert.
		
		Der aus der Topologie bekannte \emph{Satz von Tychonoff} (vgl. \cite[Satz 2.7.1]{Simon.2015}) liefert 
		unmittelbar die Kompaktheit von $H$.
	\end{proof}
	
	\begin{Bemerkung}
		Insbesondere folgt aus Hilfssatz~\ref{hilfssatz:hilbertwürfel}, dass es sich bei $H$ tatsächlich um einen polnischen Raum handelt, 
		denn alle kompakten metrischen Räume sind insbesondere 
		separabel und vollständig und damit auch polnisch.
	\end{Bemerkung}
	
	$H$ erfüllt nun die folgende Universalitätseigenschaft.
	
	\begin{Satz}[Universalitätseigenschaft des Hilbertwürfels]
		\label{satz:universalität_hilbertwürfel}
		Ein topologischer Raum $\X$ ist genau dann separabel und metrisierbar, wenn $\X$ homöomorph zu einer Teilmenge von $H$ ist.
		
		Ferner ist $\X$ genau dann polnisch, wenn $\X$ homöomorph zu einer $G_\delta$-Teilmenge von $H$ ist.
	\end{Satz}
	
	Wir werden den Beweis von Satz~\ref{satz:universalität_hilbertwürfel} simultan für metrisierbare und 
	separable bzw. für polnische Räume führen, benötigen hierfür aber zunächst noch einen Hilfssatz, der uns ermöglichen wird, 
	einen beliebigen separablen und metrischen Raum in den Hilbertwürfel einzubetten.
	
	\begin{Hilfssatz}
		\label{hilfssatz:einbettung_hilbertwürfel}
		Sei $(\X, d)$ ein separabler metrischer Raum und sei 
		$\mathcal{D} \defby \setcomp{x_n}{n \in \N} \subseteq \X$ eine abzählbare 
		dichte Teilmenge. Außerdem setzen wir
		\[\fctmap{\varphi}{\X}{H}{x}{\left(\min (1, d(x, x_n)) \right)_n} \text{.} \label{glg:3.4} \tag{3.4}\]
		Dann ist $\varphi$ ein Homöomorphismus zwischen $\X$ und $\varphi(\X)$.
	\end{Hilfssatz}
	
	\begin{proof}
		Wir zeigen zunächst, dass $\varphi$ injektiv ist. 
		Seien dazu $y, z \in H$ mit $\varphi(y) = \varphi(z)$, also gilt
		\[\min (1, d(y, x_n)) = \min (1, d(z, x_n)) \label{glg:3.5} \tag{3.5}\]
		für alle $n \in \N$. Weil es eine Folge $(y_n)_n \in \mathcal{D}^\N$ mit 
		$y_n \to y$ gibt und diese wegen \eqref{glg:3.5} auch gegen $z$ 
		konvergiert, folgt die Gleichheit von $y$ und $z$.
		
		Die Stetigkeit von $\varphi$ folgt direkt, weil die 
		Komponentenfunktionen $x \mapsto \min (1, d(x, x_n))$ jeweils stetig sind. 
		Seien nun $(y_k)_k \in \X^\N$ und $y \in \X$ mit $\varphi(y_k) \to \varphi(y)$, 
		also
		\[\min (1, d(y_k, x_n)) \; \to \; \min (1, d(y, x_n)), 
		\quad k \to \infty \label{glg:3.6} \tag{3.6}\]
		für alle $n \in \N$. Wählt man nun $\varepsilon < 1$ beliebig, 
		so existiert ein $n \in \N$ mit $d(y, x_n) \leq \varepsilon$. 
		Wegen \eqref{glg:3.6} ist dann auch
		$d(y_k, x_n) \to d(y, x_n), \; k \to \infty$. Es gilt also die Ungleichung
		$$\limsup_{k \to \infty} d(y_k, y) \; \leq \; 
		\limsup_{k \to \infty} d(y_k, x_n) + d(y, x_n) \; \leq \; 2\varepsilon$$
		und mit $\varepsilon \to 0$ folgt die Stetigkeit von $\varphi^{-1}$, 
		womit wir nachgewiesen haben, dass $\varphi$ ein Homöomorphismus auf sein Bild ist. 
	\end{proof}

	Mit Hilfe von Hilfssatz~\ref{hilfssatz:einbettung_hilbertwürfel} können nun die Aussagen von
	Satz~\ref{satz:universalität_hilbertwürfel} nachgewiesen werden.
	
	\begin{proof}[Beweis von Satz~\ref{satz:universalität_hilbertwürfel}]
		Da wir in Hilfssatz~\ref{hilfssatz:einbettung_hilbertwürfel} bereits 
		gesehen haben, dass jeder separable metrisierbare Raum homöomorph zu einer 
		Teilmenge des Hilbertwürfels $H$ ist und jede Teilmenge von $H$ separabel und metrisierbar ist,
		folgt der erste Teil von Satz~\ref{satz:universalität_hilbertwürfel}. 
		
		Sei $\X$ nun ein polnischer Raum und fixiere eine Metrik $d$, die $\X$ vollständig metrisiert. 
		Nach Hilfssatz~\ref{hilfssatz:einbettung_hilbertwürfel} ist $\X$ via $\fct{\varphi}{\X}{H}$ aus \eqref{glg:3.6}
		homöomorph zu einer Teilmenge von $H$ und da $H$ selbst polnisch ist, muss $\varphi(\X) \subseteq H$ nach 
		Satz~\ref{satz:alexandroff} eine $G_\delta$-Teilmenge sein.
		Ferner folgt aus demselben Satz, dass jede $G_\delta$-Teilmenge von $H$ polnisch ist.
	\end{proof}

	Insbesondere lässt sich unter Verwendung von Satz~\ref{satz:universalität_hilbertwürfel} recht mühelos eine weitere Abgeschlossenheitseigenschaft polnischer Räume beweisen.
	
	\begin{Folgerung}
		\label{folgerung:produkte_polnische_räume}
		Sei $\X_n$ für jedes $n \in \N$ ein topologischer Raum. Dann gelten folgende Implikationen:
		\begin{enumeratethm}
			\item Ist $\X_n$ separabel und metrisierbar für jedes $n \in \N$, so ist auch $\prod_{n=1}^{\infty} \X_n$ mit der Produkttopologie separabel und metrisierbar.
			\item Ist $\X_n$ polnisch für jedes $n \in \N$, so ist auch $\prod_{n=1}^{\infty} \X_n$ mit der Produkttopologie polnisch.
		\end{enumeratethm}
	\end{Folgerung}
	
	\begin{proof}
		Wir zeigen zunächst Aussage (a). Seien alle $\X_n$ separabel und metrisierbar. Nach Satz~\ref{satz:universalität_hilbertwürfel} bzw. 
		Hilfssatz~\ref{hilfssatz:einbettung_hilbertwürfel} existieren dann Einbettungen $\fct{\varphi_n}{\X_n}{H}, \; n \in \N$. Nun ist aber 
		\[ \prod_{n=1}^{\infty} \X_n \; \cong \; \prod_{n=1}^{\infty} \varphi_n(\X_n) \; \subseteq \; \prod_{n=1}^{\infty} H \; \cong \; H \]
		und Satz~\ref{satz:universalität_hilbertwürfel} liefert, dass $\prod_{n=1}^{\infty} \X_n$ separabel und metrisierbar ist.
		
		Für Teilaufgabe (b) setzen wir zusätzlich die Polnizität aller $\X_n$ voraus. Dann ist $\varphi_n(\X_n)$ für jedes $n \in \N$ eine $G_{\delta}$-Teilmenge von $H$, also
		existieren für alle $k, n \in \N$ derartige offene Mengen $U_n^{(k)} \subseteq H$, dass $\varphi_n(\X_n) = \bigcap_{k \in \N} U_n^{(k)}$ ist. Da nun insbesondere auch
		\[ V_n^{(k)} \; \defby \; \left( \prod_{m=1}^{n-1} H \times U_n^{(k)} \times \!\!\! \prod_{m=n+1}^{\infty} H \right) \; \subseteq \; \prod_{n=1}^{\infty} H \]
		für alle $k, n \in \N$ offen ist, folgt
		\[ \prod_{n=1}^{\infty} \X_n \; \cong \; \prod_{n=1}^{\infty} \bigcap_{k \in \N} U_n^{(k)} \; = \; \bigcap_{n \in \N} \bigcap_{k \in \N} V_n^{(k)} \; \subseteq \; \prod_{n=1}^{\infty} H \; \cong \; H \text{,} \]
		also ist $\prod_{n=1}^{\infty} \X_n$ homöomorph zu einer $G_{\delta}$-Teilmenge von $H$ und damit nach Satz~\ref{satz:universalität_hilbertwürfel} polnisch.
	\end{proof}
	
	\begin{Bemerkung}
		Die Aussagen aus Folgerung~\ref{folgerung:produkte_polnische_räume} gelten auch für endliche Produkte, denn für alle übrigen $n \in \N$ können wir $\X_n$ in diesem Fall einfach als einen einzigen Punkt wählen.
	\end{Bemerkung}
	
\end{document}