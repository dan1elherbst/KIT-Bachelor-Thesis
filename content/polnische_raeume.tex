\documentclass[../main/main.tex]{subfiles}

\begin{document}
	
	\section{Polnische Räume}
	
	In diesem Kapitel beschäftigen wir uns zunächst mit den grundlegenden Eigenschaften von polnischen Räumen. 
	An erster Stelle möchten wir daher eine Definition dieser Klasse von topologischen Räumen liefern.
	
	\begin{Definition}[Polnischer Raum]
		Ein \emph{polnischer Raum} ist ein separabler und vollständig metrisierbarer topologischer Raum.
	\end{Definition}

	Es sei hier anzumerken, dass dies tatsächlich eine rein topologische Eigenschaft ist: Wir fordern für einen polnischen Raum $X$ nur die Existenz einer Metrik,
	die die Topologie von $X$ erzeugt und bezüglich der $X$ vollständig ist, möchten uns aber die Flexibilität
	bewahren, diese Metrik nicht zu fixieren und bei Bedarf zwischen verschiedenen solchen Metriken
	zu wechseln. Es kann durchaus Metriken geben, die $X$ zwar metrisieren, jedoch nicht vollständig.
	Beispielsweise ist $\R$ ausgestattet mit der Standardtopologie offensichtlich ein polnischer Raum,
	denn $\R$ ist separabel und die euklidische Metrik metrisiert $\R$ vollständig. 
	Allerdings ist $(0, 1)$ als Teilraum von $\R$ mit der euklidischen Metrik zwar separabel, aber nicht vollständig.
	Da $(0, 1)$ homöomorph zu $\R$ ist, ist $(0, 1)$ aber dennoch ein polnischer Raum.

	Der folgende Hilfssatz~\ref{lem:opensets} behandelt die recht elementare Tatsache, dass sich in metrischen Räumen 
	abgeschlossene Mengen von außen \enquote{beliebig gut} durch offene Mengen approximieren lassen. 
	Diese Erkenntnis werden wir im weiteren Verlauf der Arbeit noch häufiger benötigen.
	
	\begin{Hilfssatz}
		\label{lem:opensets}
		Sei $(X, d)$ ein metrischer Raum und $C \subseteq X$ eine abgeschlossene 
		Teilmenge. Ferner definieren wir für $n \in \N$
		$$ A_n \defby \setcomp{y \in X}{d(y, C) < \frac{1}{n}} \quad \text{und} \quad 
		\fctmap{f_n}{X}{\R}{x}{\max \set{0, 1-n d(x, C)}} \text{,}$$
		wobei wir $d(y, C) \defby \inf_{x \in C} d(y, x)$ setzen.
		Dann gilt:
		\begin{enumeratethm}
			\item $A_n$ ist offen für alle $n \in \N$.
			\item $C = \bigcap_{n \in \N} A_n$, insbesondere ist $C$ also eine $G_\delta$-Menge.
			\item Für alle $n$ ist $\restr{f_n}{A_n^\mathsf{c}} = 0$ und $f_n$ ist lipschitzstetig.
			\item $f_n \convdown \indfct_C$.
		\end{enumeratethm}
	\end{Hilfssatz}
	
	\begin{proof}
		Aussage (a) folgt aus der Stetigkeit von $y \mapsto d(y, C)$.
		
		Weiter ist $C \subseteq A_n$ für alle $n \in \N$ und damit 
		$C \subseteq \bigcap_{n \in \N} A_n$. 
		Umgekehrt gibt es für ein beliebiges $y \in \bigcap_{n \in \N} A_n$
		eine Folge $(x_n)_n \in C^\N$ mit $x_n \rightarrow y$. 
		Wegen der Abgeschlossenheit von $C$ liegt $y$ damit in $C$, sodass (b) gezeigt ist.
		
		Aussage (c) ist klar ($f_n$ ist als Komposition 
		lipschitzstetiger Funktionen selbst lipschitzstetig).
		
		Schließlich fällt $f_n$ und für $x \in C$ gilt $f_n(x) = 1$. 
		Für $x \in C^\mathsf{c}$ ist $d(x, C) > 0$ und damit
		$$f_n(x) = \max \set{0, 1-n d(x, C)} 
		\to 0 \text{,} \quad n \to \infty \text{,}$$
		womit die Behauptung folgt.
	\end{proof}

	Mit dem \emph{Hilbertwürfel} möchten wir uns zunächst einen speziellen polnischen Raum ansehen, 
	der sich später in gewisser Weise als universell für alle polnischen Räume erweisen wird.

	\begin{Definition}[Hilbertwürfel]
		Der \emph{Hilbertwürfel} ist der topologische Raum $H = [0, 1]^\N$ 
		ausgestattet mit der Produkttopologie, also der kleinsten Topologie, 
		bezüglich der alle Projektionen $\fctmap{\pi_n}{H}{[0, 1]}{x}{x_n}$ 
		für $n \in \N$ stetig sind.
	\end{Definition}
	
	\begin{Hilfssatz}
		\label{lem:hilbertcube}
		Für den Hilbertwürfel $H$ gelten folgende Aussagen:
		\begin{enumeratethm}
			\item Eine Folge $(x^{(k)})_k \in H^\N$ konvergiert genau dann gegen 
			ein $x \in H$, wenn alle Komponenten konvergieren, also, wenn
			$\lim_{n \to \infty} x_n^{(k)} = x_n$ für alle $n \in \N$ gilt.
			\item Setzen wir für $x, y \in H$
			$$\rho(x, y) \defby \max_{n \in \N} \frac{|x_n - y_n|}{2^n} \text{,}$$
			so definiert $\rho$ eine Metrik, die $H$ metrisiert.
		\end{enumeratethm}
	\end{Hilfssatz}
	
	\begin{proof}
		Die Hinrichtung von (a) folgt aus der Stetigkeit der Projektionen 
		$\pi_n, \; n \in \N$. Für die Rückrichtung bemerke man zunächst, dass Mengen der Form 
		\[U = \prod_{n=1}^{\infty} U_n\text{,} \quad U_n \subseteq [0, 1] \text{ offen, }
		\quad U_n = [0, 1] \text{ für fast alle } n \label{eq:2.1} \tag{2.1}\]
		eine Basis der Topologie von $H$ bilden. 
		Konvergiert nun also $(x^{(k)})_k \in H^\N$ komponentenweise gegen $x \in H$, 
		so enthält jede offene Umgebung von $x$ von der Form \eqref{eq:2.1} auch fast 
		alle Folgenglieder $x^{(k)}, \; k \in \N$ und damit konvergiert 
		auch $(x^{(k)})_k$ gegen $x$.
		
		Offenbar wird durch $\rho$ aus (b) eine Metrik auf $H$ definiert. 
		Es genügt also zu zeigen, dass $\rho$ die Topologie von $H$ induziert. 
		Für $x \in H$ und $r > 0$ ist 
		$B_r(x) = \prod_{n=1}^{\infty} B_{2^n r}(x_n) \subseteq H$ 
		offen. Ist nun $U = \prod_{n=1}^{\infty} U_n$ wie in \eqref{eq:2.1}, 
		so lässt sich leicht einsehen, dass es für jedes $x \in U$ ein $r > 0$ 
		gibt mit $B_r(x) \subseteq U$, also ist $U$ offen bezüglich der von $\rho$ 
		erzeugten Topologie und insgesamt wird $H$ von $\rho$ metrisiert.
	\end{proof}
	
	Der aus der Topologie bekannte \emph{Satz von Tychonoff} liefert 
	unmittelbar, dass $H$ als Produkt kompakter topologischer Räume 
	ebenfalls kompakt ist.
	Außerdem lässt sich unter Ausnutzung von Aussage (a) aus 
	Hilfssatz~\ref{lem:hilbertcube} leicht zeigen, dass $(H, \rho)$ 
	vollständig ist. Weil Mengen der Form \eqref{eq:2.1}, bei denen 
	$U_n$ ausschließlich rationale Intervalle sind, eine abzählbare 
	Basis der Topologie von $H$ bilden, ist $H$ sogar separabel und 
	damit insgesamt ein kompakter polnischer Raum.
	
	$H$ erfüllt nun die folgende Universalitätseigenschaft.
	
	\begin{Satz}[Charakterisierung polnischer Räume]
		\label{thm:characterizationpolishspaces}
		Ein topologischer Raum $X$ ist genau dann polnisch, 
		wenn es eine $G_\delta$-Teilmenge $Y \subseteq H$ gibt, sodass $X$ 
		zu $Y$ homöomorph ist. 
	\end{Satz}

	Wir werden den Beweis von Satz~\ref{thm:characterizationpolishspaces} in seine beiden Teilrichtungen
	aufspalten und jeweils in Hilfssätze auslagern, beginnend mit der Hinrichtung.
	
	\begin{Hilfssatz}
		\label{lem:characterizationpolishspaces}
		Sei $(X, d)$ ein separabler metrischer Raum und sei 
		$D \defby \setcomp{x_n}{n \in \N} \subseteq X$ eine abzählbare 
		dichte Teilmenge. Außerdem setzen wir
		\[\fctmap{\varphi}{X}{H}{x}{\left(\min \left(1, d(x, x_n) \right) \right)_n} \text{.} \label{eq:2.2} \tag{2.2}\]
		Dann gilt das Folgende:
		\begin{enumeratethm}
			\item $\varphi$ ein Homöomorphismus zwischen $X$ und $\varphi(X)$.
			\item Ist $(X, d)$ vollständig, so ist $\varphi(X) \subseteq H$ 
			eine $G_\delta$-Teilmenge.
		\end{enumeratethm}
	\end{Hilfssatz}
	
	\begin{proof}
		Wir zeigen zunächst, dass $\varphi$ injektiv ist. 
		Dazu seien $y, z \in H$ mit $\varphi(y) = \varphi(z)$, also gilt
		\[\min(1, d(y, x_n)) = \min(1, d(z, x_n)) \label{eq:2.3} \tag{2.3}\]
		für alle $n \in \N$. Weil es eine Folge $(y_n)_n \in D^\N$ mit 
		$y_n \to y$ gibt und diese wegen \eqref{eq:2.3} auch gegen $z$ 
		konvergiert, folgt die Gleichheit von $y$ und $z$.
		
		Die Stetigkeit von $\varphi$ folgt direkt, weil die 
		Komponentenfunktionen $x \mapsto \min(1, d(x, x_n))$ jeweils stetig sind. 
		Seien nun $(y_k)_k \in X^\N$ und $y \in X$ mit $\varphi(y_k) \to \varphi(y)$, 
		also
		\[\min(1, d(y_k, x_n)) \; \to \; \min(1, d(y, x_n)), 
		\quad k \to \infty \label{2.4} \tag{2.4}\]
		für alle $n \in \N$. Wählt man nun $\varepsilon < 1$ beliebig, 
		so existiert ein $n \in \N$ mit $d(y, x_n) \leq \varepsilon$. 
		Wegen \eqref{2.4} ist dann auch
		$d(y_k, x_n) \to d(y, x_k), \; k \to \infty$. Es gilt also die Ungleichung
		$$\limsup_{k \to \infty} d(y_k, y) \; \leq \; 
		\limsup_{k \to \infty} d(y_k, x_n) + d(y, x_n) \; \leq \; 2\varepsilon$$
		und mit $\varepsilon \to 0$ folgt die Stetigkeit von $\varphi^{-1}$, 
		womit Aussage (a) gezeigt ist.
		
		Nun beweisen wir noch Aussage (b). Da $\varphi$ ein Homöomorphismus auf sein 
		Bild ist, ist $\varphi(B_{1/k}(x_m)) \subseteq \varphi(X)$ für alle 
		$k, m \in \N$ relativ offen, also gibt es jeweils offene Mengen 
		$V_{k, m} \subseteq H$ mit
		\[\varphi(B_{1/k}(x_m)) \; = \; V_{k, m} 
		\cap \varphi(X) \text{.} \label{eq:2.5} \tag{2.5}\]
		Wir möchten jetzt zeigen, dass
		\[\varphi(X) \; = \; \overline{\varphi(X)} \, \cap \, 
		\bigcap_{k \in \N} \left( \bigcup_{m \in \N} V_{k, m} \right) 
		\label{eq:2.6} \tag{2.6}\]
		gilt. 
		Weil $\overline{\varphi(X)}$ nach Hilfssatz~\ref{lem:opensets} 
		als abgeschlossene Menge eine $G_\delta$-Menge ist, 
		folgt aus \eqref{eq:2.6} direkt die Aussage (b).
		
		Für jedes $m \in \N$ ist sicherlich $\varphi(X) \subseteq 
		\bigcup_{m \in \N} V_{k, m}$, weshalb \enquote{$\subseteq$} 
		in \eqref{eq:2.6} unmittelbar folgt.
		Wir nehmen nun an, dass $z$ in der rechten Seite von \eqref{eq:2.6} liegt. 
		Dann gibt es für jedes $k \in \N$ ein $m_k \in \N$ mit $z \in V_{k, m_k}$. 
		Ferner existiert wegen $z \in \overline{\varphi(X)}$ eine Folge $(y_k)_k \in X^\N$ 
		mit $\varphi(y_k) \to z$, wobei wir zusätzlich
		$\varphi(y_k) \in \bigcap_{j=1}^{k} V_{j, m_j}$
		fordern. Weil damit nach \eqref{eq:2.5} auch für alle $k \in \N$
		\[y_k \in \bigcap_{j=1}^{k} B_{1/k}(x_m)\]
		gilt, ist $(y_k)_k$ eine Cauchyfolge. Aufgrund der Vollständigkeit von 
		$(X, d)$ gibt es also ein $y \in X$ mit $y_k \to y$, was 
		$z = \varphi(y) \in \varphi(X)$ und damit \enquote{$\supseteq$} 
		in \eqref{eq:2.6} impliziert.
	\end{proof}

	Für die Rückrichtung zeigen wir die (nur vermeintlich allgemeinere) Aussage, 
	dass $G_\delta$-Teilmengen polnischer Räume wieder polnisch sind.
	
	\begin{Hilfssatz}
		\label{lem:characterizationpolishspaces2}
		Sei $X$ ein polnischer Raum. Dann ist jede $G_\delta$-Teilmenge 
		von $X$ (ausgestattet mit der Teilraumtopologie) ebenfalls polnisch.
	\end{Hilfssatz}
	
	\begin{proof}
		Sei $P \subseteq X$ eine $G_\delta$-Teilmenge und 
		$U_n \subseteq X, \, n \in \N$ offene Mengen mit 
		$P = \bigcap_{n \in \N} U_n$, wobei wir außerdem 
		$C_n \defby U_n^{\mathsf{c}}$ schreiben. Ferner sei 
		$d$ eine Metrik, die $X$ vollständig metrisiert. 
		Da $P$ als Teilmenge eines separablen metrischen Raumes 
		selbst separabel ist, genügt es, die vollständige Metrisierbarkeit 
		von $P$ zu beweisen.
		
		Für $x, y \in P$ definieren wir
		\[d^\ast(x, y) \; \defby \; d(x, y) + \sum_{n=1}^{\infty} \min \left(
		\frac{1}{2^n}, \left| \frac{1}{d(x, C_n)} - \frac{1}{d(y, C_n)} \right|
		\right) \text{.}\]
		Offensichtlich ist $d^\ast$ eine Metrik auf $P$, 
		die dieselbe Topologie wie $d$ erzeugt. 
		
		Wir möchten nun zeigen, dass $(P, d^\ast)$ vollständig ist. 
		Sei dazu $(x_k)_k \in P^\N$ eine Cauchyfolge. Wegen $d \leq d^\ast$ 
		ist $(x_k)_k$ dann auch bezüglich $d$ eine Cauchyfolge und die 
		Vollständigkeit von $(X, d)$ liefert die Existenz eines Grenzwertes 
		$x \in X$. Angenommen $x \notin P$, so gibt es ein $m \in \N$ mit 
		$x \in C_m$. Damit ist aber für alle $k \in \N$
		$$\sup_{l \geq k} d^\ast(x_k, x_l) \; \geq \; \sup_{l \geq k} 
		\, \sum_{n=1}^{\infty} \min \left(
		\frac{1}{2^n}, \left| \frac{1}{d(x_k, C_n)} -
		\frac{1}{d(x_l, C_n)} \right|
		\right) \; \geq \; \frac{1}{2^m} \; > \; 0 \text{,}$$
		was der Tatsache widerspricht, dass $(x_k)_k$ eine Cauchyfolge 
		bezüglich $d^\ast$ ist. Also muss $x \in P$ gelten. 
		Aus der Stetigkeit von $x \mapsto d(x, C_n)$ folgt mit dem 
		Satz von Lebesgue unmittelbar die Konvergenz 
		$d^\ast(x_k, x) \to 0$ für $k \to \infty$.
	\end{proof}
	
	Da wir in Hilfssatz~\ref{lem:characterizationpolishspaces} bereits 
	gesehen haben, dass jeder polnische Raum homöomporph zu einer 
	$G_\delta$-Teilmenge vom Hilbertwürfel $H$ ist und wir durch 
	Lemma~\ref{lem:characterizationpolishspaces2} nun auch wissen, 
	dass jede $G_\delta$-Teilmenge von $H$ tatsächlich polnisch ist, 
	haben wir nun also insgesamt Satz~\ref{thm:characterizationpolishspaces}
	bewiesen.
	
	\tobechanged{TODO: Überlegen, ob man das beweisen oder nur zitieren soll und ggf. anpassen.}
	
	\begin{Hilfssatz}
		\label{lem:hilbertcubefunctionseparability}
		$(C(H), \norm{\cdot}_\infty)$ ist separabel.
	\end{Hilfssatz}
	
	\begin{proof}
		Wir definieren zunächst 
		\[\mathcal{D} \; \defby \; \mathrm{span}_\R 
		\setcomp{\fctmap{f}{H}{\R}{x
			}{\prod_{j=1}^{n} f_j(x_j)}}{n \in \N, \; f_1,
			\dots,f_n \in C([0, 1])} \text{.}\]
		Weil $\mathcal{D}$ eine punktetrennende reelle 
		Unteralgebra von $C(H)$ ist, für die keine 
		Auswertungsfunktion $\mathcal{D} \ni f \mapsto f(x), 
		\; x \in H$ konstant $0$ ist, lässt sich mit dem 
		Satz von Stone-Weierstraß folgern, dass $\mathcal{D} \subseteq C(H)$ 
		dicht bezüglich $\norm{\cdot}_\infty$ ist.
		Außerdem ist $(C([0, 1]), \norm{\cdot}_\infty)$ separabel, 
		also gibt es eine abzählbare, dichte Teilmenge 
		$\mathcal{P} \subseteq C([0, 1])$. Schließlich setzen wir
		\[\mathcal{D}^\ast \; \defby \; \mathrm{span}_\Q 
		\setcomp{\fctmap{f}{H}{\R}{x}{\prod_{j=1}^{n} f_j(x_j)}}
		{n \in \N, \; f_1,\dots,f_n \in \mathcal{P}} \text{.}\]
		Dann ist $\mathcal{D}^\ast$ abzählbar und wegen 
		$ \mathcal{D} \subseteq \overline{\mathcal{D}^\ast} 
		\subseteq C(H)$ gilt die Gleichheit $\overline{\mathcal{D}^\ast} = C(H)$. 
		Demnach ist $(C(H), \norm{\cdot}_\infty)$ also separabel.
	\end{proof}
	
	\begin{Bemerkung}
		Analog kann man den obigen Beweis allgemeiner für stetige Funktionen 
		auf einem abzählbaren Produkt kompakter Hausdorffräume, deren 
		stetige Funktionen jeweils separabel bezüglich $\norm{\cdot}_\infty$ 
		sind, führen. Das Vorgehen wird in \cite[Lemma 4.12.1 und 
		Proposition 4.12.2]{Simon.2015} skizziert.
	\end{Bemerkung}
	
\end{document}